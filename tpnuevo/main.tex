\documentclass[a4paper,10pt]{article}
\usepackage[paper=a4paper, hmargin=1.5cm, bottom=1.5cm, top=3.5cm]{geometry}
\usepackage[latin1]{inputenc}
\usepackage[T1]{fontenc}
\usepackage[spanish]{babel}
\usepackage[framemethod=tikz]{mdframed}
\usepackage[T1]{fontenc}
\usepackage{xspace}
\usepackage{xargs}
\usepackage{ifthen}
\usepackage{fancyhdr}
\usepackage{lastpage}
\usepackage{aed2-tad,aed2-symb,aed2-itef}
\usepackage{enumitem}
\usepackage{algorithm}
\usepackage{algpseudocode}
\usepackage{scrextend}
\usepackage{framed}
%\usepackage[noend]{algpseudocode}

\newcommand{\moduloNombre}[1]{\textbf{#1}}

\let\NombreFuncion=\textsc
\let\TipoVariable=\texttt
\let\ModificadorArgumento=\textbf
\newcommand{\res}{$res$\xspace}
\newcommand{\tab}{\hspace*{7mm}}

\newcommandx{\TipoFuncion}[3]{%
  \NombreFuncion{#1}(#2) \ifx#3\empty\else $\to$ \res\,: \TipoVariable{#3}\fi%
}
\newcommand{\In}[2]{\ModificadorArgumento{in} \ensuremath{#1}\,: \TipoVariable{#2}\xspace}
\newcommand{\Out}[2]{\ModificadorArgumento{out} \ensuremath{#1}\,: \TipoVariable{#2}\xspace}
\newcommand{\Inout}[2]{\ModificadorArgumento{in/out} \ensuremath{#1}\,: \TipoVariable{#2}\xspace}
\newcommand{\Aplicar}[2]{\NombreFuncion{#1}(#2)}

\newlength{\IntFuncionLengthA}
\newlength{\IntFuncionLengthB}
\newlength{\IntFuncionLengthC}
%InterfazFuncion(nombre, argumentos, valor retorno, precondicion, postcondicion, complejidad, descripcion, aliasing)
\newcommandx{\InterfazFuncion}[9][4=true,6,7,8,9]{%
  \hangindent=\parindent
  \TipoFuncion{#1}{#2}{#3}\\%
  \textbf{Pre} $\equiv$ \{#4\}\\%
  \textbf{Post} $\equiv$ \{#5\}%
  \ifx#6\empty\else\\\textbf{Complejidad:} #6\fi%
  \ifx#7\empty\else\\\textbf{Descripci�n:} #7\fi%
  \ifx#8\empty\else\\\textbf{Aliasing:} #8\fi%
  \ifx#9\empty\else\\\textbf{Requiere:} #9\fi%
}

\newenvironment{Interfaz}{%
  \parskip=2ex%
  \noindent\textbf{\Large Interfaz}%
  \par%
}{}

\newenvironment{Representacion}{%
  \vspace*{2ex}%
  \noindent\textbf{\Large Representaci�n}%
  \vspace*{2ex}%
}{}

\newenvironment{Algoritmos}{%
  \vspace*{2ex}%
  \noindent\textbf{\Large Algoritmos}%
  \vspace*{2ex}%
}{}

\newcommand{\Titulo}[1]{
  \vspace*{1ex}\par\noindent\textbf{\large #1}\par
}

\newenvironmentx{Estructura}[2][2={estr}]{%
  \par\vspace*{2ex}%
  \TipoVariable{#1} \textbf{se representa con} \TipoVariable{#2}%
  \par\vspace*{1ex}%
}{%
  \par\vspace*{2ex}%
}%

\newboolean{EstructuraHayItems}
\newlength{\lenTupla}
\newenvironmentx{Tupla}[1][1={estr}]{%
    \settowidth{\lenTupla}{\hspace*{3mm}donde \TipoVariable{#1} es \TipoVariable{tupla}$($}%
    \addtolength{\lenTupla}{\parindent}%
    \hspace*{3mm}donde \TipoVariable{#1} es \TipoVariable{tupla}$($%
    \begin{minipage}[t]{\linewidth-\lenTupla}%
    \setboolean{EstructuraHayItems}{false}%
}{%
    $)$%
    \end{minipage}
}

\newenvironmentx{Enum}[1][1={estr}]{%
    \settowidth{\lenTupla}{\hspace*{3mm}donde \TipoVariable{#1} es \TipoVariable{enum}$($}%
    \addtolength{\lenTupla}{\parindent}%
    \hspace*{3mm}donde \TipoVariable{#1} es \TipoVariable{enum}$($%
    \begin{minipage}[t]{\linewidth-\lenTupla}%
    \setboolean{EstructuraHayItems}{false}%
}{%
    $)$%
    \end{minipage}
}

\newcommandx{\tupItem}[3][1={\ }]{%
    %\hspace*{3mm}%
    \ifthenelse{\boolean{EstructuraHayItems}}{%
        ,#1%
    }{}%
    \emph{#2}: \TipoVariable{#3}%
    \setboolean{EstructuraHayItems}{true}%
		\ignorespaces%
}

\newcommandx{\tupTupItem}[3][1={\ }]{%
    %\hspace*{3mm}%
    \ifthenelse{\boolean{EstructuraHayItems}}{%
       ,#1%
    }{}%
    \emph{#2}: #3%
    \setboolean{EstructuraHayItems}{true}%
		\ignorespaces%
}

\newcommandx{\enumItem}[2][1={\ }]{%
    %\hspace*{3mm}%
    \ifthenelse{\boolean{EstructuraHayItems}}{%
        ,#1%
    }{}%
    \TipoVariable{#2}%
    \setboolean{EstructuraHayItems}{true}%
		\ignorespaces%
}

\newcommandx{\RepFc}[3][1={estr},2={e}]{%
  \tadOperacion{Rep}{#1}{bool}{}%
  \tadAxioma{Rep($#2$)}{#3}%
}%

\newcommandx{\Rep}[3][1={estr},2={e}]{%
  \tadOperacion{Rep}{#1}{bool}{}%
  \tadAxioma{Rep($#2$)}{true \ssi #3}%
}%

\newcommandx{\Abs}[5][1={estr},3={e}]{%
  \tadOperacion{Abs}{#1/#3}{#2}{Rep($#3$)}%
  \settominwidth{\hangindent}{Abs($#3$) \igobs #4: #2 $\mid$ }%
  \addtolength{\hangindent}{\parindent}%
  Abs($#3$) \igobs #4: #2 $\mid$ #5%
}%

\newcommandx{\AbsFc}[4][1={estr},3={e}]{%
  \tadOperacion{Abs}{#1/#3}{#2}{Rep($#3$)}%
  \tadAxioma{Abs($#3$)}{#4}%
}%

\newcommand{\DRef}{\ensuremath{\rightarrow}}

\newcommandx{\ComplejidadDer}[1]{
	\hfill \ensuremath{\mathcal{O}(#1)}
}

\newcommandx{\Complejidad}[1]{
	\ensuremath{\mathcal{O}(#1)}
}

\renewcommand{\thealgorithm}{}

\newenvironmentx{Algoritmo}[6]{
	%\begin{algorithm}
	
	\begin{mdframed}
		
		%\floatname{algorithm}{Algoritmo}
		%\caption{\TipoFuncion{#1}{#2}{#3}}
		\TipoFuncion{#1}{#2}{#3}
		\begin{algorithmic}[1]
			#4
		\end{algorithmic}

		\begin{flushleft}
			%\begin{framed}
				%\flushleft
				\textbf{Complejidad:} #5 \\
				\vspace{0.75em}
					#6
			%\end{framed}
		\end{flushleft}
		
	\end{mdframed}
	
	%\end{algorithm}
}

\fancypagestyle{caratula} {
   \fancyhf{}
   \cfoot{\thepage /\pageref{LastPage}}
   \renewcommand{\headrulewidth}{0pt}
   \renewcommand{\footrulewidth}{0pt}
}

\begin{document}

%%%%%%%%%%%%%%%%%%%%%%%%%%%%%%%%%%%%%%%%%%%%%%%%%%%%%%%%%%%%%%%%%%%%%%%%%%%%%%%
%% Car�tula                                                                  %%
%%%%%%%%%%%%%%%%%%%%%%%%%%%%%%%%%%%%%%%%%%%%%%%%%%%%%%%%%%%%%%%%%%%%%%%%%%%%%%%

\thispagestyle{caratula}

\begin{center}

\vspace*{1cm}

\begin{Huge}
Algoritmos y Estructuras de Datos II
\end{Huge}

\vspace{1cm}

\begin{LARGE}
Trabajo Pr�ctico 2
\end{LARGE}

\vspace{1cm}

Departamento de Computaci�n,\\
Facultad de Ciencias Exactas y Naturales,\\
Universidad de Buenos Aires

\vspace{1cm}

Segundo Cuatrimestre de 2014

\vspace{1cm}

\begin{Large}
Grupo 16
\end{Large}

\vspace{0.5cm}

\begin{tabular}{|c|c|c|}
\hline
Apellido y Nombre & LU & E-mail\\
\hline
Juan Ernesto Rinaudo        & 864/13 & jangamesdev@hotmail.com\\
Mauro Cherubini             & 835/13 & cheru.mf@gmail.com\\
Federico Beuter             & 827/13 & federicobeuter@gmail.com\\
Fernando Frassia            & 340/13 & ferfrassia@gmail.com\\
\hline
\end{tabular}

\vspace{1cm}

Reservado para la c�tedra

\begin{tabular}{|c|c|c|}
\hline
Instancia & Docente que corrigi� & Calificaci�n\\
\hline
Primera Entrega &&\\
\hline
Recuperatorio   &&\\
\hline
\end{tabular}

\end{center}

\vspace{6cm}

\newpage


\tableofcontents


%%%%%%%%%%%%%%%%%%%%%%%%%%%%%%%%%%%%%%%%%%%%%%%%%%%%%%%%%%%%%%%%%%%%%%%%%%%%%%%%%%%%%%%%%%%%%%%%%%%%%
%
%
%
% ACA EMPIEZA EL CODIGO DEL LATEX
%
%
%
%%%%%%%%%%%%%%%%%%%%%%%%%%%%%%%%%%%%%%%%%%%%%%%%%%%%%%%%%%%%%%%%%%%%%%%%%%%%%%%%%%%%%%%%%%%%%%%%%%%%%

\newpage
\section{Tad Extendidos}	

\subsection{Secu($\alpha$)}

\vspace{1em}

\tadOtrasOperaciones
\tadOperacion{elemDeSecu}{Secu($\alpha$)/s,Nat/n}{RUR}{n $<$ long($s$)}

\vspace{1em}

\tadAxiomas
\tadAxioma{elemDeSecu(s, n)}{\IF $n =$ 0 THEN prim($s$) ELSE elemDeSecu(fin(s), n-1) FI}

\vspace{1em}

\subsection{Mapa}

\vspace{1em}

\tadObservadores
\tadOperacion{restricciones}{Mapa/m}{secu(restriccion)}{}
\tadOperacion{nroConexion}{estacion/e_1, estacion/e_2, Mapa/m}{nat}{{$e_1, e_2$} $\subset$ estaciones(m) $\yluego$ conectadas?($e_1, e_2, m$)}

\vspace{1em}

\tadAxiomas
\tadAxioma{restricciones(vacio)}{$\langle$~$\rangle$}
\tadAxioma{restricciones(agregar($e, m$))}{restricciones($m$)}
\tadAxioma{restricciones(conectar($e_1, e_2, r, m$))}{restricciones($m$) $\circ$ $r$}
\tadAxioma{nroConexion($e_1, e_2, $conectar($e_3, e_4, m$))}{
	\IF ($(e_1 = e_3 \wedge e_2 = e_4) \vee (e_1 = e_4 \wedge e_2 = e_3)$) THEN long(restricciones($m$)) - 1 ELSE nroConexion($e_1, e_2, m$) - 1 FI
}
\tadAxioma{nroConexion($e_1, e_2, $agregar($e, m$))}{nroConexion($e_1, e_2, m$)}

\vspace{1em}

\subsection{Diccionario(Clave, Significado)}

\vspace{1em}

\tadOtrasOperaciones
\tadAlinearFunciones{claveMax}{Diccionario/d}
\tadOperacion{vac�o?}{Diccionario}{Bool}{}
\tadOperacion{claveMax}{Diccionario/d}{Clave}{$\neg$vac�o(d)}
\tadOperacion{secuClaves}{Diccionario}{Secu(clave)}{}

\vspace{1em}

\tadAxiomas
\tadAlinearAxiomas{secuClaves(definir(c, s, d))}
\tadAxioma{vac�o?(vac�o)}{true}
\tadAxioma{vac�o?(definir(c, s, d))}{false}
\tadAxioma{claveMax(d)}{elemMax(claves(d))}
\tadAxioma{secuClaves(vac�o)}{$\textless$$\textgreater$}
\tadAxioma{secuClaves(definir(c, s, d))}{secuClaves(d) $\circ$ c }

\vspace{1em}

\subsection{Conjunto($\alpha$$\textless$)}

\vspace{1em}

\tadOtrasOperaciones
\tadAlinearFunciones{auxMaxElem}{$\alpha$/e, Conj($\alpha$)/c}
\tadOperacion{elemMax}{Conj($\alpha$)/c}{$\alpha$}{$\neg$$\emptyset$?(c)}
\tadOperacion{auxElemMax}{$\alpha$, Conj($\alpha$)}{$\alpha$}{}

\vspace{1em}

\tadAxiomas
\tadAlinearAxiomas{auxMaxElem(e, c)}
\tadAxioma{elemMax(c)}{auxMaxElem(dameUno(c), c)}
\tadAxioma{auxElemMax(e, c)}{\IF $\emptyset?$(c) THEN e ELSE {\IF e $\textgreater$ dameUno(c) THEN auxElemMax(e, sinUno(c)) ELSE auxElemMax(dameUno(c), sinUno(c)) FI} FI}




\newpage
\section{Red}

\begin{Interfaz}
  
  %\textbf{par�metros formales}\hangindent=2\parindent\\
  %\parbox{1.7cm}{\textbf{g�neros}} \\
  %\parbox[t]{1.7cm}{\textbf{funci�n}}\parbox[t]{\textwidth-2\parindent-1.7cm}{%
    %\InterfazFuncion{Verifica?}{\In{c}{conj(tag)}, \In{r}{rest}}{$\alpha$}
    %{$res \igobs a$}
    %[$\Theta(copy(a))$]
    %[funci�n de copia de $\alpha$'s]
  %}

  \textbf{se explica con}: \tadNombre{Red, Iterador Unidireccional($\alpha$)}.

  \textbf{g�neros}: \TipoVariable{red, itConj(Compu)}.

  \Titulo{Operaciones b�sicas de Red}

  %\InterfazFuncion{NOMBRE}{INPUTS}{TIPO RES}%
  %[ACA VA EL PRE (SI LO HAY)]
  %{ACA VA EL POST}%
  %[$\Theta(COMPLEJIDAD)$]
  %[DESCRIPCION]

  \InterfazFuncion{Computadoras}{\In{r}{red}}{itConj(Compu)}
  %[ACA VA EL PRE (SI LO HUBIERA)]
  {$res \igobs$ crearIt(computadoras($r$))}
  [$\Complejidad{1}$]
  [Devuelve las computadoras de red.]
	
  \InterfazFuncion{Conectadas?}{\In{r}{red}, \In{c_1}{compu}, \In{c_2}{compu}}{bool}
  [\{$c_1,c_2$\} $\subseteq$ computadoras($r$)]
  {$res \igobs$ conectadas?($r, c_1, c_2$)}
  [$\Complejidad{|c_1| + |c_2|}$]
  [Devuelve el valor de verdad indicado por la conexi�n o desconexi�n de dos computadoras.]

  \InterfazFuncion{InterfazUsada}{\In{r}{red}, \In{c_1}{compu}, \In{c_2}{compu}}{interfaz}
  [\{$c_1, c_2$\} $\subseteq$ computadoras($r$) $\yluego$ conectadas?($r, c_1, c_2$)]
  {$res \igobs$ interfazUsada($r, c_1, c_2$)}
  [$\Complejidad{|c_1| + |c_2|}$]
  [Devuelve la interfaz que $c_1$ usa para conectarse con $c_2$]

  \InterfazFuncion{IniciarRed}{}{red}
  %[ACA VA EL PRE (SI LO HAY)]
  {$res \igobs$ iniciarRed()}
  [$\Complejidad{1}$]
  [Crea una red sin computadoras.]
	
  \InterfazFuncion{AgregarComputadora}{\Inout{r}{red}, \In{c}{compu}}{}
  [$r_0 \igobs r$ $\wedge$ $\neg$($c$ $\in$computadoras($r$))]
  {$r$ $\igobs$ agregarComputadora($r_0, c$)}
  [$\Complejidad{|c|}$]
  [Agrega una computadora a la red.]
  
   \InterfazFuncion{Conectar}{\Inout{r}{red}, \In{c_1}{compu}, \In{i_1}{interfaz}, \In{c_2}{compu}, \In{i_2}{interfaz}}{}
  [$r_0$ $\igobs$ r $\wedge$ \{$c_1,c_2$\} $\subseteq$ computadoras($r$) $\wedge$ ip($c_1$) $\neq$ ip($c_2$) $\yluego$ $\neg$ conectadas?($r, c_1, c_2$) $\wedge$ $\neg$ usaInterfaz?($r, c_1, i_1$) $\wedge$ $\neg$ usaInterfaz?($r, c_2, i_2$)]
  {$r$ $\igobs$ conectar($r, c_1, i_1, c_2, i_2$)}
  [$\Complejidad{|c_1| + |c_2|}$]
  [Conecta dos computadoras y les a�ade la interfaz correspondiente.]

  \InterfazFuncion{Vecinos}{\In{r}{red}, \In{c}{compu}}{itConj(compu)}
  [$c$ $\in$ computadoras($r$)]
  {$res$ $\igobs$ crearIt(vecinos(r, c))}
  []
  [Devuelve todas las computadoras que est�n conectadas directamente con c]

  \InterfazFuncion{UsaInterfaz?}{\In{r}{red}, \In{c}{compu}, \In{i}{interfaz}}{bool}
  [$c$ $\in$ computadoras($r$)]
  {$res$ $\igobs$ usaInterfaz?(r, c, i)}
  []
  [Verifica que una computadora use una interfaz]

  \InterfazFuncion{CaminosMinimos}{\In{r}{red}, \In{c_1}{compu}, \In{c_2}{compu}}{conj(lista(compu))}
  [\{$c_1,c_2$\} $\subseteq$ computadoras($r$)]
  {$res$ $\igobs$ caminosMinimos(r, $c_1$, $c_2$)}
  []
  [Devuelve todos los caminos minimos de conexiones entre una computadora y otra]

  \InterfazFuncion{HayCamino?}{\In{r}{red}, \In{c_1}{compu}, \In{c_2}{compu}}{bool}
  [\{$c_1,c_2$\} $\subseteq$ computadoras($r$)]
  {$res$ $\igobs$ hayCamino?(r, $c_1$, $c_2$)}
  []
  [Verifica que haya un camino de conexiones entre una computadora y otra]
  
\end{Interfaz}

\subsection{Auxiliares}

\begin{Auxiliares}

  \Titulo{Operaciones auxiliares}

  \InterfazFuncion{CaminosMinimos}{\In{r}{red}, \In{c_1}{compu}, \In{c_2}{compu}}{conj(lista(compu))}
  [\{$c_1,c_2$\} $\subseteq$ computadoras($r$)]
  {$res$ $\igobs$ caminosMinimos(r, $c_1$, $c_2$)}
  []
  [Devuelve los caminos m�nimos entre c_1 y c_2]

\end{Auxiliares}

\subsection{Representacion}

\begin{Representacion}

%ACA VA LA DESCRIPCION DE LA CACONA DE LA RED

\bigskip
\begin{Estructura}{red}[e\_red]
	\begin{Tupla}[e\_red]
		\tupTupItem{vecinosEInterfaces}{\TipoVariable{diccString}$($\ignorespaces
			% NOTA:tupTupItem es un engendro que agregue para este caso, NO USAR EN OTRO LADO, USAR tupItem EN SU LUGAR.
			\emph{compu}: \TipoVariable{string},
			tupla(
			\emph{interfaces}: \TipoVariable{diccString}$($\ignorespaces
				\emph{compu}: \TipoVariable{string}, \ignorespaces
				\emph{interfaz}: \TipoVariable{nat}\ignorespaces
			$)$,
			\emph{compusVecinas}: \TipoVariable{conj(compu)}
			$) )$} \\
			\tupTupItem{deOrigenADestino}{\TipoVariable{diccString}$($\ignorespaces
			\emph{compu}: \TipoVariable{string},
			\TipoVariable{diccString}$($\ignorespaces
				\emph{compu}: \TipoVariable{string}, \ignorespaces
				\emph{caminosMinimos}: \TipoVariable{conj(lista(compu))}\ignorespaces
			$) )$} \\
			\tupItem{computadoras}{\TipoVariable{conj(compu)}}
	\end{Tupla}
	
\end{Estructura}

\subsection{InvRep y Abs}

\begin{enumerate}
	\item{El conjunto de claves de ''uniones'' es igual al conjunto de estaciones ''estaciones''.}
	\item{''\#sendas'' es igual a la mitad de las horas de ''uniones''.}
	\item{Todo valor que se obtiene de buscar el significado del significado de cada clave de ''uniones'', es igual el valor hallado tras buscar en ''uniones'' con el sinificado de la clave como clave y la clave como significado de esta nueva clave, y no hay otras hojas ademas de estas dos, con el mismo valor.}
	\item{Todas las hojas de ''uniones'' son mayores o iguales a cero y menores a ''\#sendas''.}
	\item{La longitud de ''sendas'' es mayor o igual a ''\#sendas''.}
\end{enumerate}

\Rep[e\_mapa][m]{
	\\m.estaciones = claves(m.uniones) $\wedge$ \hfill 1.
	\\m.\#sendas = \#sendasPorDos(m.estaciones, m.uniones) / 2 $\wedge$ m.\#sendas $\leq$ long(m.sendas) $\yluego$ \hfill 2. 5.
	\\($\forall$ e1, e2: string)(e1 $\in$ claves(m.uniones) $\yluego$ e2 
	$\in$ claves(obtener(e1, m.uniones)) $\impluego$\\ 
	e2 $\in$ claves(m.uniones) $\yluego$ e1 $\in$ claves(obtener(e2, m.uniones)) $\yluego$ 
	\\ obtener(e2, obtener(e1, m.uniones)) = obtener(e1, obtener(e2, m.uniones)) $\wedge$ \hfill 3. 4.	
	\\ obtener(e2, obtener(e1, m.uniones)) $<$ m.\#sendas) $\wedge$
	\\($\forall$ e1, e2, e3, e4: string)((e1 $\in$ claves(m.uniones) $\yluego$
	e2 $\in$ claves(obtener(e1, m.uniones)) $\wedge$\\ 
	e3 $\in$ claves(m.uniones) $\yluego$ e4 $\in$ claves(obtener(e3, m.uniones))) $\impluego$
	\\ (obtener(e2, obtener(e1, m.uniones)) $=$ obtener(e4, obtener(e3, m.uniones)) $\ssi$
	\\ (e1 = e3 $\wedge$ e2 = e4) $\vee$ (e1 = e4 $\wedge$ e2 = e3)))) \hfill 3.
}

\vspace{2em}
	
\tadOperacion{\#sendasPorDos}{conj($\alpha$)\ c, dicc($\alpha$, dicc($\alpha$, $\beta$))\ d}
														{nat}{c $\subset$ claves(d)}

\vspace{1em}

\tadAxioma{\#sendasPorDos(c, d)}{\IF $\emptyset$?(c) THEN 0
												ELSE \#claves(obtener(dameUno(c),d)) $+$ \#sendasPorDos(sinUno(c), d)
												FI}

\vspace{2em}

\Abs[e\_mapa]{mapa}[m]{p}{
	\\ m.estaciones = estaciones(p) $\yluego$
	\\ ($\forall$ e1, e2: string)((e1 $\in$ estaciones(p) $\wedge$ e2 $\in$ estaciones(p)) $\impluego$
	\\ (conectadas?(e1, e2, p) $\ssi$
	\\   e1 $\in$ claves(m.uniones) $\wedge$ e2 $\in$ claves(obtener(e2, m.uniones)))) $\yluego$
	\\ ($\forall$ e1, e2: string)((e1 $\in$ estaciones(p) $\wedge$ e2 $\in$ estaciones(p)) $\yluego$
	\\ conectadas?(e1, e2, p) $\impluego$ 
	\\ (restriccion(e1, e2, p) = m.sendas[obtener(e2, obtener(e1, m.uniones))] $\wedge$
	   nroConexion(e1, e2, m) = obtener(e2, obtener(e1, m.uniones))) $\wedge$
	   long(restricciones(p)) = m.\#sendas $\yluego$
		 ($\forall$ n:nat) (n < m.\#sendas $\impluego$ m.sendas[n] = ElemDeSecu(restricciones(p), n)))
}

\end{Representacion}

\subsection{Algoritmos}

\begin{Algoritmos}
	
	%Algoritmos / Inputs / TSalida / Codigo / Complejidad Final / Justificacion
	\begin{Algoritmo}{iComputadoras}{\In{r}{red}}{itConj(Compu)}
	{
		\State $res$ $\gets$ CrearIt($r.computadoras$) \ComplejidadDer{1}
	} 
	{$\Complejidad{1}$}{}
	\end{Algoritmo}
	
	\begin{Algoritmo}{iConectadas?}{\In{r}{red}, \In{c_1}{compu}, \In{c_2}{compu}}{bool}
	{
		\State $res$ $\gets$ Definido?(Significado($r.vecinosEInterfaces$, $c_1.ip$)$.interfaces$, $c_2.ip$) \ComplejidadDer{|c_1| + |c_2|}
	}
	{$\Complejidad{|c_1| + |c_2|}$}{}
	\end{Algoritmo}
	
	\begin{Algoritmo}{iInterfazUsada}{\In{r}{red}, \In{c_1}{compu}, \In{c_2}{compu}}{interfaz}
	{
		\State $res$ $\gets$ Significado(Significado($r.vecinosEInterfaces$, $c_1.ip$)$.interfaces$, $c_2.ip$)\ComplejidadDer{|c_1| + |c_2|}
	}
	{$\Complejidad{|c_1| + |c_2|}$}{}
	\end{Algoritmo}
	
	\begin{Algoritmo}{iIniciarRed}{}{red}
	{
		\State $res$ $\gets$ tupla($vecinosEInterfaces$: Vac�o(), $deOrigenADestino$: Vac�o(), $computadoras$: Vac�o()) \ComplejidadDer{1 + 1 + 1}
	}
	{$\Complejidad{1}$}
	{$\Complejidad{1}  + \Complejidad{1} + \Complejidad{1} = \newline
	  3 * \Complejidad{1} = \Complejidad{1}$}
	\end{Algoritmo}
	
	\begin{Algoritmo}{iAgregarComputadora}{\Inout{r}{red}, \In{c}{compu}}{}
	{
		\State Agregar($r.computadoras$, $c$) \ComplejidadDer{1}
		\State Definir($r.vecinosEInterfaces$, $c.ip$, tupla(Vac�o(), Vacio())) \ComplejidadDer{|c|}
		\State Definir($r.deOrigenADestino$, $c.ip$, Vac�o()) \ComplejidadDer{|c|}
	}
	{$\Complejidad{|c|}$}
	{$\Complejidad{1} + \Complejidad{|c|} + \Complejidad{|c|} = \newline
	2 * \Complejidad{|c|} = \Complejidad{|c|}$}
	\end{Algoritmo}
	
	\begin{Algoritmo}{iConectar}{\Inout{r}{red}, \In{c_1}{compu}, \In{i_1}{interfaz}, \In{c_2}{compu}, \In{i_2}{interfaz}}{}
	{
		%Armo la primera componente de e_red
		
		\State var $tupSig1$:tupla $\gets$ Significado($r.vecinosEInterfaces$, $c_1.ip$)
		%Guardo la interfaz usada por c_1
		\State Definir($tupSig1.interfaces$, $c_2.ip$, $i_1$) \ComplejidadDer{|c_1| + |c_2| + 1}
		%Agrego c_2 a las compusVecinas de c_1
		\State Agregar($tupSig1.compusVecinas$, $c_2$) \ComplejidadDer{1}

		\State var $tupSig2$:tupla $\gets$ Significado($r.vecinosEInterfaces$, $c_2.ip$)
		%Guardo la interfaz usada por c_2		
		\State Definir($tupSig2.interfaces$, $c_1.ip$, $i_2$) \ComplejidadDer{|c_1| + |c_2| + 1}
		%Agrego c_2 a las compusVecinas de c_1
		\State Agregar($tupSig2.compusVecinas$, $c_1$) \ComplejidadDer{1}
		

		%Armo la segunda componente de e_red
				
		%Guardo los caminos m�nimos de c_1 a c_2
		\State Definir(Significado($r.deOrigenADestino$, $c_1.ip$), $c_2.ip$, CaminosMinimos(r, c_1, c_2)) \ComplejidadDer{1}
		%Guardo los caminos m�nimos de c_2 a c_1
		\State Definir(Significado($r.deOrigenADestino$, $c_2.ip$), $c_1.ip$, CaminosMinimos(r, c_2, c_1)) \ComplejidadDer{1}		
	}
	{$\Complejidad{|e_1| + |e_2|}$}
	{$\Complejidad{|e_1| + |e_2|} + \Complejidad{|e_1| + |e_2|} + \Complejidad{1} + \Complejidad{1} = \newline
	  2 * \Complejidad{1} + 2 * \Complejidad{|e_1| + |e_2|} = \newline
		2 * \Complejidad{|e_1| + |e_2|} = \Complejidad{|e_1| + |e_2|}$}
	\end{Algoritmo}
	
	\begin{Algoritmo}{iVecinos}{\In{r}{red}, \In{c}{compu}}{itConj(compu)}
	{
		\State $res$ $\gets$ crearIt((Significado($r.vecinosEInterfaces$, $c.ip$))$.compusVecinas$)
	}
	{}
	{}
	\end{Algoritmo}
	
	\begin{Algoritmo}{iUsaInterfaz}{\In{r}{red}, \In{c}{compu}, \In{i}{interfaz}}{bool}
	{
		\State var $tupVecinos$:tupla $\gets$ Significado($r.vecinosEInterfaces$, $c.ip$) \ComplejidadDer{1}
		\State var $itCompusVecinas$:itConj(compu) $\gets$ CrearIt($tupVecinos.compusVecinas$) \ComplejidadDer{1}
		\State $res$:bool $\gets$ $false$ \ComplejidadDer{1}
		\While{HaySiguiente($itCompusVecinas$) AND $�res$} \ComplejidadDer{1}
		\If {Significado($tupVecinos.interfaces$, Siguiente($itCompusVecinas$)$.ip$) $==$ $i$} \ComplejidadDer{1}
			\State $res$ $\gets$ $true$  \ComplejidadDer{1}
		\EndIf
		\State Avanzar($it$) \ComplejidadDer{1}
		\EndWhile
	}
	{}
	{}
	\end{Algoritmo}
	
	\begin{Algoritmo}{iCaminosMinimos}{\In{r}{red}, \In{c_1}{compu}, \In{c_2}{compu}}{conj(lista(compu))}
	{
		\State $res$ $\gets$ Significado(Significado($r.deOrigenADestino$, $c_1.ip$), $c_2.ip$)	\ComplejidadDer{|c_1| + |c_2|}
	}
	{}
	{}
	\end{Algoritmo}

	\begin{Algoritmo}{iHayCamino}{\In{r}{red}, \In{c_1}{compu}, \In{c_2}{compu}}{bool}
	{
		\State var $conjCaminosMinimos$ $\gets$ CaminosMinimos($r$, $c_1$, $c_2$) \ComplejidadDer{1}
		\State $res$ $\gets$ EsVacio?($conjCaminosMinimos$) \ComplejidadDer{1}
	}
	{}
	{}
	\end{Algoritmo}
	
	\begin{Algoritmo}{iCaminosMinimos}{\In{r}{red}, \In{c_1}{compu}, \In{c_2}{compu}}{bool}
	{
		%\State var $conjExcluidos$ $\gets$ Vacio()
		%\State var $camino$
	}
	{}
	{}
	\end{Algoritmo}
		
	%\begin{Algoritmo}{iNroConexion}{\In{e1}{estaci�n}, \In{e2}{estaci�n}, \In{m}{mapa}}{nat}
	%{
	%	\State $res$ $\gets$ Significado(Significado($m.uniones$, $e1$), $e2$) \ComplejidadDer{|e_1| + |e_2|}
	%}
	%{$\Complejidad{|e_1| + |e_2|}$}{}
	%\end{Algoritmo}
	%
	%\begin{Algoritmo}{iEvaluarSendas}{\In{c}{conj(tag)}, \In{m}{mapa}}{arreglo\_dimesionable de bool}
	%{
	%	\State $res$ $\gets$ arreglo[$m.\#sendas$] de bool \ComplejidadDer{1}
	%	\State var $i$: nat $\gets$ 0 \ComplejidadDer{1}
	%	\While{$i < m.\#sendas$} \ComplejidadDer{1}
	%		\State $res[i]$ $\gets$ Verifica?($c$, $m.sendas[i]$) \ComplejidadDer{R}
	%		\State $i++$ \ComplejidadDer{1}
	%	\EndWhile
	%}
	%{$\Complejidad{S * R}$}
	%{$\Complejidad{1} + \Complejidad{1} + \sum_{i = 1}^{S} (\Complejidad{R} + \Complejidad{1}) = \newline
	%	2 * \Complejidad{1} + S * (\Complejidad{R} + \Complejidad{1}) = \newline
	%	2 * \Complejidad{1} + S * \Complejidad{1} + S * \Complejidad{R} = \newline
	%	\Complejidad{S} + S * \Complejidad{R} = \newline
	%	\Complejidad{S + S * R} = \Complejidad{S * R}
	%$\newline
	%$S$ es |m.sendas| y $R$ es la longitud de la restriccion mas grande}
	%\end{Algoritmo}
	
	%\begin{Algoritmo}{iRestricciones}{\In{m}{mapa}}{arreglo\_dimesionable de restriccion}
	%{
	%	\State $res$ $\gets$ arreglo\_dimensionable[$m.\#sendas$] \ComplejidadDer{1}
	%	\State var $i$: nat $\gets$ 0 \ComplejidadDer{1}
	%	\While{$i < m.\#sendas$} \ComplejidadDer{1}
	%		\State $res[i]$ $\gets$ $m.sendas[i]$ \ComplejidadDer{1}
	%		\State $i++$ \ComplejidadDer{1}
	%	\EndWhile
	%}
	%{$\Complejidad{S}$}
	%{$\Complejidad{1} + \Complejidad{1} + \sum_{i = 1}^{S} (\Complejidad{1} + \Complejidad{1}) = \newline
	%	2 * \Complejidad{1} + S * 2 *\Complejidad{1} = \newline
	%	2 * \Complejidad{1} + 2 * \Complejidad{S} = \newline
	%	2 * \Complejidad{S} = \Complejidad{S}
	% $}
	%\end{Algoritmo}
	%
\end{Algoritmos}

\newpage
\section{Ciudad Rob�tica}

\begin{Interfaz}
  
  %\textbf{par�metros formales}\hangindent=2\parindent\\
  %\parbox{1.7cm}{\textbf{g�neros}} \\
  %\parbox[t]{1.7cm}{\textbf{funci�n}}\parbox[t]{\textwidth-2\parindent-1.7cm}{%
    %\InterfazFuncion{Verifica?}{\In{c}{conj(tag)}, \In{r}{rest}}{$\alpha$}
    %{$res \igobs a$}
    %[$\Theta(copy(a))$]
    %[funci�n de copia de $\alpha$'s]
  %}

  \textbf{se explica con}: \tadNombre{Ciudad Rob�tica, Iterador Unidireccional($\alpha$)}.

  \textbf{g�neros}: \TipoVariable{ciudad, itRURs}.

  \Titulo{Operaciones b�sicas de Ciudad Rob�tica}
	
  \InterfazFuncion{Pr�ximoRUR}{\In {c}{ciudad}}{rur}
  %[ACA VA EL PRE (SI LO HUBIERA)]
  {$res \igobs$ pr�ximoRUR($c$)}
  [\Complejidad{1}]
  [Devuelve el Pr�ximoRUR de una ciudad, esto es, de a�adirse un robot se le asignar�a este RUR.]
  
  \InterfazFuncion{Mapa}{\In {c}{ciudad}}{mapa}
  %[ACA VA EL PRE (SI LO HUBIERA)]
  {$res \igobs$ mapa($c$)}
  [\Complejidad{1}]
  [Devuelve el mapa de la ciudad.]
  
  \InterfazFuncion{Robots}{\In {c}{ciudad}}{itRURs}
  %[ACA VA EL PRE (SI LO HUBIERA)]
  {$res \igobs$ crearIt(robots($c$))}
  [\Complejidad{1}]
  [Devuelve un iterador de los robots de la ciudad.]
  
  \InterfazFuncion{Estaci�n}{\In {u}{rur}, \In {c}{ciudad}}{estacion}
  [$u$ $\in$ robots($c$)]
  {$res \igobs$ estaci�n($u,c$)}
  [\Complejidad{1}]
  [Devuelve la estaci�n en la cual est� el robot.]
  
  \InterfazFuncion{Tags}{\In {u}{rur}, \In {c}{ciudad}}{conj(tags)}
  [$u$ $\in$ robots($c$)]
  {$res \igobs$ tags($u,c$)}
  [\Complejidad{1}]
  [Devuelve los tags del robot.]
  
  \InterfazFuncion{$\#$Infracciones}{\In {u}{rur}, \In {c}{ciudad}}{nat}
  [$u$ $\in$ robots($c$)]
  {$res \igobs$ $\#$infracciones($u,c$)}
  [\Complejidad{1}]
  [Devuelve la cantidad de infracciones cometidas por el robot.]

  \InterfazFuncion{Crear}{\In {m}{mapa}}{ciudad}
  %[ACA VA EL PRE (SI LO HUBIERA)]
  {$res \igobs$ crear($m$)}
  [\Complejidad{Cardinal(Estaciones(m)) * |e_m|}]
  [Crea una ciudad con un mapa y sin robots.]
  
  \InterfazFuncion{Entrar}{\In {ts}{conj(tags)}, \In {e}{estaci�n}, \Inout {c}{ciudad}}{}
  [$c_0 \equiv c$ $\wedge$ e $\in$ estaciones($c_0$)]
  {$c$ $\igobs$ entrar($ts, e, c_0$)}
  [\Complejidad{log_2 N + |e| + S * R}]
  [A�ade un robot a la ciudad, le asigna el pr�ximoRUR y sus infracciones son nulas.]
  
  \InterfazFuncion{Mover}{\In {u}{rur}, \In {e}{estaci�n}, \Inout {c}{ciudad}}{}
  [($c_0 \equiv c$ $\wedge$ $u$ $\in$ robots($c_0$) $\wedge$ $e$ $\in$ estaciones($c_0$)) $\yluego$ conectadas?(estaci�n($u$, $c_0$), $e$ mapa($c_0$))]
  {$c$ $\igobs$ mover($u, e, c_0$)}
  [\Complejidad{|e| + |e_0| + log_2 N_e + log_2 N_{e0}}]
  [Mueve un robot desde donde est� a la estaci�n indicada.]
  
  \InterfazFuncion{Inspecci�n}{\In {e}{estaci�n}, \Inout{c}{ciudad}}{}
  [$c_0 \equiv c$ $\wedge$ $e$ $\in$ estaciones($c_0$)]
  {$c$ $\igobs$ inspeccion($e, c_0$)}
  [\Complejidad{log_2 N}]
  [Realiza la inspecci�n de la estaci�n indicada, remueve el robot con mayor cantidad de infracciones.]
	
	\Titulo{Operaciones del iterador}
  
  \InterfazFuncion{crearIt}{\In{c}{ciudad}}{itRURs}
  %[$c_0 \equiv c$ $\wedge$ $e$ $\in$ estaciones($c_0$)]
  {res $\igobs$ CrearItUni(robots(c))}
  [\Complejidad{1}]
  [Crea el iterador de robots.]
	
	\InterfazFuncion{Actual}{\In{it}{itRURs}}{rur}
  %[$c_0 \equiv c$ $\wedge$ $e$ $\in$ estaciones($c_0$)]
  {res $\igobs$ Actual(it)}
  [\Complejidad{1}]
  [Devuelve el actual del iterador de robots.]
	
	\InterfazFuncion{Avanzar}{\In{it}{itRURs}}{itRURs}
  %[$c_0 \equiv c$ $\wedge$ $e$ $\in$ estaciones($c_0$)]
  {res $\igobs$ Avanzar(it)}
  [\Complejidad{1}]
  [Avanza el iterador de robots.]
	
	\InterfazFuncion{HayMas?}{\In{it}{itRURs}}{bool}
  %[$c_0 \equiv c$ $\wedge$ $e$ $\in$ estaciones($c_0$)]
  {res $\igobs$ HayMas?(it)}
  [\Complejidad{1}]
  [Se fija si hay mas elementos en el iterador de robots.]
	
  
\end{Interfaz}

\subsection{Representacion}

\begin{Representacion}

%DESCRIPCION

\bigskip
\begin{Estructura}{ciudad}[e\_cr]
	\begin{Tupla}[e\_cr]
		\tupItem{mapa}{mapa}
		\tupTupItem{RUREnEst}{\TipoVariable{diccString}$($\ignorespaces
			\emph{estacion}: \TipoVariable{string}, 
			\emph{robs}: \TipoVariable{colaP}$($\ignorespaces
			\emph{id}: \TipoVariable{nat}, 
			\emph{inf}: \TipoVariable{nat}$))$}
		\tupTupItem{\\RURs}{\TipoVariable{vector de tupla}$($\ignorespaces
			\emph{id}: \TipoVariable{nat}, 
			\emph{esta?}: \TipoVariable{bool}, 
			\emph{e}: \TipoVariable{string}, 
			\emph{inf}: \TipoVariable{nat}, 
			\emph{carac}: \TipoVariable{conj(string)}, 
			\emph{\\sendEv}: \TipoVariable{arreglo\_dimensionable de bool}\ignorespaces
		$)$}
		\tupItem{\#RURHistoricos}{nat}
	\end{Tupla}
\end{Estructura}

\subsection{InvRep y Abs}

\begin{enumerate}
 \item{El conjunto de estaciones de 'mapa' es igual al conjunto con todas las claves de 'RURenEst'.}
 \item{La longitud de 'RURs' es mayor o igual a '\#RURHistoricos'.}
 \item{Todos los elementos de 'RURs' cumplen que su primer componente ('id') corresponde con su posicion en 'RURs'. 
       Su Componente 'e' es una de las estaciones de 'mapa', su componente 'esta?' es true si y solo si hay estaciones tales que su 
       valor asignado en 'uniones' es igual a su indice en 'RURs'. Su Componente 'inf' puede ser mayor a cero solamente si hay algun elemento en 
       'sendEv' tal que sea false. Cada elemento de 'sendEv' es igual a verificar 'carac' con la estriccion obtenida al buscar el 
       elemento con la misma posicion en la secuencia de restricciones de 'mapa'.}
 \item{Cada valor contenido en la cola del significado de cada estacion de las claves de 'uniones' pertenecen unicamente a la cola asociada a dicha estacion y a ninguna otra de las colas asociadas a otras estaciones. Y cada uno de estos valores es menor a '\#RURHistoricos' y mayor o igual a cero. Ademas la componente 'e' del elemento de la posicion igual a cada valor de las colas asociadas a cada estacion, es igual a la estacion asociada a la cola a la que pertenece el valor.}

\end{enumerate}


\Rep[e\_cr][c]{
	claves(c.RURenEst) $=$ estaciones(c.mapa) $\wedge$ \hfill1 \newline
	\#RURHistoricos $\leq$ Long(c.RURs) $\yluego$
	($\forall$ i:Nat, t:<id:Nat, esta?:Bool, e:String, \hfill2 \newline
	inf:Nat, carac:Conj(Tag), sendEv: ad(Bool)>) \newline
	(i<\#RURHistoricos $\yluego$ ElemDeSecu(c.RURs, i) = t $\impluego$ (t.e $\in$ estaciones(c.mapa) \hfill3 \newline
	$\wedge$ t.id $=$ i $\wedge$ tam(t.sendEv) $=$ long(Restricciones(c.mapa)) $\wedge$ \newline
	(t.inf > 0 $\Rightarrow$ ($\exists$ j:Nat) (j < tam(t.sendEv) $\yluego$ $\neg$ (t.sendEv[j]))) $\wedge$ \newline
	(t.esta? $\Leftrightarrow$ ($\exists$ e1: String) (e1 $\in$ claves(c.RUREnEst) $\yluego$ estaEnColaP?(obtener(e1, c.RUREnEst), t.id))) \newline
	$\wedge$ ($\forall$ h : Nat) (h < tam(t.sendEv) $\impluego$ \newline
	t.sendEv[h] = verifica?(t.carac, ElemDeSecu(Restricciones(c.mapa), h))))) $\yluego$ \newline
	($\forall$ e1, e2: String)(e1 $\in$ claves(c.RUREnEst) $\wedge$ e2 $\in$ claves(c.RUREnEst) $\wedge$ e1 $\neq$ e2 $\impluego$ \hfill4 \newline
	($\forall$ n:Nat)(estaEnColaP?(obtener(e1, c.RUREnEst), n) $\Rightarrow$ $\neg$ estaEnColaP?(obtener(e2, c.RUREnEst), n) $\wedge$
	n < \#RURHistoricos $\yluego$ ElemDeSecu(c.RURs, n).e $=$ e1))
	}
	
	\vspace{3em}
	
	\tadOperacion{estaEnColaP?}{ColaPri, Nat}{Bool}{}
	\vspace{1em}
	\tadAxioma{estaEnColaP?(cp, n)}{
	\IF vacia?(cp) 
	THEN false 
	ELSE 
	      {\IF desencolar(cp) = n 
	      THEN true 
	      ELSE estaEnColaP?(Eliminar(cp, desencolar(cp)), n) 
	      FI}
	FI}
	
	\vspace{3em}
	
	\Abs[e\_cr]{ciudad}[c]{u}{
		\\ c.\#RURHistoricos = ProximoRUR(U) $\wedge$ c.mapa = mapa(u) $\yluego$
		\\ robots(u) = RURQueEstan(c.RURs) $\yluego$ 
		\\ ($\forall$ n:Nat) (n $\in$ robots(u) $\impluego$ estacion(n,u) = c.RURs[n].e $\wedge$
		\\ tags(n,u) = c.RURs[n].carac $\wedge$ \#infracciones(n,u) = c.RURs[n].inf)}
		
	\vspace{3em}
	\tadOperacion{RURQueEstan}{secu(tupla)}{Conj(RUR)}{}
	\vspace{1em}
	tupla es <id:Nat, esta?:Bool, inf:Nat, carac:Conj(tag), sendEv:arreglo dimensionable(bool)>
	\vspace{1em}
	\tadAxioma{RURQueEstan(s)}{
	\IF vacia?(s)
	THEN $\emptyset$
	ELSE 
	      {\IF $\Pi_2$(prim(fin(s)))
	      THEN {$\Pi_1$(prim(fin(s)))} $\cup$ RURQueEstan(fin(s))
	      ELSE RURQueEstan(fin(s))
	      FI}
	FI}
	
\vspace{3em}

\begin{Estructura}{it}[e\_it]
	\begin{Tupla}[e\_it]
		\tupItem{i}{nat}
		\tupItem{maxI}{nat}
		\tupItem{ciudad}{puntero(ciudad)}
	\end{Tupla}
\end{Estructura}

\Rep[e\_it][it]{
	it.i $\leq$ it.maxI $\wedge$ maxI $=$ ciudad.\#RURHistoricos
}

\Abs[e\_it]{itUni($\alpha$)}[u]{it}{
	(HayMas?(u) $\yluego$ Actual(u) = ciudad.RURs[it.i] $\wedge$ Siguientes(u, $\emptyset$) = VSiguientes(ciudad, it.i++, $\emptyset$) $\vee$ ($\neg$HayMas?(u))
}

\vspace{2em}
	
\tadOperacion{Siguientes}{itUni u, conj(RURs) cr}{conj(RURs)}{}
														
\vspace{1em}

\tadAxioma{Siguientes(u, cr)}{\IF HayMas(u)? THEN Ag(Actual(Avanzar(u)), Siguientes(Avanzar(u), cr)) ELSE Ag($\emptyset$, cr) FI}
		
\vspace{1em}

\tadOperacion{VSiguientes}{ciudad c, Nat i, conj(RURs) cr}{conj(RURs)}{}
														
\vspace{1em}

\tadAxioma{VSiguientes(u, i, cr)}{\IF i < c.\#RURHistoricos THEN Ag(c.RURs[i], VSiguientes(u, i++, cr))) ELSE Ag($\emptyset$, cr) FI}

\vspace{2em}

\end{Representacion}

%tupla<mapa:mapa, robEnEst: dicctrie(estacion, conj(puntero(RUR))), robots: conj(RUR)>

\subsection{Algoritmos}

\begin{Algoritmos}

	\begin{Algoritmo}{iPr�ximoRUR}{\In {c}{ciudad}}{rur}
	{
		\State $res$ $\gets$ ($c.\#RURHistoricos$) \ComplejidadDer{1}
	}
	{\Complejidad{1}}
	{}
	
	\end{Algoritmo}
	
	\begin{Algoritmo}{iMapa}{\In {c}{ciudad}}{mapa}
	{
		\State $res$ $\gets$ $c.mapa$ \ComplejidadDer{1}
	}
	{\Complejidad{1}}
	{}
	\end{Algoritmo}
	
	\begin{Algoritmo}{iRobots}{\In {c}{ciudad}}{itRobots}
	{
		\State $res$ $\gets$ CrearIt($c.RURs$) \ComplejidadDer{1}
	}
	{\Complejidad{1}}
	{}
	\end{Algoritmo}
	
	\begin{Algoritmo}{iEstaci�n}{\In {u}{rur}, \In {c}{ciudad}}{estaci�n}
	{
		\State $res$ $\gets$ ($c.RURs[u]).estacion$ \ComplejidadDer{1}
	}
	{\Complejidad{1}}
	{}
	\end{Algoritmo}
	
	\begin{Algoritmo}{iTags}{\In {u}{rur}, \In {c}{ciudad}}{conj(tags)}
	{
		\State $res$ $\gets$ ($c.RURs[u]).carac$ \ComplejidadDer{1}
	}
	{\Complejidad{1}}
	{}
	\end{Algoritmo}
	
	\begin{Algoritmo}{i$\#$Infracciones}{\In {u}{rur}, \In {c}{ciudad}}{nat}
	{
		\State $res$ $\gets$ ($c.RURs[u]).inf$ \ComplejidadDer{1}
	}
	{\Complejidad{1}}
	{}
	\end{Algoritmo}
	
	\begin{Algoritmo}{iCrear}{\In {m}{mapa}}{ciudad}
	{
		\State $res$ $\gets$ tupla($mapa$: $m$, $RUREnEst$: Vac�o(), $RURs$: Vac�a(), $\#RURHistoricos$: 0) \ComplejidadDer{1}
		\State var $it$:itConj(Estacion) $\gets$ Estaciones($m$) \ComplejidadDer{1}
		\While{HaySiguiente($it$)} \ComplejidadDer{1}
		\State Definir($res.RUREnEst$, Siguiente($it$), Vac�o()) \ComplejidadDer{|e_m|}
		\State Avanzar($it$) \ComplejidadDer{1}
		\EndWhile
	}
	{$\Complejidad{Cardinal(Estaciones(m)) * |e_m|}$}
	{$\Complejidad{1} + \Complejidad{1} + \sum_{i=1}^{Cardinal(Estaciones(m))} (\Complejidad{|e_m|} + \Complejidad{1}) = \newline
		 2 * \Complejidad{1} + Cardinal(Estaciones(m)) * (\Complejidad{|e_m|} + \Complejidad{1}) = \newline 
		 Cardinal(Estaciones(m)) * (\Complejidad{|e_m|})
	$}
	%{\Complejidad{1} + \Complejidad{Cardinal(Estaciones(m))} * \Complejidad{|e_m|} = \Complejidad{Cardinal(Estaciones(m)) * |e_m|}}
	\end{Algoritmo}
	
	\begin{Algoritmo}{iEntrar}{\In {ts}{conj(tags)}, \In {e}{string}, \Inout {c}{ciudad}}{}
	{
		\State Agregar(Significado($c.RUREnEst$, $e$), 0, $c.\#RURHistoricos$) \ComplejidadDer{log_2 n + |e|}
		\State Agregar($c.RURs$, $c.\#RURHistoricos$, tupla($id$: $c.\#RURHistoricos$, $esta?$: $true$, $estacion$: $e$, $inf$: 0, $carac$: $ts$, $sendEv$: EvaluarSendas($ts$, $c.mapa$)) \ComplejidadDer{1 + S * R}
		\State $c.\#RURHistoricos++$ \ComplejidadDer{1}
	}
	{$\Complejidad{log_2 n + |e| + S * R}$}
	{$\Complejidad{log_2 n + |e|} + \Complejidad{1 + S * R} + \Complejidad{1} = \Complejidad{log_2 n + |e| + S * R}$}
	\end{Algoritmo}
	
	\begin{Algoritmo}{iMover}{\In {u}{rur}, \In {e}{estaci�n}, \Inout {c}{ciudad}}{}
	{
		\State Eliminar(Significado($c.RUREnEst$, $c.RURs[u].estacion$), $c.RURs[u].inf$, $u$) \ComplejidadDer{|e| + log_2 N_{e0}}
		\State Agregar(Significado($c.RUREnEst$, $e$), $c.RURs[u].inf$, $u$) \ComplejidadDer{|e| + log_2 N_e}
		\If{$\neg(c.RURs[u].sendEv$[NroConexion($c.RURs[u].estacion$, $e$, $c.mapa$)])} \ComplejidadDer{|e_0|	+ |e|}
			\State $c.RURs[u].inf$++ \ComplejidadDer{1}
		\EndIf
		\State $c.RURs[u].estacion$ $\gets$ $e$ \ComplejidadDer{1}
	}
	{$\Complejidad{|e| + log_2 N_e$}}
	{$\Complejidad{|e| + log_2 N_{e_0}} + \Complejidad{|e| + log_2 N_e} + \Complejidad{|e_0|, |e|} + max(\Complejidad{1}, \Complejidad{0}) + \Complejidad{1} = \newline
		\Complejidad{2 * |e| + log_2 N_e + log_2 N_{e0}} + \Complejidad{|e_0| + |e|} + 2 * \Complejidad{1} = \newline
		\Complejidad{|e| + log_2 N_e + log_2 N_{e0}} + \Complejidad{|e_0| + |e|} = \newline
		\Complejidad{2 * |e| + |e_0| + log_2 N_e + log_2 N_{e0}} = \Complejidad{|e| + |e_0| + log_2 N_e + log_2 N_{e0}}
	 $
	Donde $e_0$ es c.RURs[u]estacion antes de modificar el valor}
	\end{Algoritmo}
	
	%tupla($id$: nat, $esta?$: bool, $estacion$: string, $inf$: nat, $carac$: conj(string), $sendEv$: arreglo\_dimensionable de bool)
	\begin{Algoritmo}{iInspecci�n}{\In {e}{estaci�n}, \Inout{c}{ciudad}}{}
	{
		\State var $rur$: nat $\gets$ Desencolar(Significado($c.RUREnEst$, $e$)) \ComplejidadDer{log_2 N}
		\State $c.RURs[rur].esta?$ $\gets$ $false$ \ComplejidadDer{1}
	}
	{$\Complejidad{log_2 N}$}
	{$\Complejidad{log_2 N} + \Complejidad{1} = \Complejidad{log_2 N}$}
	\end{Algoritmo}
	
	\begin{Algoritmo}{icrearIt}{\In{c}{ciudad}}{itRURs}
	{
		\State $itRURS$ $\gets$ tupla($i: 0, maxI: c.\#RURHistoricos, ciudad: \&c$) \ComplejidadDer{1}
	}
	{$\Complejidad{1}$}{}
	\end{Algoritmo}
	
	\begin{Algoritmo}{iActual}{\In{it}{itRURs}}{rur}
	{
		\State $res$ $\gets$ $(it.ciudad \rightarrow RURs)[it.i]$ \ComplejidadDer{1}
	}
	{$\Complejidad{1}$}{}
	\end{Algoritmo}
	
	\begin{Algoritmo}{iAvanzar}{\In{it}{itRURs}}{itRURs}
	{
		\State $it.i++$ \ComplejidadDer{1}
	}
	{$\Complejidad{1}$}{}
	\end{Algoritmo}
	
	\begin{Algoritmo}{iHayMas?}{\In{it}{itRURs}}{bool}
	{
		\State $res$ $\gets$ $(it.i < it.maxI)$ \ComplejidadDer{1}
	}
	{$\Complejidad{1}$}{}
	\end{Algoritmo}

	%\Algoritmo{IENTRAR}{\In {ts}{conj(tags)}, \In {e}{estaci�n}, \Inout {c}{ciudad}}{}{
		%\State e\_ciudad.cRUR $\gets$ e\_ciudad.cRUR + 1 \ComplejidadDer{1}
		%\State var sendas $\gets$ EVALUARSENDAS(ts, e\_ciudad.m) \ComplejidadDer{S \cdot R}
		%\State var nRUR $\gets$ tupla(e\_ciudad.RURs, e, 0, ts, sendas) \ComplejidadDer{1}
		%\State AGREGARATRAS(e\_ciudad.RURs, nRUR) \ComplejidadDer{1}
		%\State obtEst = \&(SIGNIFICADO(e.RUREnEst, e)) \ComplejidadDer{|e|}
		%\State ENCOLAR(obtEst, tupla(e\_ciudad.cRUR, 0)) \ComplejidadDer{log\ N}
	%}{\Complejidad{|e| + S \cdot R + log\ N}}{
	%Justificacion.
	%}
	%
	%\newpage
	%
	%\Algoritmo{IMOVER}{\In {u}{nat}, \In {e2}{estaci�n}, \Inout {c}{ciudad}}{}{
		%\State var est $\gets$ OBTENERCOLAEST(c.RUREnEst, c.RURs[u].e) \ComplejidadDer{|e1|}
		%\State SACARRUR(est, u) \ComplejidadDer{log\ N_{e1}}
		%\State INFRACCIONO?(u, NROCONEXION(c.RURs[u].e, e2) \ComplejidadDer{|e1| + |e2|}
		%\State MODIFICARRUR(u) \ComplejidadDer{1}
		%\State est $\gets$ OBTENERCOLAEST(c.RUREnEst, e2) \ComplejidadDer{|e2|}
		%\State METERRUR(est, u) \ComplejidadDer{log\ N_{e2}}
	%}{\Complejidad{|e1| + |e2| + log\ N_{e1} + log\ N_{e2}}}{
		%$\Complejidad{|e1|} + \Complejidad{|e2|} + \Complejidad{log\ N_{e1}} + 
		%\Complejidad{log\ N_{e2}} + \Complejidad{|e1| + |e2|} + \Complejidad{1} =$\\
		%$2 * \Complejidad{|e1| + |e2|} + \Complejidad{log\ N_{e2} + log\ N_{e1}} =$\\
		%$\Complejidad{|e1| + |e2|} + \Complejidad{log\ N_{e2} + log\ N_{e1}} =$\\
		%$\Complejidad{|e1| + |e2| + log\ N_{e2} + log\ N_{e1}}$
	%}

\end{Algoritmos}

\newpage
\section{Diccionario String}

\subsection{Interfaz}

\begin{Interfaz}

  \textbf{se explica con}: \tadNombre{Diccionario(String, $\alpha$)}.

  \textbf{g�neros}: \TipoVariable{diccString($\alpha$)}.

  \Titulo{Operaciones b�sicas de Diccionario String($\alpha$)}
  
  \InterfazFuncion{Def?}{\In{clv}{string}, \In{d}{diccString($\alpha$)}}{bool}
  [true]
  {$res \igobs$ def?($clv, d$)}
  [$\Complejidad{|clv|}$]
  [Revisa si la clave ingresada se encuentra definida en el Diccionario.]
  
  \InterfazFuncion{Obtener}{\In{clv}{string}, \In{d}{diccString($\alpha$)}}{$\alpha$}
  [def?($clv, d$)]
  {$res \igobs$ obtener($clv, d$)}
  [$\Complejidad{|clv|}$]
  [Devuelve el significado de la clave.]

  \InterfazFuncion{Vac�o}{}{diccString($\alpha$)}
  [true]
  {$res \igobs$ vac�o()}
  [\Complejidad{1}]
  [Crea nuevo diccionario vacio.]
	
<<<<<<< HEAD
  \InterfazFuncion{Definir}{\Inout{d}{diccString($\alpha$)}, \In{clv}{string}, \In{def}{$\alpha$}}{}
  [$d_0 \igobs d$]
  {$d \igobs$ definir($clv, def, d_0$)}
  [$\Complejidad{|clv|}$]
  [Agrega un nueva definicion.]
	
  \InterfazFuncion{Def?}{\In{d}{diccString($\alpha$)}, \In{clv}{string}}{bool}
  %[ACA VA EL PRE (SI LO HUBIERA)]
  {$res \igobs$ def?($clv, d$)}
  [$\Complejidad{|clv|}$]
  [Revisa si la clave ingresada se encuentra definida en el Diccionario.]
	
  \InterfazFuncion{Obtener}{\In{d}{diccString($\alpha$)}, \In{clv}{string}}{diccString($\alpha$)}
  [def?($d, clv$)]
  {$res \igobs$ obtener($clv, d$)}
=======
  \InterfazFuncion{Definir}{\In{clv}{string}, \In{def}{$\alpha$}, \Inout{d}{diccString($\alpha$)}}{}
  [$d$ $\igobs$ $d_0$]
  {$d$ $\igobs$ definir($clv, def, d_0$)}
  [$\Complejidad{|clv|}$]
  [Agrega un nueva definicion.]
	
  \InterfazFuncion{Borrar}{\In{clv}{string}, \Inout{d}{diccString($\alpha$)}}{}
  [$d$ $\igobs$ $d_0$ $\wedge$ def?(clv, d)]
  {$res$ $\igobs$ borrar($k, d_0$)}
>>>>>>> 6a8b227b980cf43c591bb3166b24fd34968fec9a
  [$\Complejidad{|clv|}$]
  [Devuelve la definicion correspondiente a la clave.]
	
\end{Interfaz}


\newpage
\section{Cola Prioritaria}

\subsection{TAD \tadNombre{ColaPrioritaria}}

\begin{tad}{\tadNombre{ColaPrioritaria}}
\tadIgualdadObservacional{c1}{c2}{colaPr}{%$vacia?(c1) $\igobs$ vacia?(c2) $\yluego$\\
																					($\forall$a, b: nat)(esta?(a, b, c1) $\ssi$ 
																					esta?(a, b, c2))% $\yluego$\\
																					%($\forall$a, b: nat)(esta?(a, b, c1) $\impluego$\\ 
																					%borrar(a, b, c1) $\igobs$ borrar(a, b, c2)) $\wedge$\\
																					%($\neg$vacia?(c1) $\impluego$ 
																					%(proximo(c1) $\igobs$ proximo(c2)))
																					}

\tadGeneros{colaPr}

\tadExporta{colaPr, generadores, observadores}
\tadUsa{\tadNombre{Bool}, \tadNombre{Nat}}

\tadObservadores
\tadAlinearFunciones{proximo}{nat/a, nat/b, colaPr/c}
\tadOperacion{vacia?}{colaPr/c}{bool}{}
\tadOperacion{esta?}{nat/a, nat/b, colaPr/c}{bool}{}
\tadOperacion{borrar}{nat/a, nat/b, colaPr/c}{colaPr}{esta?(a, b, c)}
\tadOperacion{proximo}{colaPr/c}{tupla(nat, nat)}{$\neg$vacia?(c)}

\tadGeneradores
\tadAlinearFunciones{encolar}{nat/a, nat/b, colaPr/c}
\tadOperacion{vacia}{}{colaPr}{}
\tadOperacion{encolar}{nat/a, nat/b, colaPr/c}{colaPr}{$\neg$esta?(a, b, c)}

\tadOtrasOperaciones
\tadAlinearFunciones{desencolar}{colaPr/c}
\tadOperacion{desencolar}{colaPr/c}{colaPr}{$\neg$vacia?(c)}

\tadAxiomas[\paratodo{colaPr}{c} \paratodo{nat}{a, b, a1, b1}]
\tadAlinearAxiomas{borrar(a1, b1, encolar(a, b, c))}

\tadAxioma{vacia?(vacia)}{true}
\tadAxioma{vacia?(encolar(a, b , c))}{false}
\tadAxioma{esta?(a1, b1, vacia)}{false}
\tadAxioma{esta?(a1, b1, encolar(a, b, c))}{\IF a = a1 $\wedge$ b = b1 THEN
																						true ELSE esta?(a1, b1, c) FI}
\tadAxioma{borrar(a1, b1, encolar(a, b, c))}{\IF a = a1 $\wedge$ b = b1 THEN
																						 c ELSE encolar(a, b, borrar(a1, b1, c)) FI}
\tadAxioma{proximo(encolar(a, b, c))}{\IF vacia?(c) $\oluego$ $\pi_1$(proximo(c)) $\leq$ a
																			THEN {\IF $\pi_1$(proximo(c)) $=$ a THEN
																						{\IF $\pi_1$(proximo(c)) $<$ b THEN
																						$\langle$a, b$\rangle$ ELSE proximo(C) FI}
																						ELSE $\langle$a, b$\rangle$ FI}
																						ELSE proximo(C) FI}
\tadAxioma{desencolar(c)}{borrar(proximo(c))}

\end{tad}

\begin{Interfaz}
  
  \textbf{se explica con}: \tadNombre{ColaPrioritaria}.

  \textbf{g�neros}: \TipoVariable{colaP}.

  \Titulo{Operaciones b�sicas de Restricci�n}

  \InterfazFuncion{Vac�o}{}{colaP}
  %[ACA VA EL PRE (SI LO HUBIERA)]
  {$res \igobs$ vacio()}
  [$\Complejidad{1}$]
  [Crea una nueva cola.]
	
  \InterfazFuncion{Agregar}{\Inout{c}{colaP}, \In{a}{nat}, \In{b}{nat}}{}
  [$c_t \igobs c$]
  {$c \igobs$ encolar(a, b, c\_t)}
  [$\Complejidad{log_2\ N}$ con $N$ siendo la cantidad total de elementos en la cola.]
  [Agrega dos nuevos numeros a la cola.]
	
	\InterfazFuncion{Vacia?}{\In{c}{colaP}}{bool}
  %[ACA VA EL PRE (SI LO HUBIERA)]
  {$res \igobs$ vacia?()}
  [$\Complejidad{1}$]
  [Revisa si la cola contiene por lo menos algun elemento.]
	
	\InterfazFuncion{Desencolar}{\Inout{c}{colaP}}{tupla(a: nat, b: nat)}
  [$c \igobs c_t \wedge \neg$(vacia?(c))]
  {$res \igobs$ proximo(c\_t) $\wedge\ c \igobs$ desencolar(c\_t)}
  [$\Complejidad{log_2\ N}$ con $N$ siendo la cantidad total de elementos en la cola.]
  [Devuelve la tupla de numeros mas grande y la quita de la cola.]
	
	\InterfazFuncion{Tama�o}{\Inout{c}{colaP}}{nat}
  %[$c \igobs c_t \wedge \neg$(vacia?(c))]
  {$res \igobs$ tama�o(c)}
  [$\Complejidad{1}$]
  [Devuelve la cantidad total de elementos.]
	
	\InterfazFuncion{Eliminar}{\Inout{c}{colaP}, \In{a}{nat}, \In{b}{nat}}{}
  [c = c0]
  {$c \igobs$ borrar(c0, k)}
  [$\Complejidad{log_2\ N}$ con $N$ siendo la cantidad total de elementos en la cola.]
  [Si a y b se encuentran en la cola, los quita de la misma.]
	
\end{Interfaz}

\subsection{Representacion}

\begin{Representacion}

Para representar la cola elegimos hacerla sobre un arbol AVL.

\begin{Estructura}{colaP}[e\_cola]
	\begin{Tupla}[e\_cola]
		\tupTupItem{raiz}{\TipoVariable{puntero}(nodo)}
		\tupItem{tam}{nat}
		%\tupItem{cVac}{bool}
	\end{Tupla}
	
	\begin{Tupla}[nodo]
		\tupItem{pri}{nat}
		\tupItem{seg}{nat}
		\tupTupItem{padre}{\TipoVariable{puntero}(nodo)}
		\tupTupItem{izq}{\TipoVariable{puntero}(nodo)}
		\tupTupItem{der}{\TipoVariable{puntero}(nodo)}
		\tupItem{\\alt}{nat}
	\end{Tupla}
	
\end{Estructura}

\subsection{InvRep y Abs}

\textbf{InvRep en lenguaje coloquial:}

\begin{enumerate}
	\item{La componente ''tam'' de e\_cola es igual a la cantidad de nodos en el arbol.}
	\item{Todo nodo en el arbol tiene un unico padre, con excepcion de la raiz, que no tiene padre.}
	\item{La relacion de orden es total.}
	\item{Un nodo es mayor a otro si la componente ''pri'' del primero es mayor que la del segundo.}
	\item{Un nodo es menor a otro si la componente ''pri'' del primero es menor que la del segundo.}
	\item{No pueden haber dos nodos en el arbol que tengan el mismo numero en la componente ''seg''.}
	\item{Si dos nodos tienen el mismo numero en la componente ''pri'', se procede a verficar la 
	componente ''seg'' de ambos. El que tiene el mayor numero en dicha componente es el mayor, mientras
	que el otro es el menor.}
	\item{Para cada nodo, todos los elementos del subarabol que se encuentra a la derecha de la raiz
	son mayores que la misma.}
	\item{Para cada nodo, todos los elementos del subarabol que se encuentra a la izquierda de la raiz
	son menores que la misma.}
	\item{La componente ''alt'' de cada nodo es igual a la cantidad de niveles que hay que recorrer
	para llegar a la hoja mas lejana.}
	\item{Para cada nodo, la diferencia en modulo de la altura entre los dos subarboles del mismo no
	puede diferir en mas de 1.}
\end{enumerate}

\vspace{2em}

\textbf{Abs:}

\vspace{1em}

\Abs[colaP]{colaPr}[c]{p}{mismosProximos(c, p)}

\vspace{1em}

\tadOperacion{mismosProximos}{$\langle$puntero(nodo), nat$\rangle$}{bool}{}

\vspace{1em}

\tadAxioma{mismosProximos(c, p)}{\IF $\pi_1$(c) $=$ NULL $\wedge$ vacia?(p)
																	THEN true ELSE {\IF
																	($\pi_1$(c) $=$ NULL $\wedge$ $\neg$vacia?(p)) $\vee$
																	($\pi_1$(c) $\neq$ NULL $\wedge$ vacia?(p))
																	THEN false ELSE {\IF
																	maxElem(*($\pi_1$(c))) $=$ proximo(p)
																	THEN mismosProximos(borrarMax(*($\pi_1$(c)), 
																	borrar($\pi_1$(proximo(p)), $\pi_2$(proximo(p)), p) ELSE
																	false FI}FI}FI}
\vspace{1em}

Las funciones ''maxElem'' y ''borrarMax'' no han sido axiomatizadas. Ya que estamos trabajando con Arboles Binarios de Busqueda (en nuestro caso AVL) la logica de ambas funciones es la misma que esta expresada en el pseudocodigo del modulo. En particular ''maxElem'' se limita a buscar el nodo mas a la derecha del arbol, mientras que ''borrarMax'' una vez encontrado el maximo, procede a eliminarlo y reordenar el arbol.

\end{Representacion}

\subsection{Algoritmos}

\begin{Algoritmos}

\Algoritmo{IVacio}{}{colaP}{
	\State var res: colaP $\gets$ tupla(NULL, 0) \ComplejidadDer{1}
	}{\Complejidad{1}}{}

\Algoritmo{IAgregar}{\Inout{c}{colaP}, \In{a}{nat}, \In{b}{nat}}{}{
	\If{c.raiz $==$ NULL}
		\State c.raiz $\gets$ \&(tupla(a, b, NULL, NULL, NULL, 1)) \ComplejidadDer{1}
		\State c.tam $\gets$ 1 \ComplejidadDer{1}
	\Else
		\State var seguir: bool $\gets$ true \ComplejidadDer{1}
		\State var pNodo: puntero(nodo) $\gets$ c.raiz \ComplejidadDer{1}
		\State var camino: arreglo[$\lfloor log_2(c.tam)\rfloor + 1$] de puntero(nodo) 
		\ComplejidadDer{\lfloor log_2(c.tam)\rfloor + 1}
		\State var nroCamino: nat 
		\State camino[0] $\gets$ pNodo \ComplejidadDer{1}
		\State nroCamino $\gets$ 0 \ComplejidadDer{1}
		\While{seguir $==$ true} \ComplejidadDer{1}
			\If{a $\geq$ *(pNodo).pri} \ComplejidadDer{1}
				\If{a $==$ *(pNodo).pri} \ComplejidadDer{1}
					\If{b $>$ *(pNodo).seg} \ComplejidadDer{1}
						\If{*(pNodo).der $!=$ NULL} \ComplejidadDer{1}
							\State pNodo $\gets$ *(pNodo).der \ComplejidadDer{1}
							\State nroCamino $\gets$ nroCamino $+$ 1 \ComplejidadDer{1}
							\State camino[nroCamino] $\gets$ pNodo \ComplejidadDer{1}
						\Else
							\State *(pNodo).der $\gets$ \&(tupla(a, b, pNodo, NULL, NULL, 1)) \ComplejidadDer{1}
							\State nroCamino $\gets$ nroCamino $+$ 1 \ComplejidadDer{1}
							\State camino[nroCamino] $\gets$ *(pNodo).der \ComplejidadDer{1}
							\State seguir $\gets$ false \ComplejidadDer{1}
						\EndIf
					\Else
						\If{*(pNodo).izq $!=$ NULL} \ComplejidadDer{1}
							\State pNodo $\gets$ *(pNodo).izq \ComplejidadDer{1}
							\State nroCamino $\gets$ nroCamino $+$ 1 \ComplejidadDer{1}
							\State camino[nroCamino] $\gets$ pNodo \ComplejidadDer{1}
						\Else
							\State *(pNodo).izq $\gets$ \&(tupla(a, b, pNodo, NULL, NULL, 1)) \ComplejidadDer{1}
							\State nroCamino $\gets$ nroCamino $+$ 1 \ComplejidadDer{1}
							\State camino[nroCamino] $\gets$ *(pNodo).izq \ComplejidadDer{1}
							\State seguir $\gets$ false \ComplejidadDer{1}
						\EndIf
					\EndIf
				\Else
					\If{*(pNodo).der $!=$ NULL} \ComplejidadDer{1}
						\State pNodo $\gets$ *(pNodo).der \ComplejidadDer{1}
						\State nroCamino $\gets$ nroCamino $+$ 1 \ComplejidadDer{1}
						\State camino[nroCamino] $\gets$ pNodo \ComplejidadDer{1}
					\Else
						\State *(pNodo).der $\gets$ \&(tupla(a, b, pNodo, NULL, NULL, 1)) \ComplejidadDer{1}
						\State nroCamino $\gets$ nroCamino $+$ 1 \ComplejidadDer{1}
						\State camino[nroCamino] $\gets$ *(pNodo).der \ComplejidadDer{1}
						\State seguir $\gets$ false \ComplejidadDer{1}
					\EndIf					
				\EndIf
			\Else
				\If{*(pNodo).izq $!=$ NULL} \ComplejidadDer{1}
					\State pNodo $\gets$ *(pNodo).izq \ComplejidadDer{1}
					\State nroCamino $\gets$ nroCamino $+$ 1 \ComplejidadDer{1}
					\State camino[nroCamino] $\gets$ pNodo \ComplejidadDer{1}
				\Else
					\State *(pNodo).izq $\gets$ \&(tupla(a, b, pNodo, NULL, NULL, 1)) \ComplejidadDer{1}
					\State nroCamino $\gets$ nroCamino $+$ 1 \ComplejidadDer{1}
					\State camino[nroCamino] $\gets$ *(pNodo).izq \ComplejidadDer{1}
					\State seguir $\gets$ false \ComplejidadDer{1}
				\EndIf
			\EndIf
		\EndWhile
		\State c.tam $\gets$ c.tam $+$ 1 \ComplejidadDer{1}
		\State seguir $\gets$ true \ComplejidadDer{1}
		\While{nroCamino $\geq$ 0 $\wedge$ seguir $==$ true} \ComplejidadDer{\lfloor log_2\ N\rfloor + 1}
			\State pNodo $\gets$ camino[nroCamino] \ComplejidadDer{1}
			\State *(pNodo).alt $\gets$ \NombreFuncion{Altura}(pNodo) v
			\If{$|$\NombreFuncion{FactorDesbalance}(camino[nroCamino])$|$ $>$ 1} \ComplejidadDer{1}
				\State pNodo $\gets$ \NombreFuncion{Rotar}(\NombreFuncion{HijoMasAlto}
				(\NombreFuncion{HijoMasAlto}(pNodo)), HijoMasAlto(pNodo), pNodo) \ComplejidadDer{1}
				\State *(*(pNodo).izq).alt $\gets$ \NombreFuncion{Altura}(*(pNodo).izq) \ComplejidadDer{1}
				\State *(*(pNodo).der).alt $\gets$ \NombreFuncion{Altura}(*(pNodo).der) \ComplejidadDer{1}
				\State *(pNodo).alt $\gets$ \NombreFuncion{Altura}(*(pNodo)) \ComplejidadDer{1}
				\State seguir $\gets$ false \ComplejidadDer{1}
			\EndIf
			\State nroCamino $\gets$ nroCamino $-$ 1 \ComplejidadDer{1}
		\EndWhile
	\EndIf
	}{\Complejidad{1}}{
	Debido a la longitud del pseudocodigo, vamos a ignorar todas los condicionales y las asignaciones en la justificacion, ya que se realizan en tiempo constante. Solo nos vamos a centrar en dos puntos, la creacion del arreglo ''camino'' y el ultimo ciclo.\\
	Para el arreglo asignamos esa cantidad de nodos ya que contamos con un Arbol balanceado, el cual como mucho puede necesitar de $\lfloor log_2\ N\rfloor + 1$ niveles para almacenar $N$ nodos. Esta misma logica la utilizamos en el ultimo ciclo, en el cual para restaurar el balance del Arbol recorremos el mismo desde el ultimo nodo agregado (el cual es una hoja) hasta la raiz en el peor caso, corrigiendo cualquier desbalance en el camino.\\
	Esto resulta en la siguiente suma:\\
	$\Complejidad{\lfloor log_2\ N\rfloor + 1} + \Complejidad{\lfloor log_2\ N\rfloor + 1} =\newline
	2 * \Complejidad{\lfloor log_2\ N\rfloor + 1} =\newline
	\Complejidad{\lfloor log_2\ N\rfloor + 1} =\newline 
	\Complejidad{\lfloor log_2\ N\rfloor} = \Complejidad{log_2\ N}$
	}
	
\Algoritmo{IEliminar}{\Inout{c}{colaP}, \In{a}{nat}, \In{b}{nat}}{}{
	\State var pNodo: puntero(nodo) $\gets$ c.raiz \ComplejidadDer{1}
	\State ver seguir: bool $\gets$ true \ComplejidadDer{1}
	\While{pNodo $!=$ NULL $\wedge$ seguir $==$ true} \ComplejidadDer{\lfloor log_2\ N\rfloor + 1}
		\If{*(pNodo).pri $==$ a $\wedge$ *(pNodo).seg $==$ b} \ComplejidadDer{1}
			\State seguir $\gets$ false \ComplejidadDer{1}
		\Else
			\If{a $\geq$ *(pNodo).pri} \ComplejidadDer{1}
				\If{a $==$ *(pNodo).pri} \ComplejidadDer{1}
					\If{b $>$ *(pNodo).seg} \ComplejidadDer{1}
						\State pNodo $\gets$ *(pNodo).der \ComplejidadDer{1}
					\Else
						\State pNodo $\gets$ *(pNodo).izq \ComplejidadDer{1}
					\EndIf
				\Else
					 \State pNodo $\gets$ *(pNodo).der \ComplejidadDer{1}
				\EndIf
			\Else
				\State pNodo $\gets$ *(pNodo).izq \ComplejidadDer{1}
			\EndIf
		\EndIf
	\EndWhile
	\If{pNodo != NULL} \ComplejidadDer{1}
		\State bNodo: puntero(nodo) $\gets$ NULL \ComplejidadDer{1}
		\If{pNodo $==$ c.raiz} \ComplejidadDer{1}
			\If{*(pNodo).izq $==$ NULL $\wedge$ *(pNodo).der $==$ NULL} \ComplejidadDer{1}
				\State c.raiz $\gets$ NULL \ComplejidadDer{1}
				\State delete pNodo \ComplejidadDer{1}
			\EndIf
			\If{*(pNodo).izq $!=$ NULL $\wedge$ *(pNodo).der $==$ NULL} \ComplejidadDer{1}
				\State c.raiz $\gets$ *(pNodo).izq \ComplejidadDer{1}
				\State *(*(pNodo).izq).padre $\gets$ NULL \ComplejidadDer{1}
				\State delete pNodo \ComplejidadDer{1}
			\EndIf
			\If{*(pNodo).izq $==$ NULL $\wedge$ *(pNodo).der $!=$ NULL} \ComplejidadDer{1}
				\State c.raiz $\gets$ *(pNodo).der \ComplejidadDer{1}
				\State *(*(pNodo).der).padre $\gets$ NULL \ComplejidadDer{1}
				\State delete pNodo \ComplejidadDer{1}
			\EndIf
			\If{*(pNodo).izq $!=$ NULL $\wedge$ *(pNodo).der $!=$ NULL} \ComplejidadDer{1}
				\State var tNodo: puntero(nodo) $\gets$ pNodo.der \ComplejidadDer{1}
				\While{*(tNodo).izq != NULL} \ComplejidadDer{\lfloor log_2\ N\rfloor + 1}
					\State tNodo $\gets$ tNodo.izq \ComplejidadDer{1}
				\EndWhile
				\State bNodo $\gets$ *(tNodo).padre \ComplejidadDer{1}
				\If{*(tNodo).der $!=$ NULL} \ComplejidadDer{1}
					\State *(*(tNodo).der).padre $\gets$ *(tNodo).padre \ComplejidadDer{1}
				\EndIf
				\If{*(*(tNodo).padre).izq $==$ tNodo} \ComplejidadDer{1}
					\State *(*(tNodo).padre).izq $\gets$ *(tNodo).der \ComplejidadDer{1}
				\Else
					\State *(*(tNodo).padre).der $\gets$ *(tNodo).der \ComplejidadDer{1}
				\EndIf
				\State *(tNodo).padre $\gets$ *(pNodo).padre \ComplejidadDer{1}
				\State *(tNodo).izq $\gets$ *(pNodo).izq \ComplejidadDer{1}
				\State *(tNodo).der $\gets$ *(pNodo).der \ComplejidadDer{1}
				\State *(*(pNodo).izq).padre $\gets$ tNodo \ComplejidadDer{1}
				\State *(*(pNodo).der).padre $\gets$ tNodo \ComplejidadDer{1}
				\State delete pNodo \ComplejidadDer{1}
			\EndIf
		\Else
			\If{*(pNodo).izq $==$ NULL $\wedge$ *(pNodo).der $==$ NULL} \ComplejidadDer{1}
				\If{*(*(pNodo).padre).izq $==$ pNodo} \ComplejidadDer{1}
					\State *(*(pNodo).padre).izq $\gets$ NULL \ComplejidadDer{1}
					\State bNodo $\gets$ *(pNodo).padre \ComplejidadDer{1}
					\State delete pNodo \ComplejidadDer{1}
				\Else
					\State *(*(pNodo).padre).der $\gets$ NULL \ComplejidadDer{1}
					\State bNodo $\gets$ *(pNodo).padre \ComplejidadDer{1}
					\State delete pNodo \ComplejidadDer{1}
				\EndIf
			\EndIf
			\If{*(pNodo).izq $!=$ NULL $\wedge$ *(pNodo).der $==$ NULL} \ComplejidadDer{1}
				\If{*(*(pNodo).padre).izq $==$ pNodo} \ComplejidadDer{1}
					\State *(*(pNodo).padre).izq $\gets$ *(pNodo).izq \ComplejidadDer{1}
					\State *(*(pNodo).izq).padre $\gets$ *(pNodo).padre \ComplejidadDer{1}
					\State bNodo $\gets$ *(pNodo).padre \ComplejidadDer{1}
					\State delete pNodo \ComplejidadDer{1}
				\Else
					\State *(*(pNodo).padre).der $\gets$ *(pNodo).izq \ComplejidadDer{1}
					\State *(*(pNodo).izq).padre $\gets$ *(pNodo).padre \ComplejidadDer{1}
					\State bNodo $\gets$ *(pNodo).padre \ComplejidadDer{1}
					\State delete pNodo \ComplejidadDer{1}
				\EndIf
			\EndIf
			\If{*(pNodo).izq $==$ NULL $\wedge$ *(pNodo).der $!=$ NULL} \ComplejidadDer{1}
				\If{*(*(pNodo).padre).izq $==$ pNodo} \ComplejidadDer{1}
					\State *(*(pNodo).padre).izq $\gets$ *(pNodo).der \ComplejidadDer{1}
					\State *(*(pNodo).der).padre $\gets$ *(pNodo).padre \ComplejidadDer{1}
					\State bNodo $\gets$ *(pNodo).padre \ComplejidadDer{1}
					\State delete pNodo \ComplejidadDer{1}
				\Else
					\State *(*(pNodo).padre).der $\gets$ *(pNodo).der \ComplejidadDer{1}
					\State *(*(pNodo).der).padre $\gets$ *(pNodo).padre \ComplejidadDer{1}
					\State bNodo $\gets$ *(pNodo).padre \ComplejidadDer{1}
					\State delete pNodo \ComplejidadDer{1}
				\EndIf
			\EndIf
			\If{*(pNodo).izq $!=$ NULL $\wedge$ *(pNodo).der $!=$ NULL} \ComplejidadDer{1}
				\State var tNodo: puntero(nodo) $\gets$ pNodo.der \ComplejidadDer{1}
				\While{*(tNodo).izq != NULL} \ComplejidadDer{\lfloor log_2\ N\rfloor + 1}
					\State tNodo $\gets$ tNodo.izq \ComplejidadDer{1}
				\EndWhile
				\State bNodo $\gets$ *(tNodo).padre \ComplejidadDer{1}
				\If{*(tNodo).der $!=$ NULL} \ComplejidadDer{1}
					\State *(*(tNodo).der).padre $\gets$ *(tNodo).padre \ComplejidadDer{1}
				\EndIf
				\If{*(*(tNodo).padre).izq $==$ tNodo} \ComplejidadDer{1}
					\State *(*(tNodo).padre).izq $\gets$ *(tNodo).der \ComplejidadDer{1}
				\Else
					\State *(*(tNodo).padre).der $\gets$ *(tNodo).der \ComplejidadDer{1}
				\EndIf
				\If{*(*(pNodo).padre).izq $==$ pNodo} \ComplejidadDer{1}
					\State *(*(tNodo).padre).izq $\gets$ tNodo \ComplejidadDer{1}
				\Else
					\State *(*(tNodo).padre).der $\gets$ pNodo \ComplejidadDer{1}
				\EndIf
				\State *(tNodo).padre $\gets$ *(pNodo).padre \ComplejidadDer{1}
				\State *(tNodo).izq $\gets$ *(pNodo).izq \ComplejidadDer{1}
				\State *(tNodo).der $\gets$ *(pNodo).der \ComplejidadDer{1}
				\State *(*(pNodo).izq).padre $\gets$ tNodo \ComplejidadDer{1}
				\State *(*(pNodo).der).padre $\gets$ tNodo \ComplejidadDer{1}
				\State delete pNodo \ComplejidadDer{1}
			\EndIf
		\State c.tam $\gets$ c.tam $+$ 1
		\While{b $!=$ NULL} \ComplejidadDer{\lfloor log_2\ N\rfloor + 1}
			\State *(b).alt $\gets$ \NombreFuncion{SetAltura}(b) \ComplejidadDer{1}
			\If{|\NombreFuncion{FactorDeDesbalance}(b)| $>$ 1} \ComplejidadDer{1}
				\State *(b).alt $\gets$ \NombreFuncion{Rotar}(\NombreFuncion{HijoMasAlto}
				(\NombreFuncion{HijoMasAlto}(b)), \NombreFuncion{HijoMasAlto}(b), b) \ComplejidadDer{1}
				\State *(*(b).izq).alt $\gets$ \NombreFuncion{SetAltura}(*(b).izq) \ComplejidadDer{1}
				\State *(*(b).der).alt $\gets$ \NombreFuncion{SetAltura}(*(b).der) \ComplejidadDer{1}
				\State *(b).alt $\gets$ \NombreFuncion{SetAltura}(*(b)) \ComplejidadDer{1}
			\EndIf
			\State b $\gets$ *(b).padre \ComplejidadDer{1}
		\EndWhile
		\EndIf	
	\EndIf
	
	}{\Complejidad{log_2\ N}}{
	Para la complejidad de este algoritmo nos vamos a remitir al mismo proceso que en el caso anterior, vamos a ignorar los condicionales y las asignaciones ya que estas se realizan en tiempo constante para centrarnos unicamente en los ciclos cuya complejidad depende de algun parametro.\\
	En este algoritmo contamos con 4 ciclos que dependen de alguna variable, ninguno esta anidado con ningun otro ciclo, y en el peor caso solo recorremos 3 de ellos.\\
	Esto nos da la siguiente suma:\\
	$3 * \Complejidad{\lfloor log_2\ N\rfloor + 1} =\newline
	\Complejidad{\lfloor log_2\ N\rfloor + 1} =\newline
	\Complejidad{\lfloor log_2\ N\rfloor} = \Complejidad{log_2\ N}$
	
	}

\Algoritmo{IRotar}{\Inout{c}{colaP}, \In{p1}{puntero(nodo)}, \In{p2}{puntero(nodo)}, \In{p3}{puntero(nodo)}}{puntero(nodo)}{
	\State var t1: puntero(nodo) $\gets$ NULL \ComplejidadDer{1}
	\State var t2: puntero(nodo) $\gets$ NULL \ComplejidadDer{1}
	\State var t2: puntero(nodo) $\gets$ NULL v
	\If{(*(p3).pri $\leq$ *(p1).pri $\wedge$ *(p3).seg $<$ *(p1).seg) $\wedge$\\
	\	\ \ (*(p1).pri $\leq$ *(p2).pri $\wedge$ *(p1).pri $<$ *(p2).pri)} \ComplejidadDer{1}
		\State t1 $\gets$ p3 \ComplejidadDer{1}
		\State t2 $\gets$ p1 \ComplejidadDer{1}
		\State t3 $\gets$ p2 \ComplejidadDer{1}
	\EndIf
	\If{(*(p3).pri $\geq$ *(p1).pri $\wedge$ *(p3).seg $>$ *(p1).seg) $\wedge$\\
	\ \	\ (*(p1).pri $\geq$ *(p2).pri $\wedge$ *(p1).pri $>$ *(p2).pri)} \ComplejidadDer{1}
		\State t1 $\gets$ p2 \ComplejidadDer{1}
		\State t2 $\gets$ p1 \ComplejidadDer{1}
		\State t3 $\gets$ p3 \ComplejidadDer{1}
	\EndIf
	\If{(*(p3).pri $\leq$ *(p2).pri $\wedge$ *(p3).seg $<$ *(p2).seg) $\wedge$\\
	\ \	\ (*(p2).pri $\geq$ *(p1).pri $\wedge$ *(p2).pri $<$ *(p1).pri)} \ComplejidadDer{1}
		\State t1 $\gets$ p3 \ComplejidadDer{1}
		\State t2 $\gets$ p2 \ComplejidadDer{1}
		\State t3 $\gets$ p1 \ComplejidadDer{1}
	\EndIf
	\If{(*(p3).pri $\geq$ *(p2).pri $\wedge$ *(p3).seg $>$ *(p2).seg) $\wedge$\\
	\ \ \	(*(p2).pri $\geq$ *(p3).pri $\wedge$ *(p2).pri $>$ *(p3).pri)} \ComplejidadDer{1}
		\State t1 $\gets$ p1 \ComplejidadDer{1}
		\State t2 $\gets$ p2 \ComplejidadDer{1}
		\State t3 $\gets$ p3 \ComplejidadDer{1}
	\EndIf
	\If{c.raiz == p3}
		\State c.raiz $\gets$ p3 \ComplejidadDer{1}
		\State *(p3).padre $\gets$ NULL \ComplejidadDer{1}
	\Else
		\If{*(*(p3).padre).izq = p3} \ComplejidadDer{1}
			\State \NombreFuncion{Cizq}(*(p3).padre, t2) \ComplejidadDer{1}
		\Else
			\State \NombreFuncion{Cder}(*(p3).padre, t2) \ComplejidadDer{1}
		\EndIf
	\EndIf
	\If{*(t2).izq != p1 $\wedge$ *(t2).izq != p2 $\wedge$ *(t2).izq != p3}
		\State \NombreFuncion{Cder}(t1, *(t2).izq) \ComplejidadDer{1}
	\EndIf
	\If{*(t2).der != p1 $\wedge$ *(t2).der != p2 $\wedge$ *(t2).der != p3}
		\State \NombreFuncion{Cder}(t3, *(t2).der) \ComplejidadDer{1}
	\EndIf
	\State \NombreFuncion{Cizq}(t2, t1) \ComplejidadDer{1}
	\State \NombreFuncion{Cder}(t2, t3) \ComplejidadDer{1}
	\State res $\gets$ t2 \ComplejidadDer{1}
	}{\Complejidad{1}}{
	Al igual que en los dos casos anteriores, debido a la longitud del pseudocodigo, vamos a ignorar los condicionales y las asignaciones ya que se realizan en tiempo constante.\newline
	Como todas las ejecuciones del codigo se efectuan en tiempo constante, podemos ver de manera trivial que la complejidad es $\Complejidad{1}$.
	}
	
%\Algoritmo{IEliminar}{\Inout{c}{colaP}, \In{a}{nat}, \In{b}{nat}}{}{
	%\State var pNodo: puntero(nodo) $\gets$ c.raiz
	%\State ver seguir: bool $\gets$ true
	%\While{pNodo $!=$ NULL $\wedge$ seguir $==$ true}
		%\If{*(pNodo).pri $==$ a $\wedge$ *(pNodo).seg $==$ b}
			%\State seguir $\gets$ false
		%\Else
			%\If{a $\geq$ *(pNodo).pri}
				%\If{a $==$ *(pNodo).pri}
					%\If{b $>$ *(pNodo).seg}
						%\State pNodo $\gets$ *(pNodo).der
					%\Else
						%\State pNodo $\gets$ *(pNodo).izq
					%\EndIf
				%\Else
					 %\State pNodo $\gets$ *(pNodo).der
				%\EndIf
			%\Else
				%\State pNodo $\gets$ *(pNodo).izq
			%\EndIf
		%\EndIf
	%\EndWhile
	%\If{pNodo != NULL}
		%\State bNodo: puntero(nodo) $\gets$ NULL
		%\If{pNodo $==$ c.raiz}
			%\If{*(pNodo).izq $==$ NULL $\wedge$ *(pNodo).der $==$ NULL}
				%\State c.raiz $\gets$ NULL
				%\State delete pNodo
			%\EndIf
			%\If{*(pNodo).izq $!=$ NULL $\wedge$ *(pNodo).der $==$ NULL}
				%\State c.raiz $\gets$ *(pNodo).izq
				%\State *(*(pNodo).izq).padre $\gets$ NULL
				%\State delete pNodo
			%\EndIf
			%\If{*(pNodo).izq $==$ NULL $\wedge$ *(pNodo).der $!=$ NULL}
				%\State c.raiz $\gets$ *(pNodo).der
				%\State *(*(pNodo).der).padre $\gets$ NULL
				%\State delete pNodo
			%\EndIf
			%\If{*(pNodo).izq $!=$ NULL $\wedge$ *(pNodo).der $!=$ NULL}
				%\State var tNodo: puntero(nodo) $\gets$ pNodo.der
				%\While{*(tNodo).izq != NULL}
					%\State tNodo $\gets$ tNodo.izq
				%\EndWhile
				%\State bNodo $\gets$ *(tNodo).padre
				%\If{*(tNodo).der $!=$ NULL}
					%\State *(*(tNodo).der).padre $\gets$ *(tNodo).padre
				%\EndIf
				%\If{*(*(tNodo).padre).izq $==$ tNodo}
					%\State *(*(tNodo).padre).izq $\gets$ *(tNodo).der
				%\Else
					%\State *(*(tNodo).padre).der $\gets$ *(tNodo).der
				%\EndIf
				%\State *(tNodo).padre $\gets$ *(pNodo).padre
				%\State *(tNodo).izq $\gets$ *(pNodo).izq
				%\State *(tNodo).der $\gets$ *(pNodo).der
				%\State *(*(pNodo).izq).padre $\gets$ tNodo
				%\State *(*(pNodo).der).padre $\gets$ tNodo
				%\State delete pNodo
			%\EndIf
		%\Else
			%\If{*(pNodo).izq == NULL $\wedge$ *(pNodo).der == NULL}
				%\If{*(*(pNodo).padre).izq == pNodo}
					%\State *(*(pNodo).padre).izq $\gets$ NULL
					%\State bNodo $\gets$ *(pNodo).padre
					%\State delete pNodo
				%\Else
					%\State *(*(pNodo).padre).der $\gets$ NULL
					%\State bNodo $\gets$ *(pNodo).padre
					%\State delete pNodo
				%\EndIf
			%\EndIf
			%\If{*(pNodo).izq != NULL $\wedge$ *(pNodo).der == NULL}
				%\If{*(*(pNodo).padre).izq == pNodo}
					%\State *(*(pNodo).padre).izq $\gets$ *(pNodo).izq
					%\State *(*(pNodo).izq).padre $\gets$ *(pNodo).padre
					%\State bNodo $\gets$ *(pNodo).padre
					%\State delete pNodo
				%\Else
					%\State *(*(pNodo).padre).der $\gets$ *(pNodo).izq
					%\State *(*(pNodo).izq).padre $\gets$ *(pNodo).padre
					%\State bNodo $\gets$ *(pNodo).padre
					%\State delete pNodo
				%\EndIf
			%\EndIf
			%\If{*(pNodo).izq == NULL $\wedge$ *(pNodo).der != NULL}
				%\If{*(*(pNodo).padre).izq == pNodo}
					%\State *(*(pNodo).padre).izq $\gets$ *(pNodo).der
					%\State *(*(pNodo).der).padre $\gets$ *(pNodo).padre
					%\State bNodo $\gets$ *(pNodo).padre
					%\State delete pNodo
				%\Else
					%\State *(*(pNodo).padre).der $\gets$ *(pNodo).der
					%\State *(*(pNodo).der).padre $\gets$ *(pNodo).padre
					%\State bNodo $\gets$ *(pNodo).padre
					%\State delete pNodo
				%\EndIf
			%\EndIf
			%\If{*(pNodo).izq $!=$ NULL $\wedge$ *(pNodo).der $!=$ NULL}
				%\State var tNodo: puntero(nodo) $\gets$ pNodo.der
				%\While{*(tNodo).izq != NULL}
					%\State tNodo $\gets$ tNodo.izq
				%\EndWhile
				%\State bNodo $\gets$ *(tNodo).padre
				%\If{*(tNodo).der $!=$ NULL}
					%\State *(*(tNodo).der).padre $\gets$ *(tNodo).padre
				%\EndIf
				%\If{*(*(tNodo).padre).izq $==$ tNodo}
					%\State *(*(tNodo).padre).izq $\gets$ *(tNodo).der
				%\Else
					%\State *(*(tNodo).padre).der $\gets$ *(tNodo).der
				%\EndIf
				%\If{*(*(pNodo).padre).izq $==$ pNodo}
					%\State *(*(tNodo).padre).izq $\gets$ tNodo
				%\Else
					%\State *(*(tNodo).padre).der $\gets$ pNodo
				%\EndIf
				%\State *(tNodo).padre $\gets$ *(pNodo).padre
				%\State *(tNodo).izq $\gets$ *(pNodo).izq
				%\State *(tNodo).der $\gets$ *(pNodo).der
				%\State *(*(pNodo).izq).padre $\gets$ tNodo
				%\State *(*(pNodo).der).padre $\gets$ tNodo
				%\State delete pNodo
			%\EndIf
		%\While{b $!=$ NULL}
			%\State SetAltura(b)
			%\If{|}
		%\EndWhile
		%\EndIf	
	%\EndIf
	%
	%}{}{
	%}
	
\Algoritmo{ICIzq}{\In{a}{puntero(nodo)}, \In{b}{puntero(nodo)}}{}{
	\State *(a).izq = b \ComplejidadDer{1}
	\State *(b).padre = a \ComplejidadDer{1}
	}{\Complejidad{1}}{
	$\Complejidad{1} + \Complejidad{1} =\newline
	2 * \Complejidad{1} =\newline
	\Complejidad{1}$
	}
	
\Algoritmo{ICDer}{\In{a}{puntero(nodo)}, \In{b}{puntero(nodo)}}{}{
	\State *(a).der = b \ComplejidadDer{1}
	\State *(b).padre = a \ComplejidadDer{1}
	}{\Complejidad{1}}{
	$\Complejidad{1} + \Complejidad{1} =\newline
	2 * \Complejidad{1} =\newline
	\Complejidad{1}$
	}
	
\Algoritmo{ISetAltura}{\In{a}{puntero(nodo)}}{nat}{
	\If{*(a).izq == NULL} \ComplejidadDer{1}
		\If{*(a).der == NULL} \ComplejidadDer{1}
			\State res $\gets$ 1 \ComplejidadDer{1}
		\Else
			\State res $\gets$ 1 $+$ *(*(a).der).alt \ComplejidadDer{1}
		\EndIf
	\Else
		\If{*(a).der == NULL} \ComplejidadDer{1}
			\State res $\gets$ 1 $+$ *(*(a).izq).alt \ComplejidadDer{1}
		\Else
			\If{*(*(a).izq).alt $>$ *(*(a).der).alt} \ComplejidadDer{1}
				\State res $\gets$ 1 $+$ *(*(a).izq).alt \ComplejidadDer{1}
			\Else
				\State res $\gets$ 1 $+$ *(*(a).der).alt \ComplejidadDer{1}
			\EndIf
		\EndIf
	\EndIf
	}{\Complejidad{1}}{
	$\Complejidad{1} + max(\Complejidad{1} + max(\Complejidad{1}, \Complejidad{1}),
	\Complejidad{1} + max(\Complejidad{1}, \Complejidad{1} + max(\Complejidad{1}, \Complejidad{1}))) =
	\newline
	\Complejidad{1} + max(\Complejidad{1} + max(\Complejidad{1}, \Complejidad{1}),
	\Complejidad{1} + max(\Complejidad{1}, \Complejidad{1} + \Complejidad{1})) =\newline
	\Complejidad{1} + max(\Complejidad{1} + \Complejidad{1},
	\Complejidad{1} + max(\Complejidad{1}, 2 * \Complejidad{1})) =\newline
	\Complejidad{1} + max(2 * \Complejidad{1}, 3 * \Complejidad{1}) =\newline
	\Complejidad{1} + 3 * \Complejidad{1} =	4 * \Complejidad{1} = \Complejidad{1}$
	}
	
\Algoritmo{IFactorDesbalance}{\In{a}{puntero(nodo)}}{int}{
	\If{*(a).izq == NULL} \ComplejidadDer{1}
		\If{*(a).der == NULL} \ComplejidadDer{1}
			\State res $\gets$ 0 \ComplejidadDer{1}
		\Else
			\State res $\gets$ -(*(*(a).der).alt) \ComplejidadDer{1}
		\EndIf
	\Else
		\If{*(a).der == NULL} \ComplejidadDer{1}
			\State res $\gets$ *(*(a).izq).alt \ComplejidadDer{1}
		\Else
			\State res $\gets$ *(*(a).izq).alt - *(*(a).der).alt \ComplejidadDer{1}
		\EndIf
	\EndIf
	}{\Complejidad{1}}{
	$\Complejidad{1} + max(\Complejidad{1} + (max(\Complejidad{1}), max(\Complejidad{1})),
	\Complejidad{1} + (max(\Complejidad{1}), max(\Complejidad{1}))) =\newline
	\Complejidad{1} + max(\Complejidad{1} + \Complejidad{1},
	\Complejidad{1} + \Complejidad{1}) =\newline
	\Complejidad{1} + max(2 * \Complejidad{1}, 2 * \Complejidad{1}) =\newline
	\Complejidad{1} + 2 * \Complejidad{1} =\newline
	3 * \Complejidad{1} + \Complejidad{1}$
	}
	
\Algoritmo{IHijoMasAlto}{\In{a}{puntero(nodo)}}{puntero(nodo)}{
	\If{*(*(a).izq).alt > *(*(a).der).alt} \ComplejidadDer{1}
		\State res $\gets$ *(a).der \ComplejidadDer{1}
	\Else
		\State res $\gets$ *(a).izq \ComplejidadDer{1}
	\EndIf
	}{\Complejidad{1}}{
	$\Complejidad{1} + max(\Complejidad{1}, \Complejidad{1}) =\newline
	\Complejidad{1} + \Complejidad{1} =\newline
	2 * \Complejidad{1} = \Complejidad{1}$
	}
	
\Algoritmo{IVacia?}{\In{c}{colaP}}{bool}{
	\State res $\gets$ c.raiz $!=$ NULL \ComplejidadDer{1}
	}{\Complejidad{1}}{
	}

\Algoritmo{IDesencolar}{\Inout{c}{colaP}}{tupla(a: nat, b: nat)}{
	\State var pNodo: puntero(nodo) $\gets$ c.raiz \ComplejidadDer{1}
	\While{pNodo.der != NULL} \ComplejidadDer{log_2\ N}
		\State pNodo $\gets$ pNodo.der \ComplejidadDer{1}
	\EndWhile
	\State \NombreFuncion{ELIMINAR}(c, *(pNodo).pri, *(pNodo).seg) \ComplejidadDer{log_2\ N}
	}{\Complejidad{log_2\ N}}{
	$\Complejidad{1} + \Complejidad{log_2\ N} * \Complejidad{1} + \Complejidad{log_2\ N} =\newline$
	$2 * \Complejidad{log_2\ N} = \Complejidad{log_2\ N}$
	}

\Algoritmo{ITama�o}{\Inout{c}{colaP}}{nat}{
	\State res $\gets$ c.tam \ComplejidadDer{1}
	}{\Complejidad{1}}{
	}

\end{Algoritmos}

\end{document}
