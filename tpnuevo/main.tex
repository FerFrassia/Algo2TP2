\documentclass[a4paper,10pt]{article}
\usepackage[paper=a4paper, hmargin=1.5cm, bottom=1.5cm, top=3.5cm]{geometry}
\usepackage[latin1]{inputenc}
\usepackage[T1]{fontenc}
\usepackage[spanish]{babel}
\usepackage[framemethod=tikz]{mdframed}
\usepackage[T1]{fontenc}
\usepackage{xspace}
\usepackage{xargs}
\usepackage{ifthen}
\usepackage{fancyhdr}
\usepackage{lastpage}
\usepackage{aed2-tad,aed2-symb,aed2-itef}
\usepackage{enumitem}
\usepackage{algorithm}
\usepackage{algpseudocode}
\usepackage{scrextend}
\usepackage{framed}
%\usepackage[noend]{algpseudocode}

\newcommand{\moduloNombre}[1]{\textbf{#1}}

\let\NombreFuncion=\textsc
\let\TipoVariable=\texttt
\let\ModificadorArgumento=\textbf
\newcommand{\res}{$res$\xspace}
\newcommand{\tab}{\hspace*{7mm}}

\newcommandx{\TipoFuncion}[3]{%
  \NombreFuncion{#1}(#2) \ifx#3\empty\else $\to$ \res\,: \TipoVariable{#3}\fi%
}
\newcommand{\In}[2]{\ModificadorArgumento{in} \ensuremath{#1}\,: \TipoVariable{#2}\xspace}
\newcommand{\Out}[2]{\ModificadorArgumento{out} \ensuremath{#1}\,: \TipoVariable{#2}\xspace}
\newcommand{\Inout}[2]{\ModificadorArgumento{in/out} \ensuremath{#1}\,: \TipoVariable{#2}\xspace}
\newcommand{\Aplicar}[2]{\NombreFuncion{#1}(#2)}

\newlength{\IntFuncionLengthA}
\newlength{\IntFuncionLengthB}
\newlength{\IntFuncionLengthC}
%InterfazFuncion(nombre, argumentos, valor retorno, precondicion, postcondicion, complejidad, descripcion, aliasing)
\newcommandx{\InterfazFuncion}[9][4=true,6,7,8,9]{%
  \hangindent=\parindent
  \TipoFuncion{#1}{#2}{#3}\\%
  \textbf{Pre} $\equiv$ \{#4\}\\%
  \textbf{Post} $\equiv$ \{#5\}%
  \ifx#6\empty\else\\\textbf{Complejidad:} #6\fi%
  \ifx#7\empty\else\\\textbf{Descripci�n:} #7\fi%
  \ifx#8\empty\else\\\textbf{Aliasing:} #8\fi%
  \ifx#9\empty\else\\\textbf{Requiere:} #9\fi%
}

\newenvironment{Interfaz}{%
  \parskip=2ex%
  \noindent\textbf{\Large Interfaz}%
  \par%
}{}

\newenvironment{Representacion}{%
  \vspace*{2ex}%
  \noindent\textbf{\Large Representaci�n}%
  \vspace*{2ex}%
}{}

\newenvironment{Algoritmos}{%
  \vspace*{2ex}%
  \noindent\textbf{\Large Algoritmos}%
  \vspace*{2ex}%
}{}

\newcommand{\Titulo}[1]{
  \vspace*{1ex}\par\noindent\textbf{\large #1}\par
}

\newenvironmentx{Estructura}[2][2={estr}]{%
  \par\vspace*{2ex}%
  \TipoVariable{#1} \textbf{se representa con} \TipoVariable{#2}%
  \par\vspace*{1ex}%
}{%
  \par\vspace*{2ex}%
}%

\newboolean{EstructuraHayItems}
\newlength{\lenTupla}
\newenvironmentx{Tupla}[1][1={estr}]{%
    \settowidth{\lenTupla}{\hspace*{3mm}donde \TipoVariable{#1} es \TipoVariable{tupla}$($}%
    \addtolength{\lenTupla}{\parindent}%
    \hspace*{3mm}donde \TipoVariable{#1} es \TipoVariable{tupla}$($%
    \begin{minipage}[t]{\linewidth-\lenTupla}%
    \setboolean{EstructuraHayItems}{false}%
}{%
    $)$%
    \end{minipage}
}

\newenvironmentx{Enum}[1][1={estr}]{%
    \settowidth{\lenTupla}{\hspace*{3mm}donde \TipoVariable{#1} es \TipoVariable{enum}$($}%
    \addtolength{\lenTupla}{\parindent}%
    \hspace*{3mm}donde \TipoVariable{#1} es \TipoVariable{enum}$($%
    \begin{minipage}[t]{\linewidth-\lenTupla}%
    \setboolean{EstructuraHayItems}{false}%
}{%
    $)$%
    \end{minipage}
}

\newcommandx{\tupItem}[3][1={\ }]{%
    %\hspace*{3mm}%
    \ifthenelse{\boolean{EstructuraHayItems}}{%
        ,#1%
    }{}%
    \emph{#2}: \TipoVariable{#3}%
    \setboolean{EstructuraHayItems}{true}%
		\ignorespaces%
}

\newcommandx{\tupTupItem}[3][1={\ }]{%
    %\hspace*{3mm}%
    \ifthenelse{\boolean{EstructuraHayItems}}{%
       ,#1%
    }{}%
    \emph{#2}: #3%
    \setboolean{EstructuraHayItems}{true}%
		\ignorespaces%
}

\newcommandx{\enumItem}[2][1={\ }]{%
    %\hspace*{3mm}%
    \ifthenelse{\boolean{EstructuraHayItems}}{%
        ,#1%
    }{}%
    \TipoVariable{#2}%
    \setboolean{EstructuraHayItems}{true}%
		\ignorespaces%
}

\newcommandx{\RepFc}[3][1={estr},2={e}]{%
  \tadOperacion{Rep}{#1}{bool}{}%
  \tadAxioma{Rep($#2$)}{#3}%
}%

\newcommandx{\Rep}[3][1={estr},2={e}]{%
  \tadOperacion{Rep}{#1}{bool}{}%
  \tadAxioma{Rep($#2$)}{true \ssi #3}%
}%

\newcommandx{\Abs}[5][1={estr},3={e}]{%
  \tadOperacion{Abs}{#1/#3}{#2}{Rep($#3$)}%
  \settominwidth{\hangindent}{Abs($#3$) \igobs #4: #2 $\mid$ }%
  \addtolength{\hangindent}{\parindent}%
  Abs($#3$) \igobs #4: #2 $\mid$ #5%
}%

\newcommandx{\AbsFc}[4][1={estr},3={e}]{%
  \tadOperacion{Abs}{#1/#3}{#2}{Rep($#3$)}%
  \tadAxioma{Abs($#3$)}{#4}%
}%

\newcommand{\DRef}{\ensuremath{\rightarrow}}

\newcommandx{\ComplejidadDer}[1]{
	\hfill \ensuremath{\mathcal{O}(#1)}
}

\newcommandx{\Complejidad}[1]{
	\ensuremath{\mathcal{O}(#1)}
}

\renewcommand{\thealgorithm}{}

\newenvironmentx{Algoritmo}[6]{
	%\begin{algorithm}
	
	\begin{mdframed}
		
		%\floatname{algorithm}{Algoritmo}
		%\caption{\TipoFuncion{#1}{#2}{#3}}
		\TipoFuncion{#1}{#2}{#3}
		\begin{algorithmic}[1]
			#4
		\end{algorithmic}

		\begin{flushleft}
			%\begin{framed}
				%\flushleft
				\textbf{Complejidad:} #5 \\
				\vspace{0.75em}
					#6
			%\end{framed}
		\end{flushleft}
		
	\end{mdframed}
	
	%\end{algorithm}
}

\fancypagestyle{caratula} {
   \fancyhf{}
   \cfoot{\thepage /\pageref{LastPage}}
   \renewcommand{\headrulewidth}{0pt}
   \renewcommand{\footrulewidth}{0pt}
}

\begin{document}

%%%%%%%%%%%%%%%%%%%%%%%%%%%%%%%%%%%%%%%%%%%%%%%%%%%%%%%%%%%%%%%%%%%%%%%%%%%%%%%
%% Car�tula                                                                  %%
%%%%%%%%%%%%%%%%%%%%%%%%%%%%%%%%%%%%%%%%%%%%%%%%%%%%%%%%%%%%%%%%%%%%%%%%%%%%%%%

\thispagestyle{caratula}

\begin{center}

\vspace*{1cm}

\begin{Huge}
Algoritmos y Estructuras de Datos II
\end{Huge}

\vspace{1cm}

\begin{LARGE}
Trabajo Pr�ctico 2
\end{LARGE}

\vspace{1cm}

Departamento de Computaci�n,\\
Facultad de Ciencias Exactas y Naturales,\\
Universidad de Buenos Aires

\vspace{1cm}

Primer Cuatrimestre de 2015

\vspace{1cm}

\begin{Large}
Grupo 16
\end{Large}

\vspace{0.5cm}

\begin{tabular}{|c|c|c|}
\hline
Apellido y Nombre & LU & E-mail\\
\hline
Fernando Frassia            	 & 340/13 & ferfrassia@gmail.com\\
Rodrigo Seoane Quilne            & 910/11 & seoane.raq@gmail.com\\
Sebastian Matias Giambastiani    & 916/12 & sebastian.giambastiani@hotmail.com\\
\hline
\end{tabular}

\vspace{1cm}

Reservado para la c�tedra

\begin{tabular}{|c|c|c|}
\hline
Instancia & Docente que corrigi� & Calificaci�n\\
\hline
Primera Entrega &&\\
\hline
Recuperatorio   &&\\
\hline
\end{tabular}

\end{center}

\vspace{6cm}

\newpage


\tableofcontents


%%%%%%%%%%%%%%%%%%%%%%%%%%%%%%%%%%%%%%%%%%%%%%%%%%%%%%%%%%%%%%%%%%%%%%%%%%%%%%%%%%%%%%%%%%%%%%%%%%%%%
%
%
%
% ACA EMPIEZA EL CODIGO DEL LATEX
%
%
%
%%%%%%%%%%%%%%%%%%%%%%%%%%%%%%%%%%%%%%%%%%%%%%%%%%%%%%%%%%%%%%%%%%%%%%%%%%%%%%%%%%%%%%%%%%%%%%%%%%%%%

\newpage
\section{Tad Extendidos}	

\subsection{Secu($\alpha$)}

\vspace{1em}

\tadOtrasOperaciones
\tadOperacion{elemDeSecu}{Secu($\alpha$)/s,Nat/n}{RUR}{n $<$ long($s$)}

\vspace{1em}

\tadAxiomas
\tadAxioma{elemDeSecu(s, n)}{\IF $n =$ 0 THEN prim($s$) ELSE elemDeSecu(fin(s), n-1) FI}

\vspace{1em}

\subsection{Mapa}

\vspace{1em}

\tadObservadores
\tadOperacion{restricciones}{Mapa/m}{secu(restriccion)}{}
\tadOperacion{nroConexion}{estacion/e_1, estacion/e_2, Mapa/m}{nat}{{$e_1, e_2$} $\subset$ estaciones(m) $\yluego$ conectadas?($e_1, e_2, m$)}

\vspace{1em}

\tadAxiomas
\tadAxioma{restricciones(vacio)}{$\langle$~$\rangle$}
\tadAxioma{restricciones(agregar($e, m$))}{restricciones($m$)}
\tadAxioma{restricciones(conectar($e_1, e_2, r, m$))}{restricciones($m$) $\circ$ $r$}
\tadAxioma{nroConexion($e_1, e_2, $conectar($e_3, e_4, m$))}{
	\IF ($(e_1 = e_3 \wedge e_2 = e_4) \vee (e_1 = e_4 \wedge e_2 = e_3)$) THEN long(restricciones($m$)) - 1 ELSE nroConexion($e_1, e_2, m$) - 1 FI
}
\tadAxioma{nroConexion($e_1, e_2, $agregar($e, m$))}{nroConexion($e_1, e_2, m$)}

\vspace{1em}

\subsection{Diccionario(Clave, Significado)}

\vspace{1em}

\tadOtrasOperaciones
\tadAlinearFunciones{claveMax}{Diccionario/d}
\tadOperacion{vac�o?}{Diccionario}{Bool}{}
\tadOperacion{claveMax}{Diccionario/d}{Clave}{$\neg$vac�o(d)}
\tadOperacion{secuClaves}{Diccionario}{Secu(clave)}{}

\vspace{1em}

\tadAxiomas
\tadAlinearAxiomas{secuClaves(definir(c, s, d))}
\tadAxioma{vac�o?(vac�o)}{true}
\tadAxioma{vac�o?(definir(c, s, d))}{false}
\tadAxioma{claveMax(d)}{elemMax(claves(d))}
\tadAxioma{secuClaves(vac�o)}{$\textless$$\textgreater$}
\tadAxioma{secuClaves(definir(c, s, d))}{secuClaves(d) $\circ$ c }

\vspace{1em}

\subsection{Conjunto($\alpha$$\textless$)}

\vspace{1em}

\tadOtrasOperaciones
\tadAlinearFunciones{auxMaxElem}{$\alpha$/e, Conj($\alpha$)/c}
\tadOperacion{elemMax}{Conj($\alpha$)/c}{$\alpha$}{$\neg$$\emptyset$?(c)}
\tadOperacion{auxElemMax}{$\alpha$, Conj($\alpha$)}{$\alpha$}{}

\vspace{1em}

\tadAxiomas
\tadAlinearAxiomas{auxMaxElem(e, c)}
\tadAxioma{elemMax(c)}{auxMaxElem(dameUno(c), c)}
\tadAxioma{auxElemMax(e, c)}{\IF $\emptyset?$(c) THEN e ELSE {\IF e $\textgreater$ dameUno(c) THEN auxElemMax(e, sinUno(c)) ELSE auxElemMax(dameUno(c), sinUno(c)) FI} FI}




\newpage
\section{Red}

\begin{Interfaz}
  
  %\textbf{par�metros formales}\hangindent=2\parindent\\
  %\parbox{1.7cm}{\textbf{g�neros}} \\
  %\parbox[t]{1.7cm}{\textbf{funci�n}}\parbox[t]{\textwidth-2\parindent-1.7cm}{%
    %\InterfazFuncion{Verifica?}{\In{c}{conj(tag)}, \In{r}{rest}}{$\alpha$}
    %{$res \igobs a$}
    %[$\Theta(copy(a))$]
    %[funci�n de copia de $\alpha$'s]
  %}

  \textbf{se explica con}: \tadNombre{Red, Iterador Unidireccional($\alpha$)}.

  \textbf{g�neros}: \TipoVariable{red, itConj(Compu)}.

  \Titulo{Operaciones b�sicas de Red}

  %\InterfazFuncion{NOMBRE}{INPUTS}{TIPO RES}%
  %[ACA VA EL PRE (SI LO HAY)]
  %{ACA VA EL POST}%
  %[$\Theta(COMPLEJIDAD)$]
  %[DESCRIPCION]

  \InterfazFuncion{Computadoras}{\In{r}{red}}{itConj(Compu)}
  %[ACA VA EL PRE (SI LO HUBIERA)]
  {$res \igobs$ crearIt(computadoras($r$))}
  [$\Complejidad{1}$]
  [Devuelve las computadoras de red.]
	
  \InterfazFuncion{Conectadas?}{\In{r}{red}, \In{c_1}{compu}, \In{c_2}{compu}}{bool}
  [\{$c_1,c_2$\} $\subseteq$ computadoras($r$)]
  {$res \igobs$ conectadas?($r, c_1, c_2$)}
  [$\Complejidad{|c_1| + |c_2|}$]
  [Devuelve el valor de verdad indicado por la conexi�n o desconexi�n de dos computadoras.]

  \InterfazFuncion{InterfazUsada}{\In{r}{red}, \In{c_1}{compu}, \In{c_2}{compu}}{interfaz}
  [\{$c_1, c_2$\} $\subseteq$ computadoras($r$) $\yluego$ conectadas?($r, c_1, c_2$)]
  {$res \igobs$ interfazUsada($r, c_1, c_2$)}
  [$\Complejidad{|c_1| + |c_2|}$]
  [Devuelve la interfaz que $c_1$ usa para conectarse con $c_2$]

  \InterfazFuncion{IniciarRed}{}{red}
  %[ACA VA EL PRE (SI LO HAY)]
  {$res \igobs$ iniciarRed()}
  [$\Complejidad{1}$]
  [Crea una red sin computadoras.]
	
  \InterfazFuncion{AgregarComputadora}{\Inout{r}{red}, \In{c}{compu}}{}
  [$r_0 \igobs r$ $\wedge$ $\neg$($c$ $\in$computadoras($r$))]
  {$r$ $\igobs$ agregarComputadora($r_0, c$)}
  [$\Complejidad{|c|}$]
  [Agrega una computadora a la red.]
  
   \InterfazFuncion{Conectar}{\Inout{r}{red}, \In{c_1}{compu}, \In{i_1}{interfaz}, \In{c_2}{compu}, \In{i_2}{interfaz}}{}
  [$r_0$ $\igobs$ r $\wedge$ \{$c_1,c_2$\} $\subseteq$ computadoras($r$) $\wedge$ ip($c_1$) $\neq$ ip($c_2$) $\yluego$ $\neg$ conectadas?($r, c_1, c_2$) $\wedge$ $\neg$ usaInterfaz?($r, c_1, i_1$) $\wedge$ $\neg$ usaInterfaz?($r, c_2, i_2$)]
  {$r$ $\igobs$ conectar($r, c_1, i_1, c_2, i_2$)}
  [$\Complejidad{|c_1| + |c_2|}$]
  [Conecta dos computadoras y les a�ade la interfaz correspondiente.]

  \InterfazFuncion{Vecinos}{\In{r}{red}, \In{c}{compu}}{itConj(compu)}
  [$c$ $\in$ computadoras($r$)]
  {$res$ $\igobs$ crearIt(vecinos(r, c))}
  []
  [Devuelve todas las computadoras que est�n conectadas directamente con c]

  \InterfazFuncion{UsaInterfaz?}{\In{r}{red}, \In{c}{compu}, \In{i}{interfaz}}{bool}
  [$c$ $\in$ computadoras($r$)]
  {$res$ $\igobs$ usaInterfaz?(r, c, i)}
  []
  [Verifica que una computadora use una interfaz]

  \InterfazFuncion{CaminosMinimos}{\In{r}{red}, \In{c_1}{compu}, \In{c_2}{compu}}{conj(lista(compu))}
  [\{$c_1,c_2$\} $\subseteq$ computadoras($r$)]
  {$res$ $\igobs$ caminosMinimos(r, $c_1$, $c_2$)}
  []
  [Devuelve todos los caminos minimos de conexiones entre una computadora y otra]

  \InterfazFuncion{HayCamino?}{\In{r}{red}, \In{c_1}{compu}, \In{c_2}{compu}}{bool}
  [\{$c_1,c_2$\} $\subseteq$ computadoras($r$)]
  {$res$ $\igobs$ hayCamino?(r, $c_1$, $c_2$)}
  []
  [Verifica que haya un camino de conexiones entre una computadora y otra]
  
\end{Interfaz}

\subsection{Auxiliares}

\begin{Auxiliares}

  \Titulo{Operaciones auxiliares}

  \InterfazFuncion{CaminosMinimos}{\In{r}{red}, \In{c_1}{compu}, \In{c_2}{compu}}{conj(lista(compu))}
  [\{$c_1,c_2$\} $\subseteq$ computadoras($r$)]
  {$res$ $\igobs$ caminosMinimos(r, $c_1$, $c_2$)}
  []
  [Devuelve los caminos m�nimos entre c_1 y c_2]

\end{Auxiliares}

\subsection{Representacion}

\begin{Representacion}

%ACA VA LA DESCRIPCION DE LA CACONA DE LA RED

\bigskip
\begin{Estructura}{red}[e\_red]
	\begin{Tupla}[e\_red]
		\tupTupItem{vecinosEInterfaces}{\TipoVariable{diccString}$($\ignorespaces
			% NOTA:tupTupItem es un engendro que agregue para este caso, NO USAR EN OTRO LADO, USAR tupItem EN SU LUGAR.
			\emph{compu}: \TipoVariable{string},
			tupla(
			\emph{interfaces}: \TipoVariable{diccString}$($\ignorespaces
				\emph{compu}: \TipoVariable{string}, \ignorespaces
				\emph{interfaz}: \TipoVariable{nat}\ignorespaces
			$)$,
			\emph{compusVecinas}: \TipoVariable{conj(compu)}
			$) )$} \\
			\tupTupItem{deOrigenADestino}{\TipoVariable{diccString}$($\ignorespaces
			\emph{compu}: \TipoVariable{string},
			\TipoVariable{diccString}$($\ignorespaces
				\emph{compu}: \TipoVariable{string}, \ignorespaces
				\emph{caminosMinimos}: \TipoVariable{conj(lista(compu))}\ignorespaces
			$) )$} \\
			\tupItem{computadoras}{\TipoVariable{conj(compu)}}
	\end{Tupla}
	
\end{Estructura}

\subsection{InvRep y Abs}

\begin{enumerate}
	\item{El conjunto de claves de ''uniones'' es igual al conjunto de estaciones ''estaciones''.}
	\item{''\#sendas'' es igual a la mitad de las horas de ''uniones''.}
	\item{Todo valor que se obtiene de buscar el significado del significado de cada clave de ''uniones'', es igual el valor hallado tras buscar en ''uniones'' con el sinificado de la clave como clave y la clave como significado de esta nueva clave, y no hay otras hojas ademas de estas dos, con el mismo valor.}
	\item{Todas las hojas de ''uniones'' son mayores o iguales a cero y menores a ''\#sendas''.}
	\item{La longitud de ''sendas'' es mayor o igual a ''\#sendas''.}
\end{enumerate}

\Rep[e\_mapa][m]{
	\\m.estaciones = claves(m.uniones) $\wedge$ \hfill 1.
	\\m.\#sendas = \#sendasPorDos(m.estaciones, m.uniones) / 2 $\wedge$ m.\#sendas $\leq$ long(m.sendas) $\yluego$ \hfill 2. 5.
	\\($\forall$ e1, e2: string)(e1 $\in$ claves(m.uniones) $\yluego$ e2 
	$\in$ claves(obtener(e1, m.uniones)) $\impluego$\\ 
	e2 $\in$ claves(m.uniones) $\yluego$ e1 $\in$ claves(obtener(e2, m.uniones)) $\yluego$ 
	\\ obtener(e2, obtener(e1, m.uniones)) = obtener(e1, obtener(e2, m.uniones)) $\wedge$ \hfill 3. 4.	
	\\ obtener(e2, obtener(e1, m.uniones)) $<$ m.\#sendas) $\wedge$
	\\($\forall$ e1, e2, e3, e4: string)((e1 $\in$ claves(m.uniones) $\yluego$
	e2 $\in$ claves(obtener(e1, m.uniones)) $\wedge$\\ 
	e3 $\in$ claves(m.uniones) $\yluego$ e4 $\in$ claves(obtener(e3, m.uniones))) $\impluego$
	\\ (obtener(e2, obtener(e1, m.uniones)) $=$ obtener(e4, obtener(e3, m.uniones)) $\ssi$
	\\ (e1 = e3 $\wedge$ e2 = e4) $\vee$ (e1 = e4 $\wedge$ e2 = e3)))) \hfill 3.
}

\vspace{2em}
	
\tadOperacion{\#sendasPorDos}{conj($\alpha$)\ c, dicc($\alpha$, dicc($\alpha$, $\beta$))\ d}
														{nat}{c $\subset$ claves(d)}

\vspace{1em}

\tadAxioma{\#sendasPorDos(c, d)}{\IF $\emptyset$?(c) THEN 0
												ELSE \#claves(obtener(dameUno(c),d)) $+$ \#sendasPorDos(sinUno(c), d)
												FI}

\vspace{2em}

\Abs[e\_mapa]{mapa}[m]{p}{
	\\ m.estaciones = estaciones(p) $\yluego$
	\\ ($\forall$ e1, e2: string)((e1 $\in$ estaciones(p) $\wedge$ e2 $\in$ estaciones(p)) $\impluego$
	\\ (conectadas?(e1, e2, p) $\ssi$
	\\   e1 $\in$ claves(m.uniones) $\wedge$ e2 $\in$ claves(obtener(e2, m.uniones)))) $\yluego$
	\\ ($\forall$ e1, e2: string)((e1 $\in$ estaciones(p) $\wedge$ e2 $\in$ estaciones(p)) $\yluego$
	\\ conectadas?(e1, e2, p) $\impluego$ 
	\\ (restriccion(e1, e2, p) = m.sendas[obtener(e2, obtener(e1, m.uniones))] $\wedge$
	   nroConexion(e1, e2, m) = obtener(e2, obtener(e1, m.uniones))) $\wedge$
	   long(restricciones(p)) = m.\#sendas $\yluego$
		 ($\forall$ n:nat) (n < m.\#sendas $\impluego$ m.sendas[n] = ElemDeSecu(restricciones(p), n)))
}

\end{Representacion}

\subsection{Algoritmos}

\begin{Algoritmos}
	
	%Algoritmos / Inputs / TSalida / Codigo / Complejidad Final / Justificacion
	\begin{Algoritmo}{iComputadoras}{\In{r}{red}}{itConj(Compu)}
	{
		\State $res$ $\gets$ CrearIt($r.computadoras$) \ComplejidadDer{1}
	} 
	{$\Complejidad{1}$}{}
	\end{Algoritmo}
	
	\begin{Algoritmo}{iConectadas?}{\In{r}{red}, \In{c_1}{compu}, \In{c_2}{compu}}{bool}
	{
		\State $res$ $\gets$ Definido?(Significado($r.vecinosEInterfaces$, $c_1.ip$)$.interfaces$, $c_2.ip$) \ComplejidadDer{|c_1| + |c_2|}
	}
	{$\Complejidad{|c_1| + |c_2|}$}{}
	\end{Algoritmo}
	
	\begin{Algoritmo}{iInterfazUsada}{\In{r}{red}, \In{c_1}{compu}, \In{c_2}{compu}}{interfaz}
	{
		\State $res$ $\gets$ Significado(Significado($r.vecinosEInterfaces$, $c_1.ip$)$.interfaces$, $c_2.ip$)\ComplejidadDer{|c_1| + |c_2|}
	}
	{$\Complejidad{|c_1| + |c_2|}$}{}
	\end{Algoritmo}
	
	\begin{Algoritmo}{iIniciarRed}{}{red}
	{
		\State $res$ $\gets$ tupla($vecinosEInterfaces$: Vac�o(), $deOrigenADestino$: Vac�o(), $computadoras$: Vac�o()) \ComplejidadDer{1 + 1 + 1}
	}
	{$\Complejidad{1}$}
	{$\Complejidad{1}  + \Complejidad{1} + \Complejidad{1} = \newline
	  3 * \Complejidad{1} = \Complejidad{1}$}
	\end{Algoritmo}
	
	\begin{Algoritmo}{iAgregarComputadora}{\Inout{r}{red}, \In{c}{compu}}{}
	{
		\State Agregar($r.computadoras$, $c$) \ComplejidadDer{1}
		\State Definir($r.vecinosEInterfaces$, $c.ip$, tupla(Vac�o(), Vacio())) \ComplejidadDer{|c|}
		\State Definir($r.deOrigenADestino$, $c.ip$, Vac�o()) \ComplejidadDer{|c|}
	}
	{$\Complejidad{|c|}$}
	{$\Complejidad{1} + \Complejidad{|c|} + \Complejidad{|c|} = \newline
	2 * \Complejidad{|c|} = \Complejidad{|c|}$}
	\end{Algoritmo}
	
	\begin{Algoritmo}{iConectar}{\Inout{r}{red}, \In{c_1}{compu}, \In{i_1}{interfaz}, \In{c_2}{compu}, \In{i_2}{interfaz}}{}
	{
		%Armo la primera componente de e_red
		
		\State var $tupSig1$:tupla $\gets$ Significado($r.vecinosEInterfaces$, $c_1.ip$)
		%Guardo la interfaz usada por c_1
		\State Definir($tupSig1.interfaces$, $c_2.ip$, $i_1$) \ComplejidadDer{|c_1| + |c_2| + 1}
		%Agrego c_2 a las compusVecinas de c_1
		\State Agregar($tupSig1.compusVecinas$, $c_2$) \ComplejidadDer{1}

		\State var $tupSig2$:tupla $\gets$ Significado($r.vecinosEInterfaces$, $c_2.ip$)
		%Guardo la interfaz usada por c_2		
		\State Definir($tupSig2.interfaces$, $c_1.ip$, $i_2$) \ComplejidadDer{|c_1| + |c_2| + 1}
		%Agrego c_2 a las compusVecinas de c_1
		\State Agregar($tupSig2.compusVecinas$, $c_1$) \ComplejidadDer{1}
		

		%Armo la segunda componente de e_red
				
		%Guardo los caminos m�nimos de c_1 a c_2
		\State Definir(Significado($r.deOrigenADestino$, $c_1.ip$), $c_2.ip$, CaminosMinimos(r, c_1, c_2)) \ComplejidadDer{1}
		%Guardo los caminos m�nimos de c_2 a c_1
		\State Definir(Significado($r.deOrigenADestino$, $c_2.ip$), $c_1.ip$, CaminosMinimos(r, c_2, c_1)) \ComplejidadDer{1}		
	}
	{$\Complejidad{|e_1| + |e_2|}$}
	{$\Complejidad{|e_1| + |e_2|} + \Complejidad{|e_1| + |e_2|} + \Complejidad{1} + \Complejidad{1} = \newline
	  2 * \Complejidad{1} + 2 * \Complejidad{|e_1| + |e_2|} = \newline
		2 * \Complejidad{|e_1| + |e_2|} = \Complejidad{|e_1| + |e_2|}$}
	\end{Algoritmo}
	
	\begin{Algoritmo}{iVecinos}{\In{r}{red}, \In{c}{compu}}{itConj(compu)}
	{
		\State $res$ $\gets$ crearIt((Significado($r.vecinosEInterfaces$, $c.ip$))$.compusVecinas$)
	}
	{}
	{}
	\end{Algoritmo}
	
	\begin{Algoritmo}{iUsaInterfaz}{\In{r}{red}, \In{c}{compu}, \In{i}{interfaz}}{bool}
	{
		\State var $tupVecinos$:tupla $\gets$ Significado($r.vecinosEInterfaces$, $c.ip$) \ComplejidadDer{1}
		\State var $itCompusVecinas$:itConj(compu) $\gets$ CrearIt($tupVecinos.compusVecinas$) \ComplejidadDer{1}
		\State $res$:bool $\gets$ $false$ \ComplejidadDer{1}
		\While{HaySiguiente($itCompusVecinas$) AND $�res$} \ComplejidadDer{1}
		\If {Significado($tupVecinos.interfaces$, Siguiente($itCompusVecinas$)$.ip$) $==$ $i$} \ComplejidadDer{1}
			\State $res$ $\gets$ $true$  \ComplejidadDer{1}
		\EndIf
		\State Avanzar($it$) \ComplejidadDer{1}
		\EndWhile
	}
	{}
	{}
	\end{Algoritmo}
	
	\begin{Algoritmo}{iCaminosMinimos}{\In{r}{red}, \In{c_1}{compu}, \In{c_2}{compu}}{conj(lista(compu))}
	{
		\State $res$ $\gets$ Significado(Significado($r.deOrigenADestino$, $c_1.ip$), $c_2.ip$)	\ComplejidadDer{|c_1| + |c_2|}
	}
	{}
	{}
	\end{Algoritmo}

	\begin{Algoritmo}{iHayCamino}{\In{r}{red}, \In{c_1}{compu}, \In{c_2}{compu}}{bool}
	{
		\State var $conjCaminosMinimos$ $\gets$ CaminosMinimos($r$, $c_1$, $c_2$) \ComplejidadDer{1}
		\State $res$ $\gets$ EsVacio?($conjCaminosMinimos$) \ComplejidadDer{1}
	}
	{}
	{}
	\end{Algoritmo}
	
	\begin{Algoritmo}{iCaminosMinimos}{\In{r}{red}, \In{c_1}{compu}, \In{c_2}{compu}}{bool}
	{
		%\State var $conjExcluidos$ $\gets$ Vacio()
		%\State var $camino$
	}
	{}
	{}
	\end{Algoritmo}
		
	%\begin{Algoritmo}{iNroConexion}{\In{e1}{estaci�n}, \In{e2}{estaci�n}, \In{m}{mapa}}{nat}
	%{
	%	\State $res$ $\gets$ Significado(Significado($m.uniones$, $e1$), $e2$) \ComplejidadDer{|e_1| + |e_2|}
	%}
	%{$\Complejidad{|e_1| + |e_2|}$}{}
	%\end{Algoritmo}
	%
	%\begin{Algoritmo}{iEvaluarSendas}{\In{c}{conj(tag)}, \In{m}{mapa}}{arreglo\_dimesionable de bool}
	%{
	%	\State $res$ $\gets$ arreglo[$m.\#sendas$] de bool \ComplejidadDer{1}
	%	\State var $i$: nat $\gets$ 0 \ComplejidadDer{1}
	%	\While{$i < m.\#sendas$} \ComplejidadDer{1}
	%		\State $res[i]$ $\gets$ Verifica?($c$, $m.sendas[i]$) \ComplejidadDer{R}
	%		\State $i++$ \ComplejidadDer{1}
	%	\EndWhile
	%}
	%{$\Complejidad{S * R}$}
	%{$\Complejidad{1} + \Complejidad{1} + \sum_{i = 1}^{S} (\Complejidad{R} + \Complejidad{1}) = \newline
	%	2 * \Complejidad{1} + S * (\Complejidad{R} + \Complejidad{1}) = \newline
	%	2 * \Complejidad{1} + S * \Complejidad{1} + S * \Complejidad{R} = \newline
	%	\Complejidad{S} + S * \Complejidad{R} = \newline
	%	\Complejidad{S + S * R} = \Complejidad{S * R}
	%$\newline
	%$S$ es |m.sendas| y $R$ es la longitud de la restriccion mas grande}
	%\end{Algoritmo}
	
	%\begin{Algoritmo}{iRestricciones}{\In{m}{mapa}}{arreglo\_dimesionable de restriccion}
	%{
	%	\State $res$ $\gets$ arreglo\_dimensionable[$m.\#sendas$] \ComplejidadDer{1}
	%	\State var $i$: nat $\gets$ 0 \ComplejidadDer{1}
	%	\While{$i < m.\#sendas$} \ComplejidadDer{1}
	%		\State $res[i]$ $\gets$ $m.sendas[i]$ \ComplejidadDer{1}
	%		\State $i++$ \ComplejidadDer{1}
	%	\EndWhile
	%}
	%{$\Complejidad{S}$}
	%{$\Complejidad{1} + \Complejidad{1} + \sum_{i = 1}^{S} (\Complejidad{1} + \Complejidad{1}) = \newline
	%	2 * \Complejidad{1} + S * 2 *\Complejidad{1} = \newline
	%	2 * \Complejidad{1} + 2 * \Complejidad{S} = \newline
	%	2 * \Complejidad{S} = \Complejidad{S}
	% $}
	%\end{Algoritmo}
	%
\end{Algoritmos}

\newpage
\section{DCNet}

\subsection{Interfaz}

\begin{Interfaz}
  
  %\textbf{par�metros formales}\hangindent=2\parindent\\
  %\parbox{1.7cm}{\textbf{g�neros}} \\
  %\parbox[t]{1.7cm}{\textbf{funci�n}}\parbox[t]{\textwidth-2\parindent-1.7cm}{%
    %\InterfazFuncion{Verifica?}{\In{c}{conj(tag)}, \In{r}{rest}}{$\alpha$}
    %{$res \igobs a$}
    %[$\Theta(copy(a))$]
    %[funci�n de copia de $\alpha$'s]
  %}

  \textbf{se explica con}: \tadNombre{DCNet}.

  \textbf{g�neros}: \TipoVariable{dcnet}.

  \Titulo{Operaciones b�sicas de DCNet}
	

  \InterfazFuncion{Red}{\In {d}{dcnet}}{red}
  %[ACA VA EL PRE (SI LO HUBIERA)]
  {$res \igobs$ red($d$)}
  [\Complejidad{1}]
  [Devuelve la red del dcnet.]
  
  \InterfazFuncion{CaminoRecorrido}{\In {d}{dcnet}, \In{p}{paquete} }{secu(compu)}
  [$p$ $\in$ paqueteEnTransito?($d,p$)]
  {$res \igobs$ caminoRecorrido($d,p$))}
  [\Complejidad{n*log_2(k)}, donde n = $\#$computadoras(red(d)), y k la cantidad maxima de paquetes que hay en una computadora.]
  [Devuelve una secuencia con las computadoras por las que paso el paquete.]
  
  \InterfazFuncion{cantidadEnviados}{\In {d}{dcnet}, \In {c}{compu}}{nat}
  [$c$ $\in$ computadoras(red($d$))]
  {$res \igobs$ cantidadEnviados($d,c$)}
  [\Complejidad{|$c$.ip|}]
  [Devuelve la cantidad de paquetes que fueron enviados desde la computadora.]
  
  \InterfazFuncion{enEspera}{\In {d}{dcnet}, \In {c}{compu}}{itPaquete}
  [$c$ $\in$ computadoras(red($d$))]
  {$res \igobs$ enEspera($d,c$)}
  [\Complejidad{|$c$.ip|}]
  [Devuelve los paquetes que se encuentran en ese momento en la computadora.]
  
  \InterfazFuncion{IniciarDCNet}{\In {r}{red}}{dcnet}
  %[ACA VA EL PRE (SI LO HUBIERA)]
  {$res \igobs$ iniciarDCNet($r$)}
  [\Complejidad{n*L}, donde n = $\#$computadoras(red(d)), y L es el hostname mas largo de las computadoras.]
  [Inicia un dcnet con la red y sin paquetes.]

  \InterfazFuncion{CrearPaquete}{\In {p}{paquete}, \Inout {d}{dcnet}}{}
  [$d_0 \equiv d$ $\wedge$ $\neg$ (($\exists$ $p_1$: paquete)(paqueteEnTransito($s,p_1$) $\wedge$ id($p_1$) $=$ id($p$)) $\wedge$ origen($p$) $\in$ computadoras(red($d$))$\yluego$ destino($p$) $\in$ computadoras(red($d$))$\yluego$ hayCamino?(red($d$,origen($p$),destino($p$))  ]
  {$d \igobs$ crearPaquete($d_0$)}
  [\Complejidad{L+log_2(k)}, donde L es el hostname mas largo de las computadoras y k la cantidad maxima de paquetes que hay en una computadora.]
  [Agrega el paquete al dcnet.]
  
  \InterfazFuncion{AvanzarSegundo}{\Inout {d}{dcnet}}{}
  [$d_0 \equiv d$ ]
  {$d$ $\igobs$ avanzarSegundo($d_0$)}
  [\Complejidad{n*(L+log_2(k)}, donde n = $\#$computadoras(red(d)) y  y k la cantidad maxima de paquetes que hay en una computadora. ]
  [El paquete de mayor prioridad de cada computadora avanza a su proxima computadora siendo esta la del camino mas corto.]
  
  
  \InterfazFuncion{PaqueteEnTransito?}{\In {d}{dcnet}, \In {p}{paquete}}{bool}
  %[ACA VA EL PRE (SI LO HUBIERA)]
  {$res$ $\igobs$ paqueteEnTransito?($d$,$p$)}
  [\Complejidad{n*log_2(k)}, donde n = $\#$computadoras(red(d)) y  y k la cantidad maxima de paquetes que hay en una computadora.]
  [Devuelve si el paquete esta o no en alguna computadora del sistema.]
  
  
  \InterfazFuncion{LaQueMasEnvio}{\In {d}{dcnet)}}{compu}
  %[ACA VA EL PRE (SI LO HUBIERA)]
  {$res$ $\igobs$ laQueMasEnvio($d$)}
  [\Complejidad{1}]
  [Devuelve la computadora que mas paquetes envio.]
	
	\Titulo{Operaciones del iterador}
  
\end{Interfaz}

\subsection{Representacion}

\begin{Representacion}

%DESCRIPCION
El dcnet esta representado con una tupla en la cual: vamos a tener  la red (en d.red) con la que se inicia el dcnet(pasada por referencia), una tupla que tiene la computadora que mas paquetes envio y cuantos ha enviado (en d.MasEnviante). Luego, un diccionario String (d.CompYPaq) cuyas claves son las computadoras de la red  y su significado es una tupla. Esta tupla tiene tres componentes: un Diccionario Rapido (MasPriori) cuya clave es un natural (prioridad, osea la prioridad del paquete) y su significado es un conj(paquete) (PaqdePriori), luego otro Diccionario Rapido (PaqYCam) cuya clave es el paquete (habiendo establecido previamente que la relacion de orden entre paquetes es por id) y su significado es una secu(compu) (CamRecorrido), finalmente un natural (Enviados) que representa la cantidad de paquetes enviados por la computadora.



\bigskip
\begin{Estructura}{dcnet}[e\_dc]
	\begin{Tupla}[e\_dc]
		\tupItem{red}{red}
		\tupTupItem{\\MasEnviante}{\TipoVariable{tupla}$($\ignorespaces
			\emph{compu}: \TipoVariable{compu}, 
			\emph{enviados}: \TipoVariable{nat}$)$}
		\tupTupItem{\\CompYPaq}{\TipoVariable{DiccString}$($\ignorespaces
			\emph{ip}: \TipoVariable{string},
			\TipoVariable{tupla}$($\ignorespaces
			\emph{MasPriori}:{DiccRapido}$($\ignorespaces
			\emph{prioridad} :{nat},
			\emph{PaqdePriori}:{conj$($paquete$)$}
			 $)$,
			\emph{PaqYCam}:{DiccRapido} $($\ignorespaces
			\emph{paq}:{paquete},
			\emph{CamRecorrido}:{secu($compu$)}$)$, 
			\emph{Enviados}:{nat})
		
			}
			
	
	\end{Tupla}
\end{Estructura}

\subsection{InvRep y Abs}

\begin{enumerate}
 \item{El conjunto armado por las ips de las computadoras de 'red' es igual al conjunto con todas las claves de 'CompYPaq'.}
 \item{Para toda compu (c1) perteneciente a las claves de 'CompYPaq' cuyo significado.enviados es (e1), d.MasEnviante.enviados (e2) es mayor o igual a e1. Entonces, d.MasEnviante.compu esta definida en 'CompYPaq' y su significado es e2. }
 \item{Si una computadora pertenece a un significado de 'PaqYCam', entonces pertenece a claves de 'CompYPaq'}
 \item{La union de todos los significados de MasPriori es igual a las claves de 'PaqYCam'}
 \item{Si un paquete (paq) pertenece al significado de una prioridad (pri) en el diccionario MasPriori, entonces la prioridad de paq es pri}
 \item{No existen claves en el diccionario 'MasPriori' con significado vacio.}
 \item{Si un paquete (p1) esta definido en 'PaqYCam' como significado de una computadora (c1) en el diccionario 'CompYPaq', entonces p1 no puede estar definido en 'PaqYCam' como significado de una computadora (c2), siendo c1 y c2 distintas.}
 \item{Si un paquete (p1) esta definido en 'PaqYCam' como significado de una computadora(c1) en el diccionario 'CompYPaq', entonces la ultima componente de su significado (osea, 'CamRecorrido') es c1}


\end{enumerate}


\Rep[e\_dc][d]{claves(d.CompYPaq) $=$ ipCompus(computadoras(d.red))$\wedge$ \hfill1 \newline
	($\forall$ c:compu)(c.ip $\in$ claves(d.CompYPaq))\newline
	(obtener(c.ip,d.CompYPaq).enviados $\leq$ (d.MasEnviante).enviados $\wedge$ def?(d.MasEnviante).compu, d.CompYPaq) $\yluego$ \newline obtener(d.MasEnviante.compu, d.CompYPaq).enviados = (d.MasEnviante).enviados)\hfill2 \newline
	($\forall$ $c_2$:compu, p:paquete)(esta?($c_2$,obtener(p,obtener(c.ip, d.CompYPaq).PaqYCam)) $\Rightarrow$  def?($c_2$, d.CompYPaq) \hfill3 \newline
	$\wedge$ juntarSignificados(obtener(c.ip,d.CompYPaq).MasPriori,claves(obtener(c.ip,d.CompYPaq).MasPriori)
 $=$ claves(obtener(c.ip,d.CompYPaq).PaqYCam)	\hfill4 \newline
 ($\forall$ pr:Nat,p:paquete)(def?(pr,Obtener(c.ip,d.CompYPaq).MasPriori) $\yluego$ \newline 
 p $\in$ obtener(pr,c.ip,d.CompYPaq).MasPriori)$\rightarrow$ pr = p.prioridad) $\wedge$ \hfill5 \newline
 ($\forall$ pr:Nat) (Def?(pr, Obtener(c.ip, d.CompYPaq).MasPriori) $\impluego$  $\neg$ $\emptyset$? Obtener(pr, Obtener(c.ip, d.CompYPaq).MasPriori)) $\wedge$ \hfill6 \newline
 ($\forall$ p1:paquete, c2:computadora) (Def?(p, Obtener(c.ip, d.CompYPaq).PaqYCam) $\wedge$ def?(c2.ip, d.CompYPaq) $\impluego$ $\neg$def?(p, Obtener(c2.ip, d.CompYPaq).PaqYCam)  $\wedge$ \hfill7 \newline
 ($\forall$ p:paquete) (def?(p, obtener(c.ip, d.CompYPaq).PaqYCam) $\impluego$ ult(obtener(p, obtener(c.ip, d.CompYPaq).PaqYCam)) = c)	\hfill8
	 }
	
	\vspace{3em}
	
	\tadOperacion{ipCompus}{Conj(compu)}{Conj(string)}{}
	\tadOperacion{juntarSignificados}{Dicc , Conj}{Conj}{}
	
	\vspace{1em}
	\tadAxioma{juntarSignificados(dic, cl)}{
	\IF vacia?(cl) 
	THEN $\emptyset$ 
	ELSE 
	      obtener(DameUno(cl),dic) $\cup$ juntarSignificados(dic, SinUno(cl))
	FI}	
	\tadAxioma{ipCompus(cc)}{
	\IF vacia?(cp) 
	THEN $\emptyset$ 
	ELSE 
	      {Agregar(DameUno(cc).ip,ipCompus(SinUno(cc)))}
	FI}
	
	
	\vspace{3em}
	
	\Abs[e\_dc]{dcnet}[e]{d}{
		\\ red(d) = e.red $\yluego$
		\\($\forall$ p:paquete, c:compu) (paqueteEnTransito?(d, p) $\impluego$ (def?(c.ip, e.CompYPaq $\yluego$ def?(p, obtener(c.ip, e.CompYPaq).PaqYCam) $\impluego$ obtener(p, obtener(c.ip, e.CompYPaq).PaqYCam) = caminoRecorrido(d, p)))
		\\($\forall$ c:compu) (c $\in$ computadoras(red(d)) $\impluego$ ((def?(c.ip, e.CompYPaq) $\impluego$ obtener(c.ip, e.CompYPaq).Enviados = cantidadEnviados(d, c)) $\wedge$ (def?(c.ip, e.CompYPaq) $\impluego$ claves(obtener(c.ip, e.CompYPaq).PaqYCam) = enEspera(d, c))))
		
		 }
		
\vspace{3em}

\end{Representacion}

%tupla<mapa:mapa, robEnEst: dicctrie(estacion, conj(puntero(RUR))), robots: conj(RUR)>

\subsection{Algoritmos}

\begin{Algoritmos}




	\begin{Algoritmo}{iRed}{\In {d}{dcnet}}{red}
	{
		\State $res$ $\gets$ ($d$.red) \ComplejidadDer{1}
	}
	{\Complejidad{1}}
	{}
	
	\end{Algoritmo}
	
	
	
	
	\begin{Algoritmo}{iCaminoRecorrido}{\In {d}{dcnet},\In {p}{paquete}}{secu(compu)}
	{
		\State var $it$ $\gets$ \NombreFuncion {creaIt}(d.CompYPaq) \ComplejidadDer{n}
		\State var $esta$: bool $\gets$ false \ComplejidadDer{1}
		\While {\NombreFuncion {HaySiguiente}($it$) $\wedge$ $\neg$ esta}   \ComplejidadDer{n}
			\State var $diccpaq$:diccRapido  $\gets$  ((\NombreFuncion{Siguiente}($it$)).significado).PaqYCam	\ComplejidadDer{1}
			\If {\NombreFuncion {def?}($p$,$diccpaq$)}	\ComplejidadDer{log_2(k)}
				\State $esta$ $\gets$ true \ComplejidadDer{1}
				\State $res$ $\gets$ \NombreFuncion {obtener}($p$, $diccpaq$)	\ComplejidadDer{log_2(k)} 		
			\EndIf
			\State \NombreFuncion {Avanzar}($it$)  \ComplejidadDer{1}
		\EndWhile
	}
	{\Complejidad{n* log_2(k) }Donde n es la cantidad de computadoras en la red, y k la cantidad maxima de paquetes que hay en una computadora.}
	{$\Complejidad{1} + \Complejidad{1} + n* (\Complejidad{1} + \Complejidad{log_2(k)} + \Complejidad{1} + \Complejidad{log_2(k)} =$\\
	$\Complejidad{2} + n*(2 \Complejidad{ log_2(k)}) =$ \\
	 $\Complejidad{n* (log_2(k)) }$
	}
	\end{Algoritmo}
	
	
	
	
	
	\begin{Algoritmo}{iCantidadEnviados}{\In {d}{dcnet}, \In {c}{compu}}{nat}
	{
		\State $res$ $\gets$ \NombreFuncion {obtener}($c$.ip,$d$.CompYPaq).Enviados  \ComplejidadDer{L}
	}
	{\Complejidad{L} Siendo L la longitud de el IP de $c$}
	{}
	\end{Algoritmo}
	
	
	
	
	\begin{Algoritmo}{iEnEspera}{\In {d}{dcnet}, \In {c}{compu}}{itPaquete)}
	{
		\State $res$  $\gets$ \NombreFuncion {claves}(\NombreFuncion {obtener}($c$.ip, $d$.CompYPaq).PaqYCam) \ComplejidadDer{L}
	}
	{\Complejidad{L} Siendo L la longitud  del IP de $c$}
	{}
	\end{Algoritmo}
	
	
	
	
	
	
	\begin{Algoritmo}{iIniciarDCNet}{\In {r}{red}, \Inout {d}{dcnet}}{}
	{
		\State $d$.red $\gets$ r \ComplejidadDer{NOSE}
		\State  var $it$ $\gets$ \NombreFuncion {computadoras}(red) \ComplejidadDer{1}
		\State $d$.MasEnviante $\gets$ tupla(\NombreFuncion {Siguiente}($it$),0) \ComplejidadDer{1}
		\State $d$.CompyPaq $\gets$ Vacio() \ComplejidadDer{1}
		\While {\NombreFuncion{HaySiguiente}($it$)}  \ComplejidadDer{n}
			\State \NombreFuncion {Definir}(\NombreFuncion {Siguiente}($it$).ip, tupla(\NombreFuncion {Vacio}(),\NombreFuncion {Vacio}(), 0), $d$.CompyPaq) \ComplejidadDer{L + 1 + 1}
			\State \NombreFuncion {Avanzar}($it$) \ComplejidadDer{1}
		\EndWhile
		
	}
	{\Complejidad{n*L} Siendo n la cantidad de computadoras en la red y L el IP mas largo de ellas.}
	{$\Complejidad{1} + \Complejidad{1}+ \Complejidad{1} + \Complejidad{1} + n * \Complejidad{L +1 +1} + \Complejidad{1} =$ \\
	$\Complejidad{n*L}$ }
	\end{Algoritmo}
	
	
	
	
	
	
	\begin{Algoritmo}{iCrearPaquete}{\In {p}{paquete}, \Inout {d}{dcnet}}{}
	{
	
		\State var $diccprio$: diccRapido $\gets$ \NombreFuncion {obtener}($p$.origen, $d$.CompYPaq).MasPriori)	\ComplejidadDer{L}
		\State var $dicccam$: diccRapido $\gets$ \NombreFuncion {obtener}($p$.origen, $d$.CompYPaq).PaqYCam) \ComplejidadDer{L}
		\If {$\neg$ \NombreFuncion {def?}(p.prioridad, $diccprio$)}	\ComplejidadDer{log_2(k)}
			\State var cj:conjunto $\gets$ \NombreFuncion {Vacio}()\ComplejidadDer{1}
			\State \NombreFuncion {Agregar}(cj, $p$)\ComplejidadDer{1}
			\State \NombreFuncion {Definir}($p$.prioridad,cj, $diccprio$)	\ComplejidadDer{log_2(k)}
		\Else
			\State var cj:conjunto $\gets$ \NombreFuncion {obtener}($p$.prioridad, $diccprio$)\ComplejidadDer{log_2(k)}
			\State \NombreFuncion{Agregar}(cj, $p$) \ComplejidadDer{1}
			\State \NombreFuncion {Definir}($p$.prioridad,cj, $diccprio$)	\ComplejidadDer{log_2(k)}
		\EndIf
		
		\State var l:lista $\gets$ \NombreFuncion {Vacia}()\ComplejidadDer{1}
		\State \NombreFuncion {AgregarAtras}(l,p.origen)\ComplejidadDer{1}
		\State \NombreFuncion{definir}($p$, l, $dicccam$)	\ComplejidadDer{log_2(k)}
		
		
		}
	{\Complejidad{L+ log_2(k)} Donde L es la longitud de la computadora de origen del paquete, y k la cantidad de paquetes que EnEspera en esa computadora.}
	{$ \Complejidad{L} + \Complejidad{L} + \Complejidad{log_2(k)} + \Complejidad{log_2(k)} + \Complejidad{log_2(k)} + \Complejidad{log_2(k)} =$ \\
	$2*\Complejidad{L} + 4*\Complejidad{log_2(k)} =$\\
	$\Complejidad{L + log_2(k) }$}
	\end{Algoritmo}
	
	
	
	\begin{Algoritmo}{iAvanzarSegundo}{\Inout {d}{dcnet}}{}
	{
		\State var $it$ $\gets$ \NombreFuncion {computadoras}(red) \ComplejidadDer{1}
		\State var aux $\gets$ \NombreFuncion {Vacia}()	\ComplejidadDer{1}
		\While {\NombreFuncion {HaySiguiente}($it$)}	\ComplejidadDer{n}
		
			\State var $diccprio$: diccRapido $\gets$ \NombreFuncion {obtener}( \NombreFuncion{Siguiente}($it$).ip, $d$.CompYPaq).MasPriori)	\ComplejidadDer{L}
		\State var $dicccam$: diccRapido $\gets$ \NombreFuncion {obtener}(\NombreFuncion{Siguiente}($it$).ip, $d$.CompYPaq).PaqYCam) \ComplejidadDer{L}
			\If {$\neg$ \NombreFuncion {Vacio?}($diccprio$)} \ComplejidadDer{1}
				\State var $paq$: paquete $\gets$ \NombreFuncion{Primero}(\NombreFuncion {obtener}(\NombreFuncion {ClaveMax}($diccprio$), $diccprio$)) \ComplejidadDer{log_2(k)+1+1}
				\State \NombreFuncion {AgregarAdelante}(aux, tupla(paq: $paq$,pcant: $it$.ip, camrecorrido: \NombreFuncion{obtener}($paq$, $dicccam$)) \ComplejidadDer{1+ log_2(k)}
			
				\State \NombreFuncion {Eliminar}(\NombreFuncion{obtener}(\NombreFuncion {ClaveMax}($diccprio$), $diccprio$), $paq$) \ComplejidadDer{log_2(k)+log_2(k)+1}%elimino el primero entonces es o 1( porque paq era el primero)
				\If {\NombreFuncion{EsVacio?}(\NombreFuncion{obtener}(\NombreFuncion {ClaveMax}($diccprio$), $diccprio$)} \ComplejidadDer{log_2(k)}
					\State \NombreFuncion{borrar}(\NombreFuncion {ClaveMax}($diccprio$), $diccprio$) \ComplejidadDer{log_2(k)}
				\EndIf
				\State \NombreFuncion{borrar}($paq$, $dicccam$) \ComplejidadDer{log_2(k)}
				
				
				\State	\NombreFuncion {obtener}(\NombreFuncion{Siguiente}($it$).ip, $d$.CompYPaq).Enviados ++ \ComplejidadDer{L}
					\If {\NombreFuncion {obtener}(\NombreFuncion{Siguiente}($it$).ip, $d$.CompYPaq).Enviados > ($d$.MasEnviante).enviados }	\ComplejidadDer{L+1}
						\State $d$.MasEnviante $\gets$ tupla(\NombreFuncion{Siguiente}($it$), \NombreFuncion {obtener}(\NombreFuncion{Siguiente}($it$).ip, $d$.CompYPaq).Enviados )	\ComplejidadDer{L+1}
					\EndIf
							
				\EndIf
				
				\State \NombreFuncion{Avanzar}($it$) \ComplejidadDer{1}
		\EndWhile
		
		\State var $itaux$ $\gets$ \NombreFuncion{crearIt}(aux) \ComplejidadDer{1}
		\While {\NombreFuncion{HaySiguiente}($itaux$)} \ComplejidadDer{n}
		
			\State var $proxpc$: compu $\gets$ \NombreFuncion{Primero}(\NombreFuncion{Siguiente}(\NombreFuncion{CaminosMinimos}($d$.red, \NombreFuncion{Siguiente}($itaux$).pcant, \NombreFuncion{Siguiente}($itaux$).destino)) \ComplejidadDer{L + L}
			
			\State var $diccprio$: diccRapido $\gets$ \NombreFuncion {obtener}( $proxpc$.ip, $d$.CompYPaq).MasPriori)	\ComplejidadDer{L}
		\State var $dicccam$: diccRapido $\gets$ \NombreFuncion {obtener}($proxpc$.ip, $d$.CompYPaq).PaqYCam) \ComplejidadDer{L}
			
			\If {$proxpc$ $\neq$ (\NombreFuncion{Siguiente}($itaux$).paq).destino }
				\If {\NombreFuncion{def?}((\NombreFuncion{Siguiente}($itaux$).paq).prioridad, $diccprio$)}	\ComplejidadDer{log_2(k)}
			
					\State 	var $mismaprio$: conj(paquetes) $\gets$ \NombreFuncion{Agregar}(\NombreFuncion {obtener}(\NombreFuncion{Siguiente}($itaux$).paq.prioridad, $diccprio$), \NombreFuncion{Siguiente}($itaux$).paq)	\ComplejidadDer{log_2(k)}
					\State {\NombreFuncion{definir}((\NombreFuncion{Siguiente}($itaux$).paq).prioridad, $mismaprio$, $diccprio$)} \ComplejidadDer{log_2(k)}
		
		
		 		\Else 
		 			\State \NombreFuncion{Definir}(\NombreFuncion{Siguiente}($itaux$).prioridad, \NombreFuncion{Agregar}(\NombreFuncion{Vacio}(),\NombreFuncion{Siguiente}($itaux$).paq), $diccprio$)\ComplejidadDer{log_2(k)}
				\EndIf
		
			\State \NombreFuncion{definir}($p$.paq,\NombreFuncion{AgregarAtras}(\NombreFuncion{Siguiente}($itaux$).camrecorrido ,$proxpc$), $dicccam$)	\ComplejidadDer{log_2(k)}
			\EndIf
			\State \NombreFuncion{Avanzar}(\NombreFuncion{Siguiente}($itaux$)) \ComplejidadDer{1}
		\EndWhile
		
		}
	{\Complejidad{n*(L+log_2(k)} Donde n es la cantidad de computadoras, L la longitud del nombre mas largo de las computadoras, y k la cantidad mas grande de paquetes que tiene una computadora.}
	{$\Complejidad{1} + \Complejidad{1} + n * (\Complejidad{L} + \Complejidad{L} + \Complejidad{1} + \Complejidad{log_2(k)+1+1} + \Complejidad{1+log_2(k)} +\Complejidad{log_2(k)+log_2(k)+1} + \Complejidad{log_2(k)} + \Complejidad{log_2(k)} + \Complejidad{log_2(k)} + \Complejidad{L} + \Complejidad{L} + \Complejidad{L} + \Complejidad{1}) + n * (\Complejidad{L} + \Complejidad{L} + \Complejidad{log_2(k)} + \Complejidad{log_2(k)} +  \Complejidad{log_2(k)} + \Complejidad{log_2(k)} \Complejidad{log_2(k)} + \Complejidad{1} + \Complejidad{1} ) =$\\
	$n*(\Complejidad{5*L+5*log_2(k))} + n*\Complejidad{3*L+5*log_2(k)} =$\\
	$2n*(\Complejidad{L+log_2(k)} =$ \\
	 $\Complejidad{n*(L+log_2(k))$}
	 }
	\end{Algoritmo}
	
	
	\begin{Algoritmo}{iPaqueteEnTransito?}{\In {d}{dcnet}, \In {p}{paquete}}{bool}
	{
		\State var $it$  $\gets$ \NombreFuncion {crearIt}(\NombreFuncion {computadoras}(d.red) \ComplejidadDer{1}
		\State var $esta$: bool $\gets$ false \ComplejidadDer{1}
		\While{\NombreFuncion{HaySiguiente}($it$) $\wedge$ $\neg$ $esta$ }\ComplejidadDer{n}
			\State $esta$ $\gets$ \NombreFuncion{def?}(\NombreFuncion{obtener}(d.CompYPaq,\NombreFuncion{Siguiente}($i$).id).PaqYCam , $p$) \ComplejidadDer{log_2(k)}
			\State \NombreFuncion{Avanzar}($it$)\ComplejidadDer{1}
		\EndWhile
			\State $res$ $\gets$ $esta$	\ComplejidadDer{1}
	
	
	}
	{\Complejidad{n*log(k)}Donde n es la cantidad de computadoras en la red y k la cantidad maxima de paquetes que hay en alguna compu.}
	{$\Complejidad{1} + \Complejidad{1} + n *( \Complejidad{log_2(k)} + \Complejidad{1}) + \Complejidad{1}=$\\
	$\Complejidad{3}+ n*(\Complejidad{log_2(k)}=$\\
	$\Complejidad{n*log_2(k)}$}
	\end{Algoritmo}
	
	
	
	\begin{Algoritmo}{iLaQueMasEnvio}{\In {d}{dcnet}}{compu}
	{
		\State $res$  $\gets$ ($d$.MasEnviante).compu \ComplejidadDer{1}
	}
	{\Complejidad{1}}
	{}
	\end{Algoritmo}
	
	
	
	
	

	%\Algoritmo{IENTRAR}{\In {ts}{conj(tags)}, \In {e}{estaci�n}, \Inout {c}{ciudad}}{}{
		%\State e\_ciudad.cRUR $\gets$ e\_ciudad.cRUR + 1 \ComplejidadDer{1}
		%\State var sendas $\gets$ EVALUARSENDAS(ts, e\_ciudad.m) \ComplejidadDer{S \cdot R}
		%\State var nRUR $\gets$ tupla(e\_ciudad.RURs, e, 0, ts, sendas) \ComplejidadDer{1}
		%\State AGREGARATRAS(e\_ciudad.RURs, nRUR) \ComplejidadDer{1}
		%\State obtEst = \&(SIGNIFICADO(e.RUREnEst, e)) \ComplejidadDer{|e|}
		%\State ENCOLAR(obtEst, tupla(e\_ciudad.cRUR, 0)) \ComplejidadDer{log\ N}
	%}{\Complejidad{|e| + S \cdot R + log\ N}}{
	%Justificacion.
	%}
	%
	%\newpage
	%
	%\Algoritmo{IMOVER}{\In {u}{nat}, \In {e2}{estaci�n}, \Inout {c}{ciudad}}{}{
		%\State var est $\gets$ OBTENERCOLAEST(c.RUREnEst, c.RURs[u].e) \ComplejidadDer{|e1|}
		%\State SACARRUR(est, u) \ComplejidadDer{log\ N_{e1}}
		%\State INFRACCIONO?(u, NROCONEXION(c.RURs[u].e, e2) \ComplejidadDer{|e1| + |e2|}
		%\State MODIFICARRUR(u) \ComplejidadDer{1}
		%\State est $\gets$ OBTENERCOLAEST(c.RUREnEst, e2) \ComplejidadDer{|e2|}
		%\State METERRUR(est, u) \ComplejidadDer{log\ N_{e2}}
	%}{\Complejidad{|e1| + |e2| + log\ N_{e1} + log\ N_{e2}}}{
		%$\Complejidad{|e1|} + \Complejidad{|e2|} + \Complejidad{log\ N_{e1}} + 
		%\Complejidad{log\ N_{e2}} + \Complejidad{|e1| + |e2|} + \Complejidad{1} =$\\
		%$2 * \Complejidad{|e1| + |e2|} + \Complejidad{log\ N_{e2} + log\ N_{e1}} =$\\
		%$\Complejidad{|e1| + |e2|} + \Complejidad{log\ N_{e2} + log\ N_{e1}} =$\\
		%$\Complejidad{|e1| + |e2| + log\ N_{e2} + log\ N_{e1}}$
	%}

\end{Algoritmos}

\newpage
\section{Diccionario String}

\subsection{Interfaz}

\begin{Interfaz}

  \textbf{se explica con}: \tadNombre{Diccionario(String, $\alpha$)}.

  \textbf{g�neros}: \TipoVariable{diccString($\alpha$)}.

  \Titulo{Operaciones b�sicas de Diccionario String($\alpha$)}
  
  \InterfazFuncion{Def?}{\In{clv}{string}, \In{d}{diccString($\alpha$)}}{bool}
  [true]
  {$res \igobs$ def?($clv, d$)}
  [$\Complejidad{|clv|}$]
  [Revisa si la clave ingresada se encuentra definida en el Diccionario.]
  
  \InterfazFuncion{Obtener}{\In{clv}{string}, \In{d}{diccString($\alpha$)}}{$\alpha$}
  [def?($clv, d$)]
  {$res \igobs$ obtener($clv, d$)}
  [$\Complejidad{|clv|}$]
  [Devuelve el significado de la clave.]

  \InterfazFuncion{Vac�o}{}{diccString($\alpha$)}
  [true]
  {$res \igobs$ vac�o()}
  [\Complejidad{1}]
  [Crea nuevo diccionario vacio.]
	
<<<<<<< HEAD
  \InterfazFuncion{Definir}{\Inout{d}{diccString($\alpha$)}, \In{clv}{string}, \In{def}{$\alpha$}}{}
  [$d_0 \igobs d$]
  {$d \igobs$ definir($clv, def, d_0$)}
  [$\Complejidad{|clv|}$]
  [Agrega un nueva definicion.]
	
  \InterfazFuncion{Def?}{\In{d}{diccString($\alpha$)}, \In{clv}{string}}{bool}
  %[ACA VA EL PRE (SI LO HUBIERA)]
  {$res \igobs$ def?($clv, d$)}
  [$\Complejidad{|clv|}$]
  [Revisa si la clave ingresada se encuentra definida en el Diccionario.]
	
  \InterfazFuncion{Obtener}{\In{d}{diccString($\alpha$)}, \In{clv}{string}}{diccString($\alpha$)}
  [def?($d, clv$)]
  {$res \igobs$ obtener($clv, d$)}
=======
  \InterfazFuncion{Definir}{\In{clv}{string}, \In{def}{$\alpha$}, \Inout{d}{diccString($\alpha$)}}{}
  [$d$ $\igobs$ $d_0$]
  {$d$ $\igobs$ definir($clv, def, d_0$)}
  [$\Complejidad{|clv|}$]
  [Agrega un nueva definicion.]
	
  \InterfazFuncion{Borrar}{\In{clv}{string}, \Inout{d}{diccString($\alpha$)}}{}
  [$d$ $\igobs$ $d_0$ $\wedge$ def?(clv, d)]
  {$res$ $\igobs$ borrar($k, d_0$)}
>>>>>>> 6a8b227b980cf43c591bb3166b24fd34968fec9a
  [$\Complejidad{|clv|}$]
  [Devuelve la definicion correspondiente a la clave.]
	
\end{Interfaz}


\newpage
\section{DiccRapido}

\begin{Interfaz}
  
  \textbf{se explica con}: \tadNombre{Diccionario(clave, significado)}.

  \textbf{géneros}: \TipoVariable{diccRapido}.

  \Titulo{Operaciones básicas de DICCRAPIDO}

  \InterfazFuncion{Def?}{\In{c}{clave}, \In{d}{diccRapido}{bool}
  []
  {$res$ $\igobs$ def?($c, d$)}
  [$\Complejidad{log_2\ n}$, siendo n la cantidad de claves]
  [Verifica si una clave esta definida.]
  
  \InterfazFuncion{Obtener}{\In{c}{clave}}, \In{d}{diccRapido}}{significado}
  [def?($c, d$)]
  {$res$ $\igobs$ obtener($c, d$)}
  [$\Complejidad{log_2\ n}$, siendo n la cantidad de claves]
  [Devuelve el significado asociado a una clave]
  
  \InterfazFuncion{Vacio}{}{diccRapido}
  [true]
  {$res$ $\igobs$ vacio()}
  [$\Complejidad{1}$]
  [Crea un nuevo diccionario vacio]  
  
  \InterfazFuncion{Definir}{\In{c}{clave}, \In{s}{significado}, \Inout{d}{diccRapido}}{}
  [$d$ $\igobs$ $d_0$]
  {$d$ $\igobs$ definir($c, s, d_0$)}
  [$\Complejidad{log_2\ n}$, siendo n la cantidad de claves]
  [Define la clave, asociando su significado, al diccionario]
  
  \InterfazFuncion{Borrar}{\In{c}{clave}, \Inout{d}{diccRapido}}{}
  [$d$ $\igobs$ $d_0$ $\wedge$ def?($c, d_0$)]
  {$d$ $\igobs$ borrar($c, d_0$)}
  [$\Complejidad{log_2 \n}$, siendo n la cantidad de claves]
  [Borra la clave del diccionario]

  \InterfazFuncion{Claves}{\In{d}{diccRapido}}{itPaquete}
  []
  {$res$ $\igobs$ claves($d$)}
  [$\Complejidad{1}$]
  [Devuelve un iterador de las claves, que son el id del paquete, pero con las demas componentes de su tupla unida]  
  
\end{Interfaz}

\subsection{Representacion}



\begin{Representacion}

\textbf{OJO, SOLO ESTA HECHA LA INTERFAZ. DE ACA EN ADELANTE ES TODO VIEJO}

Para representar la cola de prioridad, elegimos hacerla sobre un AVL. Sabiendo que la cantidad de paquetes no está acotada, este AVL estará representado con nodos y punteros.

\begin{Estructura}{colaP(Paquete)}[estr]
	\begin{Tupla}[estr]
		\tupTupItem{raiz}{\TipoVariable{puntero}(nodo)}
		\tupItem{tam}{nat}
		%\tupItem{cVac}{bool}
	\end{Tupla}
	
	\begin{Tupla}[nodo]
		\tupItem{paquete}{Paquete}
		\tupItem{caminoRecorrido}{Secu(Compu)}
		\tupTupItem{padre}{\TipoVariable{puntero}(nodo)}
		\tupTupItem{izq}{\TipoVariable{puntero}(nodo)}
		\tupTupItem{der}{\TipoVariable{puntero}(nodo)}
		\tupTupItem{alt}{\TipoVariable{nat}}
	\end{Tupla}
	
\end{Estructura}

\subsection{InvRep y Abs}


\textbf{InvRep en lenguaje coloquial:}

\begin{enumerate}
	\item{La componente ''tam'' de e\_cola es igual a la cantidad de nodos en el arbol.}
	\item{Todo nodo en el arbol tiene un unico padre, con excepcion de la raiz, que no tiene padre.}
	\item{La relacion de orden es total.}
	\item{Un nodo es mayor a otro si la componente ''pri'' del primero es mayor que la del segundo.}
	\item{Un nodo es menor a otro si la componente ''pri'' del primero es menor que la del segundo.}
	\item{No pueden haber dos nodos en el arbol que tengan el mismo numero en la componente ''seg''.}
	\item{Si dos nodos tienen el mismo numero en la componente ''pri'', se procede a verficar la 
	componente ''seg'' de ambos. El que tiene el mayor numero en dicha componente es el mayor, mientras
	que el otro es el menor.}
	\item{Para cada nodo, todos los elementos del subarabol que se encuentra a la derecha de la raiz
	son mayores que la misma.}
	\item{Para cada nodo, todos los elementos del subarabol que se encuentra a la izquierda de la raiz
	son menores que la misma.}
	\item{La componente ''alt'' de cada nodo es igual a la cantidad de niveles que hay que recorrer
	para llegar a la hoja mas lejana.}
	\item{Para cada nodo, la diferencia en modulo de la altura entre los dos subarboles del mismo no
	puede diferir en mas de 1.}
\end{enumerate}

\vspace{2em}


\textbf{Abs:}

\vspace{1em}

\Abs[colaP(Paquete)]{colaP(Paquete)}[c]{p}{mismosProximos(c, p)}

\vspace{1em}

\tadOperacion{mismosProximos}{$\langle$puntero(nodo), nat$\rangle$}{bool}{}

\vspace{1em}

\tadAxioma{mismosProximos(c, p)}{\IF $\pi_1$(c) $=$ NULL $\wedge$ vacia?(p)
																	THEN true ELSE {\IF
																	($\pi_1$(c) $=$ NULL $\wedge$ $\neg$vacia?(p)) $\vee$
																	($\pi_1$(c) $\neq$ NULL $\wedge$ vacia?(p))
																	THEN false ELSE {\IF
																	maxElem(*($\pi_1$(c))) $=$ proximo(p)
																	THEN mismosProximos(borrarMax(*($\pi_1$(c)), 
																	borrar($\pi_1$(proximo(p)), $\pi_2$(proximo(p)), p) ELSE
																	false FI}FI}FI}
\vspace{1em}

Las funciones ''maxElem'' y ''borrarMax'' no han sido axiomatizadas. Ya que estamos trabajando con Arboles Binarios de Búsqueda (en nuestro caso AVL), la lógica de ambas funciones es la misma que está expresada en el pseudocódigo del módulo. En particular ''maxElem'' se limita a buscar el nodo más a la derecha del árbol, mientras que ''borrarMax'' (una vez encontrado el máximo) procede a eliminarlo y reordenar el árbol.

\end{Representacion}

\subsection{Algoritmos}

\begin{Algoritmos}

\Algoritmo{IVacía?}{\In{cp}{colaP(Paquete)}}{Bool}{
	\State res $\gets$ cp.raiz $==$ NULL \ComplejidadDer{1}
	}{\Complejidad{1}}{
}
	
\Algoritmo{IPróximo}{\In{cp}{colaP(Paquete)}}{Paquete}{
	\State var pNodo: puntero(nodo) $\gets$ cp.raiz \ComplejidadDer{1}
		\While{*(pNodo).der != NULL} \ComplejidadDer{log_2\ k}
			\State pNodo $\gets$ *(pNodo).der \ComplejidadDer{1}
		\EndWhile
	\State $res$ $\gets$ *(pNodo).paquete \ComplejidadDer{1}
	}{\Complejidad{log_2\ k}}{
	$\Complejidad{1} + \Complejidad{log_2\ k} * \Complejidad{1} + \Complejidad{1} =\newline$
	$\Complejidad{log_2\ k}$
}

\Algoritmo{IDesencolar}{\Inout{cp}{colaP(Paquete)}}{}{
	\State var pNodo: puntero(nodo) $\gets$ cp.raiz \ComplejidadDer{1}
	\While{*(pNodo).der != NULL} \ComplejidadDer{log_2\ k}
		\State pNodo $\gets$ *(pNodo).der \ComplejidadDer{1}
	\EndWhile
	\State \NombreFuncion{ELIMINAR}(cp, *(pNodo).pri, *(pNodo).seg) \ComplejidadDer{log_2\ k}
	}{\Complejidad{log_2\ k}}{
	$\Complejidad{1} + \Complejidad{log_2\ k} * \Complejidad{1} + \Complejidad{log_2\ k} =\newline$
	$2 * \Complejidad{log_2\ k} = \Complejidad{log_2\ k}$
}

\Algoritmo{ISuCamino}{\In{p}{Paquete}, \In{cp}{colaP(Paquete)}}{Secu(Compu)}{
	\State var pNodo: puntero(nodo) $\gets$ cp.raiz \ComplejidadDer{1}
	\While{*(pNodo).paquete != p} \ComplejidadDer{log_2\ k}
		\If{p.prioridad $\textless$ *(pNodo).paquete.prioridad}	\ComplejidadDer{1}	
			\State pNodo $\gets$ *(pNodo).izq \ComplejidadDer{1}
		\Else
			\State pNodo $\gets$ *(pNodo).der \ComplejidadDer{1}
	\EndWhile
	\State $res$ $\gets$ *(pNodo).caminoRecorrido \ComplejidadDer{1}			
	}{\Complejidad{log_2\ k}}{
	$\Complejidad{1} + \Complejidad{log_2\ k} * (\Complejidad{1} + \Complejidad{1} + \Complejidad{1}) + \Complejidad{1} = \newline$
	$2 * \Complejidad{1} + \Complejidad{log_2\ k} * 3 * \Complejidad{1} = \newline$
	$\Complejidad{log_2\ k}$
}

\Algoritmo{iVacia}{}{colaP(Paquete)}{
	\State var res: colaP(Paquete) $\gets$ tupla(NULL, 0) \ComplejidadDer{1}
	}{\Complejidad{1}}{}

\Algoritmo{IAgregar}{\Inout{c}{colaP(Paquete)}, \In{p}{Paquete}}{}{
	\If{c.raiz $==$ NULL}
		\State c.raiz $\gets$ \&(tupla(p, <>, NULL, NULL, NULL)) \ComplejidadDer{1}
		\State c.tam $\gets$ 1 \ComplejidadDer{1}
	\Else
		\State var iTam: int $\gets$ c.tam \ComplejidadDer{1}
		\State var iAux: int $\gets$ c.tam \ComplejidadDer{1}
		\State var aCoordenadas: arreglo[$\lfloor log_2(c.tam)\rfloor$] de bool 	\ComplejidadDer{\lfloor log_2(c.tam)\rfloor}
		\While{iTam $>$ 0} \ComplejidadDer{1}
			\If{iAux\%2 $==$ 0} \ComplejidadDer{1}
				\State aCoordenadas[iTam-1] $=$ false \ComplejidadDer{1}
			\Else
				\State aCoordenadas[iTam-1] $=$ true \ComplejidadDer{1}
			\EndIf
			\State iAux $\gets$ $\lfloor iAux/2 \rfloor$ \ComplejidadDer{1}
			\State iTam $\gets$ iTam - 1
		\EndWhile
	
	
		\State var seguir: bool $\gets$ true \ComplejidadDer{1}
		\State var pNodo: puntero(nodo) $\gets$ c.raiz \ComplejidadDer{1}
		\State var camino: arreglo[$\lfloor log_2(c.tam)\rfloor + 1$] de puntero(nodo) 
		\ComplejidadDer{\lfloor log_2(c.tam)\rfloor + 1}
		\State var nroCamino: nat 
		\State camino[0] $\gets$ pNodo \ComplejidadDer{1}
		\State nroCamino $\gets$ 0 \ComplejidadDer{1}
		\While{seguir $==$ true} \ComplejidadDer{1}
			\If{a $\geq$ *(pNodo).pri} \ComplejidadDer{1}
				\If{a $==$ *(pNodo).pri} \ComplejidadDer{1}
					\If{b $>$ *(pNodo).seg} \ComplejidadDer{1}
						\If{*(pNodo).der $!=$ NULL} \ComplejidadDer{1}
							\State pNodo $\gets$ *(pNodo).der \ComplejidadDer{1}
							\State nroCamino $\gets$ nroCamino $+$ 1 \ComplejidadDer{1}
							\State camino[nroCamino] $\gets$ pNodo \ComplejidadDer{1}
						\Else
							\State *(pNodo).der $\gets$ \&(tupla(a, b, pNodo, NULL, NULL, 1)) \ComplejidadDer{1}
							\State nroCamino $\gets$ nroCamino $+$ 1 \ComplejidadDer{1}
							\State camino[nroCamino] $\gets$ *(pNodo).der \ComplejidadDer{1}
							\State seguir $\gets$ false \ComplejidadDer{1}
						\EndIf
					\Else
						\If{*(pNodo).izq $!=$ NULL} \ComplejidadDer{1}
							\State pNodo $\gets$ *(pNodo).izq \ComplejidadDer{1}
							\State nroCamino $\gets$ nroCamino $+$ 1 \ComplejidadDer{1}
							\State camino[nroCamino] $\gets$ pNodo \ComplejidadDer{1}
						\Else
							\State *(pNodo).izq $\gets$ \&(tupla(a, b, pNodo, NULL, NULL, 1)) \ComplejidadDer{1}
							\State nroCamino $\gets$ nroCamino $+$ 1 \ComplejidadDer{1}
							\State camino[nroCamino] $\gets$ *(pNodo).izq \ComplejidadDer{1}
							\State seguir $\gets$ false \ComplejidadDer{1}
						\EndIf
					\EndIf
				\Else
					\If{*(pNodo).der $!=$ NULL} \ComplejidadDer{1}
						\State pNodo $\gets$ *(pNodo).der \ComplejidadDer{1}
						\State nroCamino $\gets$ nroCamino $+$ 1 \ComplejidadDer{1}
						\State camino[nroCamino] $\gets$ pNodo \ComplejidadDer{1}
					\Else
						\State *(pNodo).der $\gets$ \&(tupla(a, b, pNodo, NULL, NULL, 1)) \ComplejidadDer{1}
						\State nroCamino $\gets$ nroCamino $+$ 1 \ComplejidadDer{1}
						\State camino[nroCamino] $\gets$ *(pNodo).der \ComplejidadDer{1}
						\State seguir $\gets$ false \ComplejidadDer{1}
					\EndIf					
				\EndIf
			\Else
				\If{*(pNodo).izq $!=$ NULL} \ComplejidadDer{1}
					\State pNodo $\gets$ *(pNodo).izq \ComplejidadDer{1}
					\State nroCamino $\gets$ nroCamino $+$ 1 \ComplejidadDer{1}
					\State camino[nroCamino] $\gets$ pNodo \ComplejidadDer{1}
				\Else
					\State *(pNodo).izq $\gets$ \&(tupla(a, b, pNodo, NULL, NULL, 1)) \ComplejidadDer{1}
					\State nroCamino $\gets$ nroCamino $+$ 1 \ComplejidadDer{1}
					\State camino[nroCamino] $\gets$ *(pNodo).izq \ComplejidadDer{1}
					\State seguir $\gets$ false \ComplejidadDer{1}
				\EndIf
			\EndIf
		\EndWhile
		\State c.tam $\gets$ c.tam $+$ 1 \ComplejidadDer{1}
		\State seguir $\gets$ true \ComplejidadDer{1}
		\While{nroCamino $\geq$ 0 $\wedge$ seguir $==$ true} \ComplejidadDer{\lfloor log_2\ N\rfloor + 1}
			\State pNodo $\gets$ camino[nroCamino] \ComplejidadDer{1}
			\State *(pNodo).alt $\gets$ \NombreFuncion{Altura}(pNodo) v
			\If{$|$\NombreFuncion{FactorDesbalance}(camino[nroCamino])$|$ $>$ 1} \ComplejidadDer{1}
				\State pNodo $\gets$ \NombreFuncion{Rotar}(\NombreFuncion{HijoMasAlto}
				(\NombreFuncion{HijoMasAlto}(pNodo)), HijoMasAlto(pNodo), pNodo) \ComplejidadDer{1}
				\State *(*(pNodo).izq).alt $\gets$ \NombreFuncion{Altura}(*(pNodo).izq) \ComplejidadDer{1}
				\State *(*(pNodo).der).alt $\gets$ \NombreFuncion{Altura}(*(pNodo).der) \ComplejidadDer{1}
				\State *(pNodo).alt $\gets$ \NombreFuncion{Altura}(*(pNodo)) \ComplejidadDer{1}
				\State seguir $\gets$ false \ComplejidadDer{1}
			\EndIf
			\State nroCamino $\gets$ nroCamino $-$ 1 \ComplejidadDer{1}
		\EndWhile
	\EndIf
	}{\Complejidad{1}}{
	Debido a la longitud del pseudocodigo, vamos a ignorar todas los condicionales y las asignaciones en la justificacion, ya que se realizan en tiempo constante. Solo nos vamos a centrar en dos puntos, la creacion del arreglo ''camino'' y el ultimo ciclo.\\
	Para el arreglo asignamos esa cantidad de nodos ya que contamos con un Arbol balanceado, el cual como mucho puede necesitar de $\lfloor log_2\ N\rfloor + 1$ niveles para almacenar $N$ nodos. Esta misma logica la utilizamos en el ultimo ciclo, en el cual para restaurar el balance del Arbol recorremos el mismo desde el ultimo nodo agregado (el cual es una hoja) hasta la raiz en el peor caso, corrigiendo cualquier desbalance en el camino.\\
	Esto resulta en la siguiente suma:\\
	$\Complejidad{\lfloor log_2\ N\rfloor + 1} + \Complejidad{\lfloor log_2\ N\rfloor + 1} =\newline
	2 * \Complejidad{\lfloor log_2\ N\rfloor + 1} =\newline
	\Complejidad{\lfloor log_2\ N\rfloor + 1} =\newline 
	\Complejidad{\lfloor log_2\ N\rfloor} = \Complejidad{log_2\ N}$
	}
	
\Algoritmo{IEliminar}{\Inout{c}{colaP(Paquete)}, \In{a}{nat}, \In{b}{nat}}{}{
	\State var pNodo: puntero(nodo) $\gets$ c.raiz \ComplejidadDer{1}
	\State ver seguir: bool $\gets$ true \ComplejidadDer{1}
	\While{pNodo $!=$ NULL $\wedge$ seguir $==$ true} \ComplejidadDer{\lfloor log_2\ N\rfloor + 1}
		\If{*(pNodo).pri $==$ a $\wedge$ *(pNodo).seg $==$ b} \ComplejidadDer{1}
			\State seguir $\gets$ false \ComplejidadDer{1}
		\Else
			\If{a $\geq$ *(pNodo).pri} \ComplejidadDer{1}
				\If{a $==$ *(pNodo).pri} \ComplejidadDer{1}
					\If{b $>$ *(pNodo).seg} \ComplejidadDer{1}
						\State pNodo $\gets$ *(pNodo).der \ComplejidadDer{1}
					\Else
						\State pNodo $\gets$ *(pNodo).izq \ComplejidadDer{1}
					\EndIf
				\Else
					 \State pNodo $\gets$ *(pNodo).der \ComplejidadDer{1}
				\EndIf
			\Else
				\State pNodo $\gets$ *(pNodo).izq \ComplejidadDer{1}
			\EndIf
		\EndIf
	\EndWhile
	\If{pNodo != NULL} \ComplejidadDer{1}
		\State bNodo: puntero(nodo) $\gets$ NULL \ComplejidadDer{1}
		\If{pNodo $==$ c.raiz} \ComplejidadDer{1}
			\If{*(pNodo).izq $==$ NULL $\wedge$ *(pNodo).der $==$ NULL} \ComplejidadDer{1}
				\State c.raiz $\gets$ NULL \ComplejidadDer{1}
				\State delete pNodo \ComplejidadDer{1}
			\EndIf
			\If{*(pNodo).izq $!=$ NULL $\wedge$ *(pNodo).der $==$ NULL} \ComplejidadDer{1}
				\State c.raiz $\gets$ *(pNodo).izq \ComplejidadDer{1}
				\State *(*(pNodo).izq).padre $\gets$ NULL \ComplejidadDer{1}
				\State delete pNodo \ComplejidadDer{1}
			\EndIf
			\If{*(pNodo).izq $==$ NULL $\wedge$ *(pNodo).der $!=$ NULL} \ComplejidadDer{1}
				\State c.raiz $\gets$ *(pNodo).der \ComplejidadDer{1}
				\State *(*(pNodo).der).padre $\gets$ NULL \ComplejidadDer{1}
				\State delete pNodo \ComplejidadDer{1}
			\EndIf
			\If{*(pNodo).izq $!=$ NULL $\wedge$ *(pNodo).der $!=$ NULL} \ComplejidadDer{1}
				\State var tNodo: puntero(nodo) $\gets$ pNodo.der \ComplejidadDer{1}
				\While{*(tNodo).izq != NULL} \ComplejidadDer{\lfloor log_2\ N\rfloor + 1}
					\State tNodo $\gets$ tNodo.izq \ComplejidadDer{1}
				\EndWhile
				\State bNodo $\gets$ *(tNodo).padre \ComplejidadDer{1}
				\If{*(tNodo).der $!=$ NULL} \ComplejidadDer{1}
					\State *(*(tNodo).der).padre $\gets$ *(tNodo).padre \ComplejidadDer{1}
				\EndIf
				\If{*(*(tNodo).padre).izq $==$ tNodo} \ComplejidadDer{1}
					\State *(*(tNodo).padre).izq $\gets$ *(tNodo).der \ComplejidadDer{1}
				\Else
					\State *(*(tNodo).padre).der $\gets$ *(tNodo).der \ComplejidadDer{1}
				\EndIf
				\State *(tNodo).padre $\gets$ *(pNodo).padre \ComplejidadDer{1}
				\State *(tNodo).izq $\gets$ *(pNodo).izq \ComplejidadDer{1}
				\State *(tNodo).der $\gets$ *(pNodo).der \ComplejidadDer{1}
				\State *(*(pNodo).izq).padre $\gets$ tNodo \ComplejidadDer{1}
				\State *(*(pNodo).der).padre $\gets$ tNodo \ComplejidadDer{1}
				\State delete pNodo \ComplejidadDer{1}
			\EndIf
		\Else
			\If{*(pNodo).izq $==$ NULL $\wedge$ *(pNodo).der $==$ NULL} \ComplejidadDer{1}
				\If{*(*(pNodo).padre).izq $==$ pNodo} \ComplejidadDer{1}
					\State *(*(pNodo).padre).izq $\gets$ NULL \ComplejidadDer{1}
					\State bNodo $\gets$ *(pNodo).padre \ComplejidadDer{1}
					\State delete pNodo \ComplejidadDer{1}
				\Else
					\State *(*(pNodo).padre).der $\gets$ NULL \ComplejidadDer{1}
					\State bNodo $\gets$ *(pNodo).padre \ComplejidadDer{1}
					\State delete pNodo \ComplejidadDer{1}
				\EndIf
			\EndIf
			\If{*(pNodo).izq $!=$ NULL $\wedge$ *(pNodo).der $==$ NULL} \ComplejidadDer{1}
				\If{*(*(pNodo).padre).izq $==$ pNodo} \ComplejidadDer{1}
					\State *(*(pNodo).padre).izq $\gets$ *(pNodo).izq \ComplejidadDer{1}
					\State *(*(pNodo).izq).padre $\gets$ *(pNodo).padre \ComplejidadDer{1}
					\State bNodo $\gets$ *(pNodo).padre \ComplejidadDer{1}
					\State delete pNodo \ComplejidadDer{1}
				\Else
					\State *(*(pNodo).padre).der $\gets$ *(pNodo).izq \ComplejidadDer{1}
					\State *(*(pNodo).izq).padre $\gets$ *(pNodo).padre \ComplejidadDer{1}
					\State bNodo $\gets$ *(pNodo).padre \ComplejidadDer{1}
					\State delete pNodo \ComplejidadDer{1}
				\EndIf
			\EndIf
			\If{*(pNodo).izq $==$ NULL $\wedge$ *(pNodo).der $!=$ NULL} \ComplejidadDer{1}
				\If{*(*(pNodo).padre).izq $==$ pNodo} \ComplejidadDer{1}
					\State *(*(pNodo).padre).izq $\gets$ *(pNodo).der \ComplejidadDer{1}
					\State *(*(pNodo).der).padre $\gets$ *(pNodo).padre \ComplejidadDer{1}
					\State bNodo $\gets$ *(pNodo).padre \ComplejidadDer{1}
					\State delete pNodo \ComplejidadDer{1}
				\Else
					\State *(*(pNodo).padre).der $\gets$ *(pNodo).der \ComplejidadDer{1}
					\State *(*(pNodo).der).padre $\gets$ *(pNodo).padre \ComplejidadDer{1}
					\State bNodo $\gets$ *(pNodo).padre \ComplejidadDer{1}
					\State delete pNodo \ComplejidadDer{1}
				\EndIf
			\EndIf
			\If{*(pNodo).izq $!=$ NULL $\wedge$ *(pNodo).der $!=$ NULL} \ComplejidadDer{1}
				\State var tNodo: puntero(nodo) $\gets$ pNodo.der \ComplejidadDer{1}
				\While{*(tNodo).izq != NULL} \ComplejidadDer{\lfloor log_2\ N\rfloor + 1}
					\State tNodo $\gets$ tNodo.izq \ComplejidadDer{1}
				\EndWhile
				\State bNodo $\gets$ *(tNodo).padre \ComplejidadDer{1}
				\If{*(tNodo).der $!=$ NULL} \ComplejidadDer{1}
					\State *(*(tNodo).der).padre $\gets$ *(tNodo).padre \ComplejidadDer{1}
				\EndIf
				\If{*(*(tNodo).padre).izq $==$ tNodo} \ComplejidadDer{1}
					\State *(*(tNodo).padre).izq $\gets$ *(tNodo).der \ComplejidadDer{1}
				\Else
					\State *(*(tNodo).padre).der $\gets$ *(tNodo).der \ComplejidadDer{1}
				\EndIf
				\If{*(*(pNodo).padre).izq $==$ pNodo} \ComplejidadDer{1}
					\State *(*(tNodo).padre).izq $\gets$ tNodo \ComplejidadDer{1}
				\Else
					\State *(*(tNodo).padre).der $\gets$ pNodo \ComplejidadDer{1}
				\EndIf
				\State *(tNodo).padre $\gets$ *(pNodo).padre \ComplejidadDer{1}
				\State *(tNodo).izq $\gets$ *(pNodo).izq \ComplejidadDer{1}
				\State *(tNodo).der $\gets$ *(pNodo).der \ComplejidadDer{1}
				\State *(*(pNodo).izq).padre $\gets$ tNodo \ComplejidadDer{1}
				\State *(*(pNodo).der).padre $\gets$ tNodo \ComplejidadDer{1}
				\State delete pNodo \ComplejidadDer{1}
			\EndIf
		\State c.tam $\gets$ c.tam $+$ 1
		\While{b $!=$ NULL} \ComplejidadDer{\lfloor log_2\ N\rfloor + 1}
			\State *(b).alt $\gets$ \NombreFuncion{SetAltura}(b) \ComplejidadDer{1}
			\If{|\NombreFuncion{FactorDeDesbalance}(b)| $>$ 1} \ComplejidadDer{1}
				\State *(b).alt $\gets$ \NombreFuncion{Rotar}(\NombreFuncion{HijoMasAlto}
				(\NombreFuncion{HijoMasAlto}(b)), \NombreFuncion{HijoMasAlto}(b), b) \ComplejidadDer{1}
				\State *(*(b).izq).alt $\gets$ \NombreFuncion{SetAltura}(*(b).izq) \ComplejidadDer{1}
				\State *(*(b).der).alt $\gets$ \NombreFuncion{SetAltura}(*(b).der) \ComplejidadDer{1}
				\State *(b).alt $\gets$ \NombreFuncion{SetAltura}(*(b)) \ComplejidadDer{1}
			\EndIf
			\State b $\gets$ *(b).padre \ComplejidadDer{1}
		\EndWhile
		\EndIf	
	\EndIf
	
	}{\Complejidad{log_2\ N}}{
	Para la complejidad de este algoritmo nos vamos a remitir al mismo proceso que en el caso anterior, vamos a ignorar los condicionales y las asignaciones ya que estas se realizan en tiempo constante para centrarnos unicamente en los ciclos cuya complejidad depende de algun parametro.\\
	En este algoritmo contamos con 4 ciclos que dependen de alguna variable, ninguno esta anidado con ningun otro ciclo, y en el peor caso solo recorremos 3 de ellos.\\
	Esto nos da la siguiente suma:\\
	$3 * \Complejidad{\lfloor log_2\ N\rfloor + 1} =\newline
	\Complejidad{\lfloor log_2\ N\rfloor + 1} =\newline
	\Complejidad{\lfloor log_2\ N\rfloor} = \Complejidad{log_2\ N}$
	
	}

\Algoritmo{IRotar}{\Inout{c}{colaP(Paquete)}, \In{p1}{puntero(nodo)}, \In{p2}{puntero(nodo)}, \In{p3}{puntero(nodo)}}{puntero(nodo)}{
	\State var t1: puntero(nodo) $\gets$ NULL \ComplejidadDer{1}
	\State var t2: puntero(nodo) $\gets$ NULL \ComplejidadDer{1}
	\State var t2: puntero(nodo) $\gets$ NULL v
	\If{(*(p3).pri $\leq$ *(p1).pri $\wedge$ *(p3).seg $<$ *(p1).seg) $\wedge$\\
	\	\ \ (*(p1).pri $\leq$ *(p2).pri $\wedge$ *(p1).pri $<$ *(p2).pri)} \ComplejidadDer{1}
		\State t1 $\gets$ p3 \ComplejidadDer{1}
		\State t2 $\gets$ p1 \ComplejidadDer{1}
		\State t3 $\gets$ p2 \ComplejidadDer{1}
	\EndIf
	\If{(*(p3).pri $\geq$ *(p1).pri $\wedge$ *(p3).seg $>$ *(p1).seg) $\wedge$\\
	\ \	\ (*(p1).pri $\geq$ *(p2).pri $\wedge$ *(p1).pri $>$ *(p2).pri)} \ComplejidadDer{1}
		\State t1 $\gets$ p2 \ComplejidadDer{1}
		\State t2 $\gets$ p1 \ComplejidadDer{1}
		\State t3 $\gets$ p3 \ComplejidadDer{1}
	\EndIf
	\If{(*(p3).pri $\leq$ *(p2).pri $\wedge$ *(p3).seg $<$ *(p2).seg) $\wedge$\\
	\ \	\ (*(p2).pri $\geq$ *(p1).pri $\wedge$ *(p2).pri $<$ *(p1).pri)} \ComplejidadDer{1}
		\State t1 $\gets$ p3 \ComplejidadDer{1}
		\State t2 $\gets$ p2 \ComplejidadDer{1}
		\State t3 $\gets$ p1 \ComplejidadDer{1}
	\EndIf
	\If{(*(p3).pri $\geq$ *(p2).pri $\wedge$ *(p3).seg $>$ *(p2).seg) $\wedge$\\
	\ \ \	(*(p2).pri $\geq$ *(p3).pri $\wedge$ *(p2).pri $>$ *(p3).pri)} \ComplejidadDer{1}
		\State t1 $\gets$ p1 \ComplejidadDer{1}
		\State t2 $\gets$ p2 \ComplejidadDer{1}
		\State t3 $\gets$ p3 \ComplejidadDer{1}
	\EndIf
	\If{c.raiz == p3}
		\State c.raiz $\gets$ p3 \ComplejidadDer{1}
		\State *(p3).padre $\gets$ NULL \ComplejidadDer{1}
	\Else
		\If{*(*(p3).padre).izq = p3} \ComplejidadDer{1}
			\State \NombreFuncion{Cizq}(*(p3).padre, t2) \ComplejidadDer{1}
		\Else
			\State \NombreFuncion{Cder}(*(p3).padre, t2) \ComplejidadDer{1}
		\EndIf
	\EndIf
	\If{*(t2).izq != p1 $\wedge$ *(t2).izq != p2 $\wedge$ *(t2).izq != p3}
		\State \NombreFuncion{Cder}(t1, *(t2).izq) \ComplejidadDer{1}
	\EndIf
	\If{*(t2).der != p1 $\wedge$ *(t2).der != p2 $\wedge$ *(t2).der != p3}
		\State \NombreFuncion{Cder}(t3, *(t2).der) \ComplejidadDer{1}
	\EndIf
	\State \NombreFuncion{Cizq}(t2, t1) \ComplejidadDer{1}
	\State \NombreFuncion{Cder}(t2, t3) \ComplejidadDer{1}
	\State res $\gets$ t2 \ComplejidadDer{1}
	}{\Complejidad{1}}{
	Al igual que en los dos casos anteriores, debido a la longitud del pseudocodigo, vamos a ignorar los condicionales y las asignaciones ya que se realizan en tiempo constante.\newline
	Como todas las ejecuciones del codigo se efectuan en tiempo constante, podemos ver de manera trivial que la complejidad es $\Complejidad{1}$.
	}
	
%\Algoritmo{IEliminar}{\Inout{c}{colaP(Paquete)}, \In{a}{nat}, \In{b}{nat}}{}{
	%\State var pNodo: puntero(nodo) $\gets$ c.raiz
	%\State ver seguir: bool $\gets$ true
	%\While{pNodo $!=$ NULL $\wedge$ seguir $==$ true}
		%\If{*(pNodo).pri $==$ a $\wedge$ *(pNodo).seg $==$ b}
			%\State seguir $\gets$ false
		%\Else
			%\If{a $\geq$ *(pNodo).pri}
				%\If{a $==$ *(pNodo).pri}
					%\If{b $>$ *(pNodo).seg}
						%\State pNodo $\gets$ *(pNodo).der
					%\Else
						%\State pNodo $\gets$ *(pNodo).izq
					%\EndIf
				%\Else
					 %\State pNodo $\gets$ *(pNodo).der
				%\EndIf
			%\Else
				%\State pNodo $\gets$ *(pNodo).izq
			%\EndIf
		%\EndIf
	%\EndWhile
	%\If{pNodo != NULL}
		%\State bNodo: puntero(nodo) $\gets$ NULL
		%\If{pNodo $==$ c.raiz}
			%\If{*(pNodo).izq $==$ NULL $\wedge$ *(pNodo).der $==$ NULL}
				%\State c.raiz $\gets$ NULL
				%\State delete pNodo
			%\EndIf
			%\If{*(pNodo).izq $!=$ NULL $\wedge$ *(pNodo).der $==$ NULL}
				%\State c.raiz $\gets$ *(pNodo).izq
				%\State *(*(pNodo).izq).padre $\gets$ NULL
				%\State delete pNodo
			%\EndIf
			%\If{*(pNodo).izq $==$ NULL $\wedge$ *(pNodo).der $!=$ NULL}
				%\State c.raiz $\gets$ *(pNodo).der
				%\State *(*(pNodo).der).padre $\gets$ NULL
				%\State delete pNodo
			%\EndIf
			%\If{*(pNodo).izq $!=$ NULL $\wedge$ *(pNodo).der $!=$ NULL}
				%\State var tNodo: puntero(nodo) $\gets$ pNodo.der
				%\While{*(tNodo).izq != NULL}
					%\State tNodo $\gets$ tNodo.izq
				%\EndWhile
				%\State bNodo $\gets$ *(tNodo).padre
				%\If{*(tNodo).der $!=$ NULL}
					%\State *(*(tNodo).der).padre $\gets$ *(tNodo).padre
				%\EndIf
				%\If{*(*(tNodo).padre).izq $==$ tNodo}
					%\State *(*(tNodo).padre).izq $\gets$ *(tNodo).der
				%\Else
					%\State *(*(tNodo).padre).der $\gets$ *(tNodo).der
				%\EndIf
				%\State *(tNodo).padre $\gets$ *(pNodo).padre
				%\State *(tNodo).izq $\gets$ *(pNodo).izq
				%\State *(tNodo).der $\gets$ *(pNodo).der
				%\State *(*(pNodo).izq).padre $\gets$ tNodo
				%\State *(*(pNodo).der).padre $\gets$ tNodo
				%\State delete pNodo
			%\EndIf
		%\Else
			%\If{*(pNodo).izq == NULL $\wedge$ *(pNodo).der == NULL}
				%\If{*(*(pNodo).padre).izq == pNodo}
					%\State *(*(pNodo).padre).izq $\gets$ NULL
					%\State bNodo $\gets$ *(pNodo).padre
					%\State delete pNodo
				%\Else
					%\State *(*(pNodo).padre).der $\gets$ NULL
					%\State bNodo $\gets$ *(pNodo).padre
					%\State delete pNodo
				%\EndIf
			%\EndIf
			%\If{*(pNodo).izq != NULL $\wedge$ *(pNodo).der == NULL}
				%\If{*(*(pNodo).padre).izq == pNodo}
					%\State *(*(pNodo).padre).izq $\gets$ *(pNodo).izq
					%\State *(*(pNodo).izq).padre $\gets$ *(pNodo).padre
					%\State bNodo $\gets$ *(pNodo).padre
					%\State delete pNodo
				%\Else
					%\State *(*(pNodo).padre).der $\gets$ *(pNodo).izq
					%\State *(*(pNodo).izq).padre $\gets$ *(pNodo).padre
					%\State bNodo $\gets$ *(pNodo).padre
					%\State delete pNodo
				%\EndIf
			%\EndIf
			%\If{*(pNodo).izq == NULL $\wedge$ *(pNodo).der != NULL}
				%\If{*(*(pNodo).padre).izq == pNodo}
					%\State *(*(pNodo).padre).izq $\gets$ *(pNodo).der
					%\State *(*(pNodo).der).padre $\gets$ *(pNodo).padre
					%\State bNodo $\gets$ *(pNodo).padre
					%\State delete pNodo
				%\Else
					%\State *(*(pNodo).padre).der $\gets$ *(pNodo).der
					%\State *(*(pNodo).der).padre $\gets$ *(pNodo).padre
					%\State bNodo $\gets$ *(pNodo).padre
					%\State delete pNodo
				%\EndIf
			%\EndIf
			%\If{*(pNodo).izq $!=$ NULL $\wedge$ *(pNodo).der $!=$ NULL}
				%\State var tNodo: puntero(nodo) $\gets$ pNodo.der
				%\While{*(tNodo).izq != NULL}
					%\State tNodo $\gets$ tNodo.izq
				%\EndWhile
				%\State bNodo $\gets$ *(tNodo).padre
				%\If{*(tNodo).der $!=$ NULL}
					%\State *(*(tNodo).der).padre $\gets$ *(tNodo).padre
				%\EndIf
				%\If{*(*(tNodo).padre).izq $==$ tNodo}
					%\State *(*(tNodo).padre).izq $\gets$ *(tNodo).der
				%\Else
					%\State *(*(tNodo).padre).der $\gets$ *(tNodo).der
				%\EndIf
				%\If{*(*(pNodo).padre).izq $==$ pNodo}
					%\State *(*(tNodo).padre).izq $\gets$ tNodo
				%\Else
					%\State *(*(tNodo).padre).der $\gets$ pNodo
				%\EndIf
				%\State *(tNodo).padre $\gets$ *(pNodo).padre
				%\State *(tNodo).izq $\gets$ *(pNodo).izq
				%\State *(tNodo).der $\gets$ *(pNodo).der
				%\State *(*(pNodo).izq).padre $\gets$ tNodo
				%\State *(*(pNodo).der).padre $\gets$ tNodo
				%\State delete pNodo
			%\EndIf
		%\While{b $!=$ NULL}
			%\State SetAltura(b)
			%\If{|}
		%\EndWhile
		%\EndIf	
	%\EndIf
	%
	%}{}{
	%}
	
\Algoritmo{ICIzq}{\In{a}{puntero(nodo)}, \In{b}{puntero(nodo)}}{}{
	\State *(a).izq = b \ComplejidadDer{1}
	\State *(b).padre = a \ComplejidadDer{1}
	}{\Complejidad{1}}{
	$\Complejidad{1} + \Complejidad{1} =\newline
	2 * \Complejidad{1} =\newline
	\Complejidad{1}$
	}
	
\Algoritmo{ICDer}{\In{a}{puntero(nodo)}, \In{b}{puntero(nodo)}}{}{
	\State *(a).der = b \ComplejidadDer{1}
	\State *(b).padre = a \ComplejidadDer{1}
	}{\Complejidad{1}}{
	$\Complejidad{1} + \Complejidad{1} =\newline
	2 * \Complejidad{1} =\newline
	\Complejidad{1}$
	}
	
\Algoritmo{ISetAltura}{\In{a}{puntero(nodo)}}{nat}{
	\If{*(a).izq == NULL} \ComplejidadDer{1}
		\If{*(a).der == NULL} \ComplejidadDer{1}
			\State res $\gets$ 1 \ComplejidadDer{1}
		\Else
			\State res $\gets$ 1 $+$ *(*(a).der).alt \ComplejidadDer{1}
		\EndIf
	\Else
		\If{*(a).der == NULL} \ComplejidadDer{1}
			\State res $\gets$ 1 $+$ *(*(a).izq).alt \ComplejidadDer{1}
		\Else
			\If{*(*(a).izq).alt $>$ *(*(a).der).alt} \ComplejidadDer{1}
				\State res $\gets$ 1 $+$ *(*(a).izq).alt \ComplejidadDer{1}
			\Else
				\State res $\gets$ 1 $+$ *(*(a).der).alt \ComplejidadDer{1}
			\EndIf
		\EndIf
	\EndIf
	}{\Complejidad{1}}{
	$\Complejidad{1} + max(\Complejidad{1} + max(\Complejidad{1}, \Complejidad{1}),
	\Complejidad{1} + max(\Complejidad{1}, \Complejidad{1} + max(\Complejidad{1}, \Complejidad{1}))) =
	\newline
	\Complejidad{1} + max(\Complejidad{1} + max(\Complejidad{1}, \Complejidad{1}),
	\Complejidad{1} + max(\Complejidad{1}, \Complejidad{1} + \Complejidad{1})) =\newline
	\Complejidad{1} + max(\Complejidad{1} + \Complejidad{1},
	\Complejidad{1} + max(\Complejidad{1}, 2 * \Complejidad{1})) =\newline
	\Complejidad{1} + max(2 * \Complejidad{1}, 3 * \Complejidad{1}) =\newline
	\Complejidad{1} + 3 * \Complejidad{1} =	4 * \Complejidad{1} = \Complejidad{1}$
	}
	
\Algoritmo{IFactorDesbalance}{\In{a}{puntero(nodo)}}{int}{
	\If{*(a).izq == NULL} \ComplejidadDer{1}
		\If{*(a).der == NULL} \ComplejidadDer{1}
			\State res $\gets$ 0 \ComplejidadDer{1}
		\Else
			\State res $\gets$ -(*(*(a).der).alt) \ComplejidadDer{1}
		\EndIf
	\Else
		\If{*(a).der == NULL} \ComplejidadDer{1}
			\State res $\gets$ *(*(a).izq).alt \ComplejidadDer{1}
		\Else
			\State res $\gets$ *(*(a).izq).alt - *(*(a).der).alt \ComplejidadDer{1}
		\EndIf
	\EndIf
	}{\Complejidad{1}}{
	$\Complejidad{1} + max(\Complejidad{1} + (max(\Complejidad{1}), max(\Complejidad{1})),
	\Complejidad{1} + (max(\Complejidad{1}), max(\Complejidad{1}))) =\newline
	\Complejidad{1} + max(\Complejidad{1} + \Complejidad{1},
	\Complejidad{1} + \Complejidad{1}) =\newline
	\Complejidad{1} + max(2 * \Complejidad{1}, 2 * \Complejidad{1}) =\newline
	\Complejidad{1} + 2 * \Complejidad{1} =\newline
	3 * \Complejidad{1} + \Complejidad{1}$
	}
	
\Algoritmo{IHijoMasAlto}{\In{a}{puntero(nodo)}}{puntero(nodo)}{
	\If{*(*(a).izq).alt > *(*(a).der).alt} \ComplejidadDer{1}
		\State res $\gets$ *(a).der \ComplejidadDer{1}
	\Else
		\State res $\gets$ *(a).izq \ComplejidadDer{1}
	\EndIf
	}{\Complejidad{1}}{
	$\Complejidad{1} + max(\Complejidad{1}, \Complejidad{1}) =\newline
	\Complejidad{1} + \Complejidad{1} =\newline
	2 * \Complejidad{1} = \Complejidad{1}$
	}

\Algoritmo{ITamaño}{\Inout{c}{colaP(Paquete)}}{nat}{
	\State res $\gets$ c.tam \ComplejidadDer{1}
	}{\Complejidad{1}}{
	}

\end{Algoritmos}

\end{document}
