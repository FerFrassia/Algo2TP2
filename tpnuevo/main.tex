\documentclass[a4paper,10pt]{article}
\usepackage[paper=a4paper, hmargin=1.5cm, bottom=1.5cm, top=3.5cm]{geometry}
\usepackage[latin1]{inputenc}
\usepackage[T1]{fontenc}
\usepackage[spanish]{babel}
\usepackage[framemethod=tikz]{mdframed}
\usepackage[T1]{fontenc}
\usepackage{xspace}
\usepackage{xargs}
\usepackage{ifthen}
\usepackage{fancyhdr}
\usepackage{lastpage}
\usepackage{aed2-tad,aed2-symb,aed2-itef}
\usepackage{enumitem}
\usepackage{algorithm}
\usepackage{algpseudocode}
\usepackage{scrextend}
\usepackage{framed}
%\usepackage[noend]{algpseudocode}

\newcommand{\moduloNombre}[1]{\textbf{#1}}

\let\NombreFuncion=\textsc
\let\TipoVariable=\texttt
\let\ModificadorArgumento=\textbf
\newcommand{\res}{$res$\xspace}
\newcommand{\tab}{\hspace*{7mm}}

\newcommandx{\TipoFuncion}[3]{%
  \NombreFuncion{#1}(#2) \ifx#3\empty\else $\to$ \res\,: \TipoVariable{#3}\fi%
}
\newcommand{\In}[2]{\ModificadorArgumento{in} \ensuremath{#1}\,: \TipoVariable{#2}\xspace}
\newcommand{\Out}[2]{\ModificadorArgumento{out} \ensuremath{#1}\,: \TipoVariable{#2}\xspace}
\newcommand{\Inout}[2]{\ModificadorArgumento{in/out} \ensuremath{#1}\,: \TipoVariable{#2}\xspace}
\newcommand{\Aplicar}[2]{\NombreFuncion{#1}(#2)}

\newlength{\IntFuncionLengthA}
\newlength{\IntFuncionLengthB}
\newlength{\IntFuncionLengthC}
%InterfazFuncion(nombre, argumentos, valor retorno, precondicion, postcondicion, complejidad, descripcion, aliasing)
\newcommandx{\InterfazFuncion}[9][4=true,6,7,8,9]{%
  \hangindent=\parindent
  \TipoFuncion{#1}{#2}{#3}\\%
  \textbf{Pre} $\equiv$ \{#4\}\\%
  \textbf{Post} $\equiv$ \{#5\}%
  \ifx#6\empty\else\\\textbf{Complejidad:} #6\fi%
  \ifx#7\empty\else\\\textbf{Descripci�n:} #7\fi%
  \ifx#8\empty\else\\\textbf{Aliasing:} #8\fi%
  \ifx#9\empty\else\\\textbf{Requiere:} #9\fi%
}

\newenvironment{Interfaz}{%
  \parskip=2ex%
  \noindent\textbf{\Large Interfaz}%
  \par%
}{}

\newenvironment{Representacion}{%
  \vspace*{2ex}%
  \noindent\textbf{\Large Representaci�n}%
  \vspace*{2ex}%
}{}

\newenvironment{Algoritmos}{%
  \vspace*{2ex}%
  \noindent\textbf{\Large Algoritmos}%
  \vspace*{2ex}%
}{}

\newcommand{\Titulo}[1]{
  \vspace*{1ex}\par\noindent\textbf{\large #1}\par
}

\newenvironmentx{Estructura}[2][2={estr}]{%
  \par\vspace*{2ex}%
  \TipoVariable{#1} \textbf{se representa con} \TipoVariable{#2}%
  \par\vspace*{1ex}%
}{%
  \par\vspace*{2ex}%
}%

\newboolean{EstructuraHayItems}
\newlength{\lenTupla}
\newenvironmentx{Tupla}[1][1={estr}]{%
    \settowidth{\lenTupla}{\hspace*{3mm}donde \TipoVariable{#1} es \TipoVariable{tupla}$($}%
    \addtolength{\lenTupla}{\parindent}%
    \hspace*{3mm}donde \TipoVariable{#1} es \TipoVariable{tupla}$($%
    \begin{minipage}[t]{\linewidth-\lenTupla}%
    \setboolean{EstructuraHayItems}{false}%
}{%
    $)$%
    \end{minipage}
}

\newenvironmentx{Enum}[1][1={estr}]{%
    \settowidth{\lenTupla}{\hspace*{3mm}donde \TipoVariable{#1} es \TipoVariable{enum}$($}%
    \addtolength{\lenTupla}{\parindent}%
    \hspace*{3mm}donde \TipoVariable{#1} es \TipoVariable{enum}$($%
    \begin{minipage}[t]{\linewidth-\lenTupla}%
    \setboolean{EstructuraHayItems}{false}%
}{%
    $)$%
    \end{minipage}
}

\newcommandx{\tupItem}[3][1={\ }]{%
    %\hspace*{3mm}%
    \ifthenelse{\boolean{EstructuraHayItems}}{%
        ,#1%
    }{}%
    \emph{#2}: \TipoVariable{#3}%
    \setboolean{EstructuraHayItems}{true}%
		\ignorespaces%
}

\newcommandx{\tupTupItem}[3][1={\ }]{%
    %\hspace*{3mm}%
    \ifthenelse{\boolean{EstructuraHayItems}}{%
       ,#1%
    }{}%
    \emph{#2}: #3%
    \setboolean{EstructuraHayItems}{true}%
		\ignorespaces%
}

\newcommandx{\enumItem}[2][1={\ }]{%
    %\hspace*{3mm}%
    \ifthenelse{\boolean{EstructuraHayItems}}{%
        ,#1%
    }{}%
    \TipoVariable{#2}%
    \setboolean{EstructuraHayItems}{true}%
		\ignorespaces%
}

\newcommandx{\RepFc}[3][1={estr},2={e}]{%
  \tadOperacion{Rep}{#1}{bool}{}%
  \tadAxioma{Rep($#2$)}{#3}%
}%

\newcommandx{\Rep}[3][1={estr},2={e}]{%
  \tadOperacion{Rep}{#1}{bool}{}%
  \tadAxioma{Rep($#2$)}{true \ssi #3}%
}%

\newcommandx{\Abs}[5][1={estr},3={e}]{%
  \tadOperacion{Abs}{#1/#3}{#2}{Rep($#3$)}%
  \settominwidth{\hangindent}{Abs($#3$) \igobs #4: #2 $\mid$ }%
  \addtolength{\hangindent}{\parindent}%
  Abs($#3$) \igobs #4: #2 $\mid$ #5%
}%

\newcommandx{\AbsFc}[4][1={estr},3={e}]{%
  \tadOperacion{Abs}{#1/#3}{#2}{Rep($#3$)}%
  \tadAxioma{Abs($#3$)}{#4}%
}%

\newcommand{\DRef}{\ensuremath{\rightarrow}}

\newcommandx{\ComplejidadDer}[1]{
	\hfill \ensuremath{\mathcal{O}(#1)}
}

\newcommandx{\Complejidad}[1]{
	\ensuremath{\mathcal{O}(#1)}
}

\renewcommand{\thealgorithm}{}

\newenvironmentx{Algoritmo}[6]{
	%\begin{algorithm}
	
	\begin{mdframed}
		
		%\floatname{algorithm}{Algoritmo}
		%\caption{\TipoFuncion{#1}{#2}{#3}}
		\TipoFuncion{#1}{#2}{#3}
		\begin{algorithmic}[1]
			#4
		\end{algorithmic}

		\begin{flushleft}
			%\begin{framed}
				%\flushleft
				\textbf{Complejidad:} #5 \\
				\vspace{0.75em}
					#6
			%\end{framed}
		\end{flushleft}
		
	\end{mdframed}
	
	%\end{algorithm}
}

\fancypagestyle{caratula} {
   \fancyhf{}
   \cfoot{\thepage /\pageref{LastPage}}
   \renewcommand{\headrulewidth}{0pt}
   \renewcommand{\footrulewidth}{0pt}
}

\begin{document}

%%%%%%%%%%%%%%%%%%%%%%%%%%%%%%%%%%%%%%%%%%%%%%%%%%%%%%%%%%%%%%%%%%%%%%%%%%%%%%%
%% Car�tula                                                                  %%
%%%%%%%%%%%%%%%%%%%%%%%%%%%%%%%%%%%%%%%%%%%%%%%%%%%%%%%%%%%%%%%%%%%%%%%%%%%%%%%

\thispagestyle{caratula}

\begin{center}

\vspace*{1cm}

\begin{Huge}
Algoritmos y Estructuras de Datos II
\end{Huge}

\vspace{1cm}

\begin{LARGE}
Trabajo Pr�ctico 2
\end{LARGE}

\vspace{1cm}

Departamento de Computaci�n,\\
Facultad de Ciencias Exactas y Naturales,\\
Universidad de Buenos Aires

\vspace{1cm}

Segundo Cuatrimestre de 2014

\vspace{1cm}

\begin{Large}
Grupo 16
\end{Large}

\vspace{0.5cm}

\begin{tabular}{|c|c|c|}
\hline
Apellido y Nombre & LU & E-mail\\
\hline
Juan Ernesto Rinaudo        & 864/13 & jangamesdev@hotmail.com\\
Mauro Cherubini             & 835/13 & cheru.mf@gmail.com\\
Federico Beuter             & 827/13 & federicobeuter@gmail.com\\
Fernando Frassia            & 340/13 & ferfrassia@gmail.com\\
\hline
\end{tabular}

\vspace{1cm}

Reservado para la c�tedra

\begin{tabular}{|c|c|c|}
\hline
Instancia & Docente que corrigi� & Calificaci�n\\
\hline
Primera Entrega &&\\
\hline
Recuperatorio   &&\\
\hline
\end{tabular}

\end{center}

\vspace{6cm}

\newpage


\tableofcontents


%%%%%%%%%%%%%%%%%%%%%%%%%%%%%%%%%%%%%%%%%%%%%%%%%%%%%%%%%%%%%%%%%%%%%%%%%%%%%%%%%%%%%%%%%%%%%%%%%%%%%
%
%
%
% ACA EMPIEZA EL CODIGO DEL LATEX
%
%
%
%%%%%%%%%%%%%%%%%%%%%%%%%%%%%%%%%%%%%%%%%%%%%%%%%%%%%%%%%%%%%%%%%%%%%%%%%%%%%%%%%%%%%%%%%%%%%%%%%%%%%

\newpage
\section{Tad Extendidos}	

\subsection{Secu($\alpha$)}

\vspace{2em}

\tadOtrasOperaciones
\tadOperacion{elemDeSecu}{Secu($\alpha$)/s,Nat/n}{RUR}{n $<$ long($s$)}

\vspace{1em}

\tadAxiomas
\tadAxioma{elemDeSecu(s, n)}{\IF $n =$ 0 THEN prim($s$) ELSE elemDeSecu(fin(s), n-1) FI}

\vspace{2em}

\subsection{Mapa}

\vspace{2em}

\tadObservadores
\tadOperacion{restricciones}{Mapa/m}{secu(restriccion)}{}
\tadOperacion{nroConexion}{estacion/e_1, estacion/e_2, Mapa/m}{nat}{{$e_1, e_2$} $\subset$ estaciones(m) $\yluego$ conectadas?($e_1, e_2, m$)}

\vspace{1em}

\tadAxiomas
\tadAxioma{restricciones(vacio)}{$\langle$~$\rangle$}
\tadAxioma{restricciones(agregar($e, m$))}{restricciones($m$)}
\tadAxioma{restricciones(conectar($e_1, e_2, r, m$))}{restricciones($m$) $\circ$ $r$}
\tadAxioma{nroConexion($e_1, e_2, $conectar($e_3, e_4, m$))}{
	\IF ($(e_1 = e_3 \wedge e_2 = e_4) \vee (e_1 = e_4 \wedge e_2 = e_3)$) THEN long(restricciones($m$)) - 1 ELSE nroConexion($e_1, e_2, m$) - 1 FI
}
\tadAxioma{nroConexion($e_1, e_2, $agregar($e, m$))}{nroConexion($e_1, e_2, m$)}

\vspace{2em}

\newpage
\section{Mapa}

\begin{Interfaz}
  
  %\textbf{par�metros formales}\hangindent=2\parindent\\
  %\parbox{1.7cm}{\textbf{g�neros}} \\
  %\parbox[t]{1.7cm}{\textbf{funci�n}}\parbox[t]{\textwidth-2\parindent-1.7cm}{%
    %\InterfazFuncion{Verifica?}{\In{c}{conj(tag)}, \In{r}{rest}}{$\alpha$}
    %{$res \igobs a$}
    %[$\Theta(copy(a))$]
    %[funci�n de copia de $\alpha$'s]
  %}

  \textbf{se explica con}: \tadNombre{Red, Iterador Unidireccional($\alpha$)}.

  \textbf{g�neros}: \TipoVariable{red, itConj(Compu)}.

  \Titulo{Operaciones b�sicas de Red}

  %\InterfazFuncion{NOMBRE}{INPUTS}{TIPO RES}%
  %[ACA VA EL PRE (SI LO HAY)]
  %{ACA VA EL POST}%
  %[$\Theta(COMPLEJIDAD)$]
  %[DESCRIPCION]

  \InterfazFuncion{Computadoras}{\In{r}{red}}{itConj(Compu)}
  %[ACA VA EL PRE (SI LO HUBIERA)]
  {$res \igobs$ crearIt(computadoras($r$))}
  [$\Complejidad{1}$]
  [Devuelve las computadoras de red.]
	
  \InterfazFuncion{Conectadas?}{\In{r}{red}, \In{c_1}{compu}, \In{c_2}{compu}}{bool}
  [\{$c_1,c_2$\} $\subseteq$ computadoras($r$)]
  {$res \igobs$ conectadas?($r, c_1, c_2$)}
  [$\Complejidad{|c_1| + |c_2|}$]
  [Devuelve el valor de verdad indicado por la conexi�n o desconexi�n de dos computadoras.]

  \InterfazFuncion{InterfazUsada}{\In{r}{red}, \In{c_1}{compu}, \In{c_2}{compu}}{interfaz}
  [\{$c_1, c_2$\} $\subseteq$ computadoras($r$) $\yluego$ conectadas?($r, c_1, c_2$)]
  {$res \igobs$ interfazUsada($r, c_1, c_2$)}
  [$\Complejidad{|c_1| + |c_2|}$]
  [Devuelve la interfaz que $c_1$ usa para conectarse con $c_2$]

  \InterfazFuncion{IniciarRed}{}{red}
  %[ACA VA EL PRE (SI LO HAY)]
  {$res \igobs$ iniciarRed()}
  [$\Complejidad{1}$]
  [Crea una red sin computadoras.]
	
  \InterfazFuncion{AgregarComputadora}{\Inout{r}{red}, \In{c}{compu}}{}
  [$r_0 \igobs r$ $\wedge$ $\neg$($c$ $\in$computadoras($r$))]
  {$r$ $\igobs$ agregarComputadora($r_0, c$)}
  [$\Complejidad{|c|}$]
  [Agrega una computadora a la red.]

  \InterfazFuncion{Conectar}{\Inout{r}{red}, \In{c_1}{compu}, \In{i_1}{interfaz}, \In{c_2}{compu}, \In{i_2}{interfaz}}{}
  [$r_0$ $\igobs$ r $\wedge$ \{$c_1,c_2$\} $\subseteq$ computadoras($r$) $\wedge$ ip($c_1$) $\neq$ ip($c_2$) $\yluego$ $\neg$ conectadas?($r, c_1, c_2$) $\wedge$ $\neg$ usaInterfaz?($r, c_1, i_1$) $\wedge$ $\neg$ usaInterfaz?($r, c_2, i_2$)]
  {$r$ $\igobs$ conectar($r, c_1, i_1, c_2, i_2$)}
  [$\Complejidad{|c_1| + |c_2|}$]
  [Conecta dos computadoras y les a�ade la interfaz correspondiente.]
  
  %\InterfazFuncion{}
%
 % \InterfazFuncion{NroConexion}{\In{e_1}{estaci�n}, \In{e_2}{estaci�n}, \In{m}{mapa}}{nat}
  %[\{$e_1,e_2$\} $\subset$ estaciones($m$) $\yluego$ conectadas?($e_1, e_2, m$)]
  %{$res \igobs$ nroConexion($e_1$, $e_2$, $m$)}
  %[$\Complejidad{|e_1| + |e_2|}$]
  %[Obtiene el Nro. de Senda entre dos estaciones.]
	
  %\InterfazFuncion{EvaluarSendas}{\In{c}{conj(tag)}, \In{m}{mapa}}{arreglo\_dimesionable de bool}
  %[\{e1,e2\} $\in$ estaciones(m) $\yluego$ $res \igobs$ conectadas?(e1, e2, m)]
  %{$(\forall i$: Nat) (0 $\leq i <$ long(restricciones($m$)) $\impluego$ res[$i$] $\igobs$ verifica?($c$, elemDeSecu(restricciones($m$), $i$)))}
  %[$\Complejidad{S \cdot R}$]
  %[Devuelve un arreglo con todas las sendas evaluadas con respecto al conjunto $c$.
	%NOTA: La $R$ es el costo de evaluar la restriccion mas grande.]
	
%	\InterfazFuncion{Restricciones}{\In{m}{mapa}}{arreglo\_dimesionable de restriccion}
  %[\{e1,e2\} $\in$ estaciones(m) $\yluego$ $res \igobs$ conectadas?(e1, e2, m)]
 % {$(\forall i$: Nat) (0 $\leq i <$ long(restricciones($m$)) $\impluego$ res[$i$] $\igobs$ elemDeSecu(restricciones($m$), $i$)}
  %[$\Complejidad{S}$]
  %[Devuelve un arreglo con todas las restricciones]
  
\end{Interfaz}

\subsection{Representaciorepresentacionn}

\begin{Representacion}

%ACA VA LA DESCRIPCION DE LA CACONA DEL MAPA

\bigskip
\begin{Estructura}{red}[e\_red]
	\begin{Tupla}[e\_red]
		\tupTupItem{vecinosEInterfaces}{\TipoVariable{diccString}$($\ignorespaces
			% NOTA:tupTupItem es un engendro que agregue para este caso, NO USAR EN OTRO LADO, USAR tupItem EN SU LUGAR.
			\emph{compu}: \TipoVariable{string},
			\TipoVariable{diccString}$($\ignorespaces
				\emph{compu}: \TipoVariable{string}, \ignorespaces
				\emph{interfaz}: \TipoVariable{nat}\ignorespaces
			$))$} \\
			\tupTupItem{deOrigenADestino}{\TipoVariable{diccString}$($\ignorespaces
			\emph{compu}: \TipoVariable{string},
			\TipoVariable{diccString}$($\ignorespaces
				\emph{compu}: \TipoVariable{string}, \ignorespaces
				\emph{secu(compu)}: \TipoVariable{secu(string)}\ignorespaces
			$))$} \\
			\tupItem{computadoras}{\TipoVariable{conj(compu)}}
	\end{Tupla}
	
\end{Estructura}

%\begin{Estructura}{mapa}[e\_mapa]
	%\begin{Tupla}[e\_mapa]
		%\tupTupItem{unionesDe}{DiccString(\TipoVariable{conj}$($\ignorespaces
			%% NOTA:tupTupItem es un engendro que agregue para este caso, NO USAR EN OTRO LADO, USAR tupItem EN SU LUGAR.
			%\TipoVariable{tupla}$($\ignorespaces
				%\emph{con}: \TipoVariable{estacion},
				%\emph{res}: \TipoVariable{restricci�n}\ignorespaces
			%$)))$}
		%\tupItem{\\estaciones}{conj$($estacion$)$}
	%\end{Tupla}
	%
%\end{Estructura}

\subsection{InvRep y Abs}

\begin{enumerate}
	\item{El conjunto de claves de ''uniones'' es igual al conjunto de estaciones ''estaciones''.}
	\item{''\#sendas'' es igual a la mitad de las horas de ''uniones''.}
	\item{Todo valor que se obtiene de buscar el significado del significado de cada clave de ''uniones'', es igual el valor hallado tras buscar en ''uniones'' con el sinificado de la clave como clave y la clave como significado de esta nueva clave, y no hay otras hojas ademas de estas dos, con el mismo valor.}
	\item{Todas las hojas de ''uniones'' son mayores o iguales a cero y menores a ''\#sendas''.}
	\item{La longitud de ''sendas'' es mayor o igual a ''\#sendas''.}
\end{enumerate}

\Rep[e\_mapa][m]{
	\\m.estaciones = claves(m.uniones) $\wedge$ \hfill 1.
	\\m.\#sendas = \#sendasPorDos(m.estaciones, m.uniones) / 2 $\wedge$ m.\#sendas $\leq$ long(m.sendas) $\yluego$ \hfill 2. 5.
	\\($\forall$ e1, e2: string)(e1 $\in$ claves(m.uniones) $\yluego$ e2 
	$\in$ claves(obtener(e1, m.uniones)) $\impluego$\\ 
	e2 $\in$ claves(m.uniones) $\yluego$ e1 $\in$ claves(obtener(e2, m.uniones)) $\yluego$ 
	\\ obtener(e2, obtener(e1, m.uniones)) = obtener(e1, obtener(e2, m.uniones)) $\wedge$ \hfill 3. 4.	
	\\ obtener(e2, obtener(e1, m.uniones)) $<$ m.\#sendas) $\wedge$
	\\($\forall$ e1, e2, e3, e4: string)((e1 $\in$ claves(m.uniones) $\yluego$
	e2 $\in$ claves(obtener(e1, m.uniones)) $\wedge$\\ 
	e3 $\in$ claves(m.uniones) $\yluego$ e4 $\in$ claves(obtener(e3, m.uniones))) $\impluego$
	\\ (obtener(e2, obtener(e1, m.uniones)) $=$ obtener(e4, obtener(e3, m.uniones)) $\ssi$
	\\ (e1 = e3 $\wedge$ e2 = e4) $\vee$ (e1 = e4 $\wedge$ e2 = e3)))) \hfill 3.
}

\vspace{2em}
	
\tadOperacion{\#sendasPorDos}{conj($\alpha$)\ c, dicc($\alpha$, dicc($\alpha$, $\beta$))\ d}
														{nat}{c $\subset$ claves(d)}

\vspace{1em}

\tadAxioma{\#sendasPorDos(c, d)}{\IF $\emptyset$?(c) THEN 0
												ELSE \#claves(obtener(dameUno(c),d)) $+$ \#sendasPorDos(sinUno(c), d)
												FI}

\vspace{2em}

\Abs[e\_mapa]{mapa}[m]{p}{
	\\ m.estaciones = estaciones(p) $\yluego$
	\\ ($\forall$ e1, e2: string)((e1 $\in$ estaciones(p) $\wedge$ e2 $\in$ estaciones(p)) $\impluego$
	\\ (conectadas?(e1, e2, p) $\ssi$
	\\   e1 $\in$ claves(m.uniones) $\wedge$ e2 $\in$ claves(obtener(e2, m.uniones)))) $\yluego$
	\\ ($\forall$ e1, e2: string)((e1 $\in$ estaciones(p) $\wedge$ e2 $\in$ estaciones(p)) $\yluego$
	\\ conectadas?(e1, e2, p) $\impluego$ 
	\\ (restriccion(e1, e2, p) = m.sendas[obtener(e2, obtener(e1, m.uniones))] $\wedge$
	   nroConexion(e1, e2, m) = obtener(e2, obtener(e1, m.uniones))) $\wedge$
	   long(restricciones(p)) = m.\#sendas $\yluego$
		 ($\forall$ n:nat) (n < m.\#sendas $\impluego$ m.sendas[n] = ElemDeSecu(restricciones(p), n)))
}

\end{Representacion}

\subsection{Algoritmos}

\begin{Algoritmos}
	
	%Algoritmos / Inputs / TSalida / Codigo / Complejidad Final / Justificacion
	\begin{Algoritmo}{iComputadoras}{\In{r}{red}}{itConj(Compu)}
	{
		\State $res$ $\gets$ CrearIt($r.computadoras$) \ComplejidadDer{1}
	} 
	{$\Complejidad{1}$}{}
	\end{Algoritmo}
	
	\begin{Algoritmo}{iConectadas?}{\In{r}{red}, \In{c_1}{compu}, \In{c_2}{compu}}{bool}
	{
		\State $res$ $\gets$ Definido?(Significado($r.vecinosEInterfaces$, $c_1$), $c_2$) \ComplejidadDer{|c_1| + |c_2|}
	}
	{$\Complejidad{|c_1| + |c_2|}$}{}
	\end{Algoritmo}
	
	\begin{Algoritmo}{iInterfazUsada}{\In{r}{red}, \In{c_1}{compu}, \In{c_2}{compu}}{interfaz}
	{
		\State $res$ $\gets$ Significado(Significado($r.vecinosEInterfaces$, $c_1$), $c_2$)\ComplejidadDer{|c_1| + |c_2|}
	}
	{$\Complejidad{|c_1| + |c_2|}$}{}
	\end{Algoritmo}
	
	\begin{Algoritmo}{iIniciarRed}{}{red}
	{
		\State $res$ $\gets$ tupla($vecinosEInterfaces$: Vac�o(), $deOrigenADestino$: Vac�o(), $computadoras$: Vac�o()) \ComplejidadDer{1 + 1 + 1}
	}
	{$\Complejidad{1}$}
	{$\Complejidad{1}  + \Complejidad{1} + \Complejidad{1} = \newline
	  3 * \Complejidad{1} = \Complejidad{1}$}
	\end{Algoritmo}
	
	\begin{Algoritmo}{iAgregarComputadora}{\Inout{r}{red}, \In{c}{compu}}{}
	{
		\State Agregar($r.computadoras$, $c$) \ComplejidadDer{1}
		\State Definir($r.vecinosEInterfaces$, $c$, Vac�o()) \ComplejidadDer{|c|}
		\State Definir($r.deOrigenADestino$, $c$, Vac�o()) \ComplejidadDer{|c|}
	}
	{$\Complejidad{|c|}$}
	{$\Complejidad{1} + \Complejidad{|c|} + \Complejidad{|c|} = \newline
	2 * \Complejidad{|c|} = \Complejidad{|c|}$}
	\end{Algoritmo}
	
	\begin{Algoritmo}{iConectar}{\Inout{r}{red}, \In{c_1}{compu}, \In{i_1}{interfaz}, \In{c_2}{compu}, \In{i_2}{interfaz}}{}
	{
		%\State Definir(Significado($m.uniones$, $e1$), $e2$, $m.\#senda$) \ComplejidadDer{|e_1| + |e_2|}
		%\State Definir(Significado($m.uniones$, $e2$), $e1$, $m.\#senda$) \ComplejidadDer{|e_2| + |e_1|}
		%\State Agregar($m.sendas$, $m.\#sendas$, $r$) \ComplejidadDer{1}
		%\State $m.\#sendas++$ \ComplejidadDer{1}
	
		\State Definir(Significado($r.vecinosEInterfaces$, $c_1$), $c_2$, $i_1$) \ComplejidadDer{|c_1| + |c_2| + 1}
		\State Definir(Significado($r.vecinosEInterfaces$, $c_2$), $c_1$, $i_2$) \ComplejidadDer{|c_2| + |c_1| + 1} 
		\State 
	}
	{$\Complejidad{|e_1| + |e_2|}$}
	{$\Complejidad{|e_1| + |e_2|} + \Complejidad{|e_1| + |e_2|} + \Complejidad{1} + \Complejidad{1} = \newline
	  2 * \Complejidad{1} + 2 * \Complejidad{|e_1| + |e_2|} = \newline
		2 * \Complejidad{|e_1| + |e_2|} = \Complejidad{|e_1| + |e_2|}$}
	\end{Algoritmo}
	
	%\begin{Algoritmo}{iNroConexion}{\In{e1}{estaci�n}, \In{e2}{estaci�n}, \In{m}{mapa}}{nat}
	%{
	%	\State $res$ $\gets$ Significado(Significado($m.uniones$, $e1$), $e2$) \ComplejidadDer{|e_1| + |e_2|}
	%}
	%{$\Complejidad{|e_1| + |e_2|}$}{}
	%\end{Algoritmo}
	%
	%\begin{Algoritmo}{iEvaluarSendas}{\In{c}{conj(tag)}, \In{m}{mapa}}{arreglo\_dimesionable de bool}
	%{
	%	\State $res$ $\gets$ arreglo[$m.\#sendas$] de bool \ComplejidadDer{1}
	%	\State var $i$: nat $\gets$ 0 \ComplejidadDer{1}
	%	\While{$i < m.\#sendas$} \ComplejidadDer{1}
	%		\State $res[i]$ $\gets$ Verifica?($c$, $m.sendas[i]$) \ComplejidadDer{R}
	%		\State $i++$ \ComplejidadDer{1}
	%	\EndWhile
	%}
	%{$\Complejidad{S * R}$}
	%{$\Complejidad{1} + \Complejidad{1} + \sum_{i = 1}^{S} (\Complejidad{R} + \Complejidad{1}) = \newline
	%	2 * \Complejidad{1} + S * (\Complejidad{R} + \Complejidad{1}) = \newline
	%	2 * \Complejidad{1} + S * \Complejidad{1} + S * \Complejidad{R} = \newline
	%	\Complejidad{S} + S * \Complejidad{R} = \newline
	%	\Complejidad{S + S * R} = \Complejidad{S * R}
	%$\newline
	%$S$ es |m.sendas| y $R$ es la longitud de la restriccion mas grande}
	%\end{Algoritmo}
	
	%\begin{Algoritmo}{iRestricciones}{\In{m}{mapa}}{arreglo\_dimesionable de restriccion}
	%{
	%	\State $res$ $\gets$ arreglo\_dimensionable[$m.\#sendas$] \ComplejidadDer{1}
	%	\State var $i$: nat $\gets$ 0 \ComplejidadDer{1}
	%	\While{$i < m.\#sendas$} \ComplejidadDer{1}
	%		\State $res[i]$ $\gets$ $m.sendas[i]$ \ComplejidadDer{1}
	%		\State $i++$ \ComplejidadDer{1}
	%	\EndWhile
	%}
	%{$\Complejidad{S}$}
	%{$\Complejidad{1} + \Complejidad{1} + \sum_{i = 1}^{S} (\Complejidad{1} + \Complejidad{1}) = \newline
	%	2 * \Complejidad{1} + S * 2 *\Complejidad{1} = \newline
	%	2 * \Complejidad{1} + 2 * \Complejidad{S} = \newline
	%	2 * \Complejidad{S} = \Complejidad{S}
	% $}
	%\end{Algoritmo}
	%
\end{Algoritmos}

\newpage
\section{DCNet}

\subsection{Interfaz}

\begin{Interfaz}
  
  %\textbf{par�metros formales}\hangindent=2\parindent\\
  %\parbox{1.7cm}{\textbf{g�neros}} \\
  %\parbox[t]{1.7cm}{\textbf{funci�n}}\parbox[t]{\textwidth-2\parindent-1.7cm}{%
    %\InterfazFuncion{Verifica?}{\In{c}{conj(tag)}, \In{r}{rest}}{$\alpha$}
    %{$res \igobs a$}
    %[$\Theta(copy(a))$]
    %[funci�n de copia de $\alpha$'s]
  %}

  \textbf{se explica con}: \tadNombre{DCNet, Iterador Unidireccional($\alpha$)}.

  \textbf{g�neros}: \TipoVariable{dcnet}.

  \Titulo{Operaciones b�sicas de DCNet}
	

  \InterfazFuncion{Red}{\In {d}{dcnet}}{red}
  %[ACA VA EL PRE (SI LO HUBIERA)]
  {$res \igobs$ red($d$)}
  [\Complejidad{1}]
  [Devuelve la red del dcnet.]
  
  \InterfazFuncion{CaminoRecorrido}{\In {d}{dcnet}, \In{p}{paquete} }{secu(compu)}
  [$p$ $\in$ paqueteEnTransito?($d,p$)]
  {$res \igobs$ caminoRecorrido($d,p$))}
  [\Complejidad{n*log_2 (k)}]
  [Devuelve una secuencia con las computadoras por las que paso el paquete.]
  
  \InterfazFuncion{cantidadEnviados}{\In {d}{dcnet}, \In {c}{compu}}{nat}
  [$c$ $\in$ computadoras(red($d$))]
  {$res \igobs$ cantidadEnviados($d,c$)}
  [\Complejidad{|$c$.id|}]
  [Devuelve la cantidad de paquetes que fueron enviados desde la computadora.]
  
  \InterfazFuncion{enEspera}{\In {d}{dcnet}, \In {c}{compu}}{itPaquete}
  [$c$ $\in$ computadoras(red($d$))]
  {$res \igobs$ enEspera($d,c$)}
  [\Complejidad{|$c$.id|}]
  [Devuelve los paquetes que se encuentran en ese momento en la computadora.]
  
  \InterfazFuncion{IniciarDCNet}{\In {r}{red}}{dcnet}
  %[ACA VA EL PRE (SI LO HUBIERA)]
  {$res \igobs$ iniciarDCNet($r$)}
  [\Complejidad{N*L}]
  [Inicia un dcnet con la red y sin paquetes.]

  \InterfazFuncion{CrearPaquete}{\In {p}{paquete}, \Inout {d}{dcnet}}{}
  [$d_0 \equiv d$ $\wedge$ $\neg$ (($\exists$ $p_1$: paquete)(paqueteEnTransito($s,p_1$) $\wedge$ id($p_1$) $=$ id($p$)) $\wedge$ origen($p$) $\in$ computadoras(red($d$))$\yluego$ destino($p$) $\in$ computadoras(red($d$))$\yluego$ hayCamino?(red($d$,origen($p$),destino($p$))  ]
  {$res \igobs$ iniciarDCNet($r$)}
  [\Complejidad{L+log_2(k)}]
  [Agrega el paquete al dcnet.]
  
  \InterfazFuncion{AvanzarSegundo}{\Inout {d}{dcnet}}{}
  [$d_0 \equiv d$ ]
  {$d$ $\igobs$ avanzarSegundo($c_0$)}
  [\Complejidad{N*(L+log_2(k)}]
  [El paquete de mayor prioridad de cada computadora avanza a su proxima computadora siendo esta la del camino mas corto.]
  
  
  \InterfazFuncion{PaqueteEnTransito?}{\In {d}{dcnet}, \In {p}{paquete}}{bool}
  [ ]
  {$res$ $\igobs$ paqueteEnTransito?($d$,$p$)}
  [\Complejidad{N*log_2(k)}]
  [Devuelve si el paquete esta o no en alguna computadora del sistema.]
  
  
  \InterfazFuncion{LaQueMasEnvio}{\In {d}{dcnet)}}{compu}
  []
  {$res$ $\igobs$ laQueMasEnvio($d$)}
  [\Complejidad{1}]
  [Devuelve la computadora que mas paquetes envio.]
	
	\Titulo{Operaciones del iterador}
  
\end{Interfaz}

\subsection{Representacion}

\begin{Representacion}

%DESCRIPCION

\bigskip
\begin{Estructura}{dcnet}[e\_dc]
	\begin{Tupla}[e\_dc]
		\tupItem{red}{red}
		\tupTupItem{\\MasEnviante}{\TipoVariable{tupla}$($\ignorespaces
			\emph{compu}: \TipoVariable{compu}, 
			\emph{enviados}: \TipoVariable{nat}$)$}
		\tupTupItem{\\CompYPaq}{\TipoVariable{DiccString}$($\ignorespaces
			\emph{compu}: \TipoVariable{compu},
			\TipoVariable{tupla}$($\ignorespaces
			\emph{MasPriori}:{DiccRapido}$($\ignorespaces
			\emph{prioridad} :{nat},
			\emph{PaqdePriori}:{conj$($paquete$)$}
			 $)$,
			\emph{PaqYCam}:{DiccRapido} $($\ignorespaces
			\emph{paq}:{paquete},
			\emph{CamRecorrido}:{secu$($compu$)$
			}
			}
	
	\end{Tupla}
\end{Estructura}

\subsection{InvRep y Abs}

\begin{enumerate}
 \item{El conjunto de estaciones de 'mapa' es igual al conjunto con todas las claves de 'RURenEst'.}
 \item{La longitud de 'RURs' es mayor o igual a '\#RURHistoricos'.}
 \item{Todos los elementos de 'RURs' cumplen que su primer componente ('id') corresponde con su posicion en 'RURs'. 
       Su Componente 'e' es una de las estaciones de 'mapa', su componente 'esta?' es true si y solo si hay estaciones tales que su 
       valor asignado en 'uniones' es igual a su indice en 'RURs'. Su Componente 'inf' puede ser mayor a cero solamente si hay algun elemento en 
       'sendEv' tal que sea false. Cada elemento de 'sendEv' es igual a verificar 'carac' con la estriccion obtenida al buscar el 
       elemento con la misma posicion en la secuencia de restricciones de 'mapa'.}
 \item{Cada valor contenido en la cola del significado de cada estacion de las claves de 'uniones' pertenecen unicamente a la cola asociada a dicha estacion y a ninguna otra de las colas asociadas a otras estaciones. Y cada uno de estos valores es menor a '\#RURHistoricos' y mayor o igual a cero. Ademas la componente 'e' del elemento de la posicion igual a cada valor de las colas asociadas a cada estacion, es igual a la estacion asociada a la cola a la que pertenece el valor.}

\end{enumerate}


\Rep[e\_cr][c]{
	claves(c.RURenEst) $=$ estaciones(c.mapa) $\wedge$ \hfill1 \newline
	\#RURHistoricos $\leq$ Long(c.RURs) $\yluego$
	($\forall$ i:Nat, t:<id:Nat, esta?:Bool, e:String, \hfill2 \newline
	inf:Nat, carac:Conj(Tag), sendEv: ad(Bool)>) \newline
	(i<\#RURHistoricos $\yluego$ ElemDeSecu(c.RURs, i) = t $\impluego$ (t.e $\in$ estaciones(c.mapa) \hfill3 \newline
	$\wedge$ t.id $=$ i $\wedge$ tam(t.sendEv) $=$ long(Restricciones(c.mapa)) $\wedge$ \newline
	(t.inf > 0 $\Rightarrow$ ($\exists$ j:Nat) (j < tam(t.sendEv) $\yluego$ $\neg$ (t.sendEv[j]))) $\wedge$ \newline
	(t.esta? $\Leftrightarrow$ ($\exists$ e1: String) (e1 $\in$ claves(c.RUREnEst) $\yluego$ estaEnColaP?(obtener(e1, c.RUREnEst), t.id))) \newline
	$\wedge$ ($\forall$ h : Nat) (h < tam(t.sendEv) $\impluego$ \newline
	t.sendEv[h] = verifica?(t.carac, ElemDeSecu(Restricciones(c.mapa), h))))) $\yluego$ \newline
	($\forall$ e1, e2: String)(e1 $\in$ claves(c.RUREnEst) $\wedge$ e2 $\in$ claves(c.RUREnEst) $\wedge$ e1 $\neq$ e2 $\impluego$ \hfill4 \newline
	($\forall$ n:Nat)(estaEnColaP?(obtener(e1, c.RUREnEst), n) $\Rightarrow$ $\neg$ estaEnColaP?(obtener(e2, c.RUREnEst), n) $\wedge$
	n < \#RURHistoricos $\yluego$ ElemDeSecu(c.RURs, n).e $=$ e1))
	}
	
	\vspace{3em}
	
	\tadOperacion{estaEnColaP?}{ColaPri, Nat}{Bool}{}
	\vspace{1em}
	\tadAxioma{estaEnColaP?(cp, n)}{
	\IF vacia?(cp) 
	THEN false 
	ELSE 
	      {\IF desencolar(cp) = n 
	      THEN true 
	      ELSE estaEnColaP?(Eliminar(cp, desencolar(cp)), n) 
	      FI}
	FI}
	
	\vspace{3em}
	
	\Abs[e\_cr]{ciudad}[c]{u}{
		\\ c.\#RURHistoricos = ProximoRUR(U) $\wedge$ c.mapa = mapa(u) $\yluego$
		\\ robots(u) = RURQueEstan(c.RURs) $\yluego$ 
		\\ ($\forall$ n:Nat) (n $\in$ robots(u) $\impluego$ estacion(n,u) = c.RURs[n].e $\wedge$
		\\ tags(n,u) = c.RURs[n].carac $\wedge$ \#infracciones(n,u) = c.RURs[n].inf)}
		
	\vspace{3em}
	\tadOperacion{RURQueEstan}{secu(tupla)}{Conj(RUR)}{}
	\vspace{1em}
	tupla es <id:Nat, esta?:Bool, inf:Nat, carac:Conj(tag), sendEv:arreglo dimensionable(bool)>
	\vspace{1em}
	\tadAxioma{RURQueEstan(s)}{
	\IF vacia?(s)
	THEN $\emptyset$
	ELSE 
	      {\IF $\Pi_2$(prim(fin(s)))
	      THEN {$\Pi_1$(prim(fin(s)))} $\cup$ RURQueEstan(fin(s))
	      ELSE RURQueEstan(fin(s))
	      FI}
	FI}
	
\vspace{3em}

\begin{Estructura}{it}[e\_it]
	\begin{Tupla}[e\_it]
		\tupItem{i}{nat}
		\tupItem{maxI}{nat}
		\tupItem{ciudad}{puntero(ciudad)}
	\end{Tupla}
\end{Estructura}

\Rep[e\_it][it]{
	it.i $\leq$ it.maxI $\wedge$ maxI $=$ ciudad.\#RURHistoricos
}

\Abs[e\_it]{itUni($\alpha$)}[u]{it}{
	(HayMas?(u) $\yluego$ Actual(u) = ciudad.RURs[it.i] $\wedge$ Siguientes(u, $\emptyset$) = VSiguientes(ciudad, it.i++, $\emptyset$) $\vee$ ($\neg$HayMas?(u))
}

\vspace{2em}
	
\tadOperacion{Siguientes}{itUni u, conj(RURs) cr}{conj(RURs)}{}
														
\vspace{1em}

\tadAxioma{Siguientes(u, cr)}{\IF HayMas(u)? THEN Ag(Actual(Avanzar(u)), Siguientes(Avanzar(u), cr)) ELSE Ag($\emptyset$, cr) FI}
		
\vspace{1em}

\tadOperacion{VSiguientes}{ciudad c, Nat i, conj(RURs) cr}{conj(RURs)}{}
														
\vspace{1em}

\tadAxioma{VSiguientes(u, i, cr)}{\IF i < c.\#RURHistoricos THEN Ag(c.RURs[i], VSiguientes(u, i++, cr))) ELSE Ag($\emptyset$, cr) FI}

\vspace{2em}

\end{Representacion}

%tupla<mapa:mapa, robEnEst: dicctrie(estacion, conj(puntero(RUR))), robots: conj(RUR)>

\subsection{Algoritmos}

\begin{Algoritmos}




	\begin{Algoritmo}{iRed}{\In {d}{dcnet}}{red}
	{
		\State $res$ $\gets$ ($d$.red) \ComplejidadDer{1}
	}
	{\Complejidad{1}}
	{}
	
	\end{Algoritmo}
	
	
	
	
	\begin{Algoritmo}{iCaminoRecorrido}{\In {d}{dcnet},\In {p}{paquete}}{secu(compu)}
	{
		\State var $it$ $\gets$ \NombreFuncion {computadoras}(d.red) \ComplejidadDer{1}
		\State var $esta$: bool $\gets$ false
		\While {\NombreFuncion {HaySiguiente}($it$) $\wedge$ $\neg$ esta}   \ComplejidadDer{n}
			\State var:diccRapido $diccpaq$ $\gets$  \NombreFuncion {obtener}(\NombreFuncion{Siguiente}($it$).id, $d$.CompYPaq).PaqYCam)	\ComplejidadDer{L}
			\If {\NombreFuncion {def?}($p$,$diccpaq$)}	\ComplejidadDer{L+ log_2(N)}
				\State $esta$ $\gets$ true \ComplejidadDer{1}
				\State $res$ $\gets$ \NombreFuncion {obtener}($p$, $diccpaq$).CamRecorrido	\ComplejidadDer{L + log_2(N)+ 1} 		
			\EndIf
			\State \NombreFuncion {Avanzar}($it$)  \ComplejidadDer{1}
		\EndWhile
	}
	{\Complejidad{ }}
	{}
	\end{Algoritmo}
	
	
	
	
	
	\begin{Algoritmo}{iCantidadEnviados}{\In {d}{dcnet}, \In {c}{compu}}{nat}
	{
		\State $res$ $\gets$ \NombreFuncion {obtener}($c$.id,$d$.CompYPaq).Enviados  \ComplejidadDer{L}
	}
	{\Complejidad{L}}
	{Siendo L la longitud de el ID de $c$}
	\end{Algoritmo}
	
	
	
	
	\begin{Algoritmo}{iEnEspera}{\In {d}{dcnet}, \In {c}{compu}}{itPaquete)}
	{
		\State $res$  $\gets$ \NombreFuncion {claves}(\NombreFuncion {obtener}($c$.id, $d$.CompYPaq).PaqYCam) \ComplejidadDer{L}
	}
	{\Complejidad{|L|}}
	{Siendo L la longitud  del ID de $c$}
	\end{Algoritmo}
	
	
	
	
	
	
	\begin{Algoritmo}{iIniciarDCNet}{\In {r}{red}, \Inout {d}{dcnet}}{}
	{
		\State $d$.red $\gets$ r \ComplejidadDer{NOSE}
		\State  var $it$ $\gets$ \NombreFuncion {computadoras}(red) \ComplejidadDer{1}
		\State $d$.MasEnviante $\gets$ tupla(\NombreFuncion {Siguiente}($it$),0) \ComplejidadDer{1}
		\State $d$.CompyPaq $\gets$ Vacio() \ComplejidadDer{1}
		\While {\NombreFuncion{HaySiguiente}($it$)}  \ComplejidadDer{N}
			\State \NombreFuncion {Definir}(\NombreFuncion {Siguiente}($it$).id, tupla(\NombreFuncion {Vacio}(),\NombreFuncion {Vacio}(), 0), $d$.CompyPaq) \ComplejidadDer{L + 1 + 1}
			\State \NombreFuncion {Avanzar}($it$) \ComplejidadDer{1}
		\EndWhile
		
	}
	{\Complejidad{N*L}}
	{Siendo N la cantidad de computadoras en la red y L el ID mas largo de ellas.}
	\end{Algoritmo}
	
	
	
	
	
	
	\begin{Algoritmo}{iCrearPaquete}{\In {p}{paquete}, \Inout {d}{dcnet}}{}
	{
	
		\State var $diccprio$: diccRapido $\gets$ \NombreFuncion {obtener}($p$.origen, $d$.CompYPaq).MasPriori)	\ComplejidadDer{L}
		\State var $dicccam$: diccRapido $\gets$ \NombreFuncion {obtener}($p$.origen, $d$.CompYPaq).PaqYCam) \ComplejidadDer{L}
		\If {$\neg$ \NombreFuncion {def?}(p.prioridad, $diccprio$)}	\ComplejidadDer{log_2(s)}
			\State \NombreFuncion {Definir}($p$.prioridad,\NombreFuncion {Agregar}(\NombreFuncion {Vacio}(), $p$), $diccprio$)	\ComplejidadDer{log_2(s)}
		\Else 
			\State \NombreFuncion {Definir}($p$.prioridad,\NombreFuncion{Agregar}(\NombreFuncion {obtener}($p$.prioridad, $diccprio$), $p$), $diccprio$	\ComplejidadDer{log_2(s)}
		\EndIf
		
		\State \NombreFuncion{definir}($p$, $dicccam$, \NombreFuncion {AgregarAtras}(<>,p.origen)	\ComplejidadDer{log_2(k)}
		
		
		}
	{\Complejidad{L+ log_2(k)}}
	{}
	\end{Algoritmo}
	
	
	
	\begin{Algoritmo}{iAvanzarSegundo}{\Inout {d}{dcnet}}{}
	{
		\State var $it$ $\gets$ \NombreFuncion {computadoras}(red) \ComplejidadDer{1}
		\State var aux $\gets$ \NombreFuncion {Vacia}()	\ComplejidadDer{1}
		\While {\NombreFuncion {HaySiguiente}($it$)}	\ComplejidadDer{N}
		
			\State var $diccprio$: diccRapido $\gets$ \NombreFuncion {obtener}( \NombreFuncion{Siguiente}($it$).id, $d$.CompYPaq).MasPriori)	\ComplejidadDer{L}
		\State var $dicccam$: diccRapido $\gets$ \NombreFuncion {obtener}(\NombreFuncion{Siguiente}($it$).id, $d$.CompYPaq).PaqYCam) \ComplejidadDer{L}
			\If {$\neg$ \NombreFuncion {Vacio?}($diccprio$)} \ComplejidadDer{1}
				\State var $paq$: paquete $\gets$ \NombreFuncion{Primero}(\NombreFuncion {obtener}(\NombreFuncion {DameMax}($diccprio$), $diccprio$)) \ComplejidadDer{log_2(k)+1+1}
				\State \NombreFuncion {AgregarAdelante}(aux, tupla(paq: $paq$,pcant: $it$.id, camrecorrido: \NombreFuncion{obtener}($paq$, $dicccam$)) \ComplejidadDer{1+ log_2(k)}
			
				\State \NombreFuncion {Eliminar}(\NombreFuncion{obtener}(\NombreFuncion {DameMax}($diccprio$), $diccprio$), $paq$) \ComplejidadDer{log_2(k)+log_2(k)+1}%elimino el primero entonces es o 1( porque paq era el primero)
				\If {\NombreFuncion{EsVacio?}(\NombreFuncion{obtener}(\NombreFuncion {DameMax}($diccprio$), $diccprio$)} \ComplejidadDer{log_2(k)}
					\State \NombreFuncion{borrar}(\NombreFuncion {DameMax}($diccprio$), $diccprio$) \ComplejidadDer{log_2(k)}
				\EndIf
				\State \NombreFuncion{borrar}($paq$, $dicccam$) \ComplejidadDer{log_2(k)}
				
				
				\State	\NombreFuncion {obtener}(\NombreFuncion{Siguiente}($it$).id, $d$.CompYPaq).Enviados ++ \ComplejidadDer{L}
					\If {\NombreFuncion {obtener}(\NombreFuncion{Siguiente}($it$).id, $d$.CompYPaq).Enviados > ($d$.MasEnviante).enviados }	\ComplejidadDer{L+1}
						\State $d$.MasEnviante $\gets$ tupla(\NombreFuncion{Siguiente}($it$), \NombreFuncion {obtener}(\NombreFuncion{Siguiente}($it$).id, $d$.CompYPaq).Enviados )	\ComplejidadDer{L+1}
					\EndIf
							
				\EndIf
				
				\State \NombreFuncion{Avanzar}($it$) \ComplejidadDer{1}
		\EndWhile
		
		\State var $itaux$ $\gets$ \NombreFuncion{crearIt}(aux) \ComplejidadDer{1}
		\While {\NombreFuncion{HaySiguiente}($itaux$)} \ComplejidadDer{Nk}
		
			\State var $proxpc$: compu $\gets$ \NombreFuncion{Primero}(\NombreFuncion{Siguiente}(\NombreFuncion{CaminosMinimos}($d$.red, $itaux$.pcant, $itaux$.destino)) \ComplejidadDer{L_1 + L_2}
			
			\State var $diccprio$: diccRapido $\gets$ \NombreFuncion {obtener}( $proxpc$.id, $d$.CompYPaq).MasPriori)	\ComplejidadDer{L}
		\State var $dicccam$: diccRapido $\gets$ \NombreFuncion {obtener}($proxpc$.id, $d$.CompYPaq).PaqYCam) \ComplejidadDer{L}
			
			
			\If {\NombreFuncion{def?}(($itaux$.paq).prioridad, $diccprio$)}	\ComplejidadDer{log_2(k)}
			
			\State 	var $mismaprio$: conj(paquetes) $\gets$ \NombreFuncion{Agregar}(\NombreFuncion {obtener}($it3$.paq.prioridad, $diccprio$), $it3$.paq)	\ComplejidadDer{log_2(k)}
			\State {\NombreFuncion{definir}(($it3$.paq).prioridad, $mismaprio$, $diccprio$)} \ComplejidadDer{log_2(k)}
		
		
		 	\Else 
		 		\State \NombreFuncion{Definir}($it3$.prioridad, \NombreFuncion{Agregar}(\NombreFuncion{Vacio}(),$it3$.paq), $diccprio$)\ComplejidadDer{log_2(k)}
			\EndIf
		
			\State \NombreFuncion{definir}($p$.paq,\NombreFuncion{AgregarAtras}($it3$.camrecorrido ,$proxpc$), $dicccam$)	\ComplejidadDer{log_2(k)}
			\State \NombreFuncion{EliminarSiguiente}($it3$) \ComplejidadDer{1}
			\State \NombreFuncion{Avanzar}($it3$) \ComplejidadDer{1}
		\EndWhile
		
		}
	{\Complejidad{N*(L+log_2(k)}}
	{}
	\end{Algoritmo}
	
	
	\begin{Algoritmo}{iPaqueteEnTransito?}{\In {d}{dcnet}, \In {p}{paquete}}{bool}
	{
		\State var $it$  $\gets$ \NombreFuncion {crearIt}(\NombreFuncion {computadoras}(d.red) \ComplejidadDer{1}
		\State var $esta$: bool $\gets$ false \ComplejidadDer{1}
		\While{\NombreFuncion{HaySiguiente}($it$) $\wedge$ $\neg$ $esta$ }\ComplejidadDer{}
			\State $esta$ $\gets$ \NombreFuncion{def?}(\NombreFuncion{obtener}(d.CompYPaq,$i$.id).PaqYCam , $p$) \ComplejidadDer{log_2(k)}
			\State \NombreFuncion{Avanzar}($it$)\ComplejidadDer{1}
		\EndWhile
			\State $res$ $\gets$ $esta$	\ComplejidadDer{1}
	
	
	}
	{\Complejidad{N*log(k)}}
	{}
	\end{Algoritmo}
	
	
	
	\begin{Algoritmo}{iLaQueMasEnvio}{\In {d}{dcnet}}{compu}
	{
		\State $res$  $\gets$ ($d$.MasEnviante).compu \ComplejidadDer{1}
	}
	{\Complejidad{1}}
	{}
	\end{Algoritmo}
	
	
	
	
	

	%\Algoritmo{IENTRAR}{\In {ts}{conj(tags)}, \In {e}{estaci�n}, \Inout {c}{ciudad}}{}{
		%\State e\_ciudad.cRUR $\gets$ e\_ciudad.cRUR + 1 \ComplejidadDer{1}
		%\State var sendas $\gets$ EVALUARSENDAS(ts, e\_ciudad.m) \ComplejidadDer{S \cdot R}
		%\State var nRUR $\gets$ tupla(e\_ciudad.RURs, e, 0, ts, sendas) \ComplejidadDer{1}
		%\State AGREGARATRAS(e\_ciudad.RURs, nRUR) \ComplejidadDer{1}
		%\State obtEst = \&(SIGNIFICADO(e.RUREnEst, e)) \ComplejidadDer{|e|}
		%\State ENCOLAR(obtEst, tupla(e\_ciudad.cRUR, 0)) \ComplejidadDer{log\ N}
	%}{\Complejidad{|e| + S \cdot R + log\ N}}{
	%Justificacion.
	%}
	%
	%\newpage
	%
	%\Algoritmo{IMOVER}{\In {u}{nat}, \In {e2}{estaci�n}, \Inout {c}{ciudad}}{}{
		%\State var est $\gets$ OBTENERCOLAEST(c.RUREnEst, c.RURs[u].e) \ComplejidadDer{|e1|}
		%\State SACARRUR(est, u) \ComplejidadDer{log\ N_{e1}}
		%\State INFRACCIONO?(u, NROCONEXION(c.RURs[u].e, e2) \ComplejidadDer{|e1| + |e2|}
		%\State MODIFICARRUR(u) \ComplejidadDer{1}
		%\State est $\gets$ OBTENERCOLAEST(c.RUREnEst, e2) \ComplejidadDer{|e2|}
		%\State METERRUR(est, u) \ComplejidadDer{log\ N_{e2}}
	%}{\Complejidad{|e1| + |e2| + log\ N_{e1} + log\ N_{e2}}}{
		%$\Complejidad{|e1|} + \Complejidad{|e2|} + \Complejidad{log\ N_{e1}} + 
		%\Complejidad{log\ N_{e2}} + \Complejidad{|e1| + |e2|} + \Complejidad{1} =$\\
		%$2 * \Complejidad{|e1| + |e2|} + \Complejidad{log\ N_{e2} + log\ N_{e1}} =$\\
		%$\Complejidad{|e1| + |e2|} + \Complejidad{log\ N_{e2} + log\ N_{e1}} =$\\
		%$\Complejidad{|e1| + |e2| + log\ N_{e2} + log\ N_{e1}}$
	%}

\end{Algoritmos}

\newpage
\section{Diccionario String($\alpha$)}

\begin{Interfaz}

  \textbf{se explica con}: \tadNombre{Diccionario(String, $\alpha$)}.

  \textbf{g�neros}: \TipoVariable{diccString($\alpha$)}.

  \Titulo{Operaciones b�sicas de Diccionario String($\alpha$)}
  
  \InterfazFuncion{Def?}{\In{clv}{string}, \In{d}{diccString($\alpha$)}}{bool}
  [true]
  {$res \igobs$ def?($clv, d$)}
  [$\Complejidad{|clv|}$]
  [Revisa si la clave ingresada se encuentra definida en el Diccionario.]
  
  \InterfazFuncion{Obtener}{\In{clv}{string}, \In{d}{diccString($\alpha$)}}{$\alpha$}
  [def?($clv, d$)]
  {$res \igobs$ obtener($clv, d$)}
  [$\Complejidad{|clv|}$]
  [Devuelve el significado de la clave.]

  \InterfazFuncion{Vac�o}{}{diccString($\alpha$)}
  [true]
  {$res \igobs$ vac�o()}
  [\Complejidad{1}]
  [Crea nuevo diccionario vacio.]
	
  \InterfazFuncion{Definir}{\In{clv}{string}, \In{def}{$\alpha$}, \Inout{d}{diccString($\alpha$)}}{}
  [$d$ $\igobs$ $d_0$]
  {$d$ $\igobs$ definir($clv, def, d_0$)}
  [$\Complejidad{|clv|}$]
  [Agrega un nueva definicion.]
	
  \InterfazFuncion{Borrar}{\In{clv}{string}, \Inout{d}{diccString($\alpha$)}}{}
  [$d$ $\igobs$ $d_0$ $\wedge$ def?(clv, d)]
  {$res$ $\igobs$ borrar($k, d_0$)}
  [$\Complejidad{|clv|}$]
  [Devuelve la definicion correspondiente a la clave.]
	
\end{Interfaz}


\newpage
\section{DiccRapido}

\begin{Interfaz}
  
  \textbf{se explica con}: \tadNombre{Diccionario(clave, significado)}.

  \textbf{g�neros}: \TipoVariable{diccRapido}.

  \Titulo{Operaciones b�sicas de DICCRAPIDO}

  \InterfazFuncion{Def?}{\In{c}{clave}, \In{d}{diccRapido}}{bool}
  [true]
  {$res$ $\igobs$ def?($c, d$)}
  [\Complejidad{log_2\ n}, siendo n la cantidad de claves]
  [Verifica si una clave est� definida.]
  
 \InterfazFuncion{Obtener}{\In{c}{clave}, \In{d}{diccRapido}}{significado}
 [def?($c, d$)]
 {$res$ $\igobs$ obtener($c, d$)}
 [$\Complejidad{log_2\ n}$, siendo n la cantidad de claves]
 [Devuelve el significado asociado a una clave]
  
 \InterfazFuncion{Vac�o}{}{diccRapido}
 [true]
 {$res$ $\igobs$ vac�o()}
 [$\Complejidad{1}$]
 [Crea un nuevo diccionario vac�o]  
  
 \InterfazFuncion{Definir}{\In{c}{clave}, \In{s}{significado}, \Inout{d}{diccRapido}}{}
 [$d$ $\igobs$ $d_0$]
 {$d$ $\igobs$ definir($c, s, d_0$)}
 [$\Complejidad{log_2\ n}$, siendo n la cantidad de claves]
 [Define la clave, asociando su significado, al diccionario]
  
 \InterfazFuncion{Borrar}{\In{c}{clave}, \Inout{d}{diccRapido}}{}
 [$d$ $\igobs$ $d_0$ $\wedge$ def?($c, d_0$)]
 {$d$ $\igobs$ borrar($c, d_0$)}
 [$\Complejidad{log_2\ n}$, siendo n la cantidad de claves]
 [Borra la clave del diccionario]

 \InterfazFuncion{Claves}{\In{d}{diccRapido}}{itPaquete}
 [true]
 {$res$ $\igobs$ claves($d$)}
 [$\Complejidad{1}$]
 [Devuelve un iterador de paquete]  
  
\end{Interfaz}

\subsection{Representacion}

\begin{Representacion}

\end{Representacion}

\subsection{Algoritmos}

\begin{Algoritmos}

\end{Algoritmos}






\end{document}
