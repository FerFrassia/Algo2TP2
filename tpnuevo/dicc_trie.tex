\section{Diccionario String}

\subsection{Interfaz}

\begin{Interfaz}

 \textbf{par�metros formales}: string, $\beta$;

  \textbf{se explica con}: \tadNombre{Diccionario(clave, significado)}.

  \textbf{g�neros}: \TipoVariable{diccString(string, $\beta$)}.

  \Titulo{Operaciones b�sicas de Diccionario String(string, $\beta$)}
  
  \InterfazFuncion{Def?}{\In{c}{string}, \In{d}{diccString(string, $\beta$)}}{bool}
  [true]
  {$res \igobs$ def?($c, d$)}
  [$\Complejidad{|c|}$]
  [Revisa si la clave ingresada se encuentra definida en el diccionario]

  \InterfazFuncion{Vac�o}{}{diccString(string, $\beta$)}
  [true]
  {$res \igobs$ vac�o()}
  [\Complejidad{1}]
  [Crea nuevo diccionario vac�o]
	
  \InterfazFuncion{Obtener}{\In{c}{string}, \In{d}{diccString(string, $\beta$)}}{$\beta$}
  [def?($c, d$)]
  {$res \igobs$ obtener($c, d$)}
  [$\Complejidad{|c|}$]
  [Devuelve el significado correspondiente a la clave]
  [Res es modificable si y s�lo si d es modificable]

  \InterfazFuncion{Definir}{\In{c}{string}, \In{s}{$\beta$}, \Inout{d}{diccString(string, $\beta$)}}{}
  [$d$ $\igobs$ $d_0$]
  {$d$ $\igobs$ definir($c, s, d_0$)}
  [$\Complejidad{|c|}$]
  [Define la clave, asociando su significado, al diccionario]
  [Los elementos c y s se pasan por referencia]
	
  \InterfazFuncion{Borrar}{\In{c}{string}, \Inout{d}{diccString(string, $\beta$)}}{}
  [$d$ $\igobs$ $d_0$ $\wedge$ def?($c, d_0$)]
  {$res$ $\igobs$ borrar($c, d_0$)}
  [$\Complejidad{|c|}$]
  [Borra la clave (y su significado) del diccionario]
  
  \vspace{2em}
  La representaci�n y los algoritmos del m�dulo no hacen falta explicitarlos porque est�n dados por la c�tedra. 
  \\ De todas formas, cabe recordar que este m�dulo se representa sobre un Trie.
	
\end{Interfaz}
